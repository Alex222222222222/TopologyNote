\documentclass[twoside]{book}

\usepackage[a4paper]{geometry}

\usepackage{amssymb}
\usepackage{amsmath}
\usepackage{amsthm}
\usepackage{amsfonts}

\usepackage{multicol}

% see https://tex.stackexchange.com/a/39755
% Simulated package file
\begin{filecontents}{envcode.sty}
      \newcommand\NewEnvCode[2]{%
       \expandafter\def\csname code@#1\endcsname{#2}%
       \expandafter\def\csname change@code@#1\endcsname{%
        \expandafter\let\expandafter\Code\csname code@#1\endcsname
       }%
      }
      
      \newcommand\MakeDefault{%
       \expandafter\let\expandafter\code@@default\csname code@\@currenvir\endcsname
      }
      
      \newcommand\RunEnvCode{%
       \let\Code=\code@@default
       \csname change@code@\@currenvir\endcsname
       \Code
      }
      
      \AtBeginDocument{\MakeDefault}
\end{filecontents}

\usepackage{envcode}

% for cross reference
% \usepackage{hyperref}

% for fixed length of cell of tables
% \usepackage{tabularx}
% \usepackage{longtable}

\newtheorem{definition}{Definition}[section]
\newtheorem{lemma}{Lemma}[section]
\newtheorem{theorem}{Theorem}[section]
\newtheorem{corollary}{Corollary}[section]
\newtheorem{proposition}{Proposition}[section]

\NewEnvCode{document}{default code}
\NewEnvCode{theorem}{ theorem }
\NewEnvCode{lemma}{ lemma }
\NewEnvCode{corollary}{ corollary }
\NewEnvCode{proposition}{ proposition }
\NewEnvCode{definition}{ definition }

\title{Topology Note}
\author{Alex}

\newcommand{\textmathbb}[1]{ $ \mathbb{#1} $ }
\newcommand{\defineNewWord}[1]{\textit{\textbf{#1}}}
\newcommand{\omitObviuos}{\footnote{We omit the proof of this \RunEnvCode as it is obvious.}}
\newcommand{\mt}[1]{ $ #1 $ }
\newcommand{\tmb}[1]{\textmathbb{#1}}
\newcommand{\mtb}[1]{\textmathbb{#1}}
\newcommand{\productSet}[2]{ $ #1 \times #2 $ }
\newcommand{\closure}[1]{\overline{#1}}
\newcommand{\mclosure}[1]{ $ \overline{#1} $ }
\newcommand{\boundary}[1]{\text{Bd}#1}
\newcommand{\interior}[1]{\text{Int}#1}
\newcommand{\norm}[1]{||#1||}

% commonly used definitions
\newcommand{\topology}{topology}
\newcommand{\basis}{basis}
\newcommand{\topological}{topological}
\newcommand{\closed}{closed}
\newcommand{\hausdorff}{Hausdorff}
\newcommand{\Hausdorff}{Hausdorff}
\newcommand{\intersection}{intersection}
\newcommand{\union}{union}
\newcommand{\arbitrary}{arbitrary}
\newcommand{\intersect}{intersect}
\newcommand{\ioi}{ if and only if }
\newcommand{\element}{element}
\newcommand{\neighbourhood}{neighbourhood}
\newcommand{\converge}{converge}
\newcommand{\induction}{induction}
\newcommand{\continuous}{continuous}
\newcommand{\homeomorphism}{homeomorphism}
\newcommand{\equivalent}{equivalent}
\newcommand{\imbedding}{imbedding}
\newcommand{\diam}{\mathrm{diam}}

\begin{document}
      \maketitle

      \newpage

      \tableofcontents

      \newpage

      \section*{Definitions}

\begin{multicols}{2}

\vspace{1em}\large{\textbf{B}}

basis, \pageref{def:Basis}

boundary, \pageref{def:Boundary}

\vspace{1em}\large{\textbf{C}}

closed, \pageref{def:Closed}

\hspace{1em}closed in, \pageref{def:ClosedIn}

closure, \pageref{def:Closure}

coarser, \pageref{def:Comparable}

\hspace{1em}strictly coarser, \pageref{def:Comparable}

finer, \pageref{def:Comparable}

\hspace{1em}strictly finer, \pageref{def:Comparable}

larger, \pageref{def:Comparable}

\hspace{1em}strictly larger, \pageref{def:Comparable}

smaller, \pageref{def:Comparable}

\hspace{1em}strictly smaller, \pageref{def:Comparable}

continuous, \pageref{def:Continuous}

\hspace{1em}continuous relative to, \pageref{def:ContinuousRelativeTo}

converge, \pageref{def:Converge}

convex, \pageref{def:Convex}

coordinate functions, \pageref{def:CoordinateFunctions}

\vspace{1em}\large{\textbf{D}}

diagonal, \pageref{def:Diagonal}

discrete topology, \pageref{def:DiscreteTopology}

\vspace{1em}\large{\textbf{F}}

finite complement topology, \pageref{def:FiniteComplementTopology}

\vspace{1em}\large{\textbf{H}}

Hausdorff space, \pageref{def:HausdorffSpace}

homeomorphism, \pageref{def:Homeomorphism}

\vspace{1em}\large{\textbf{I}}

interior, \pageref{def:Interior}

intersect, \pageref{def:Intersect}

interval, \pageref{def:Interval}

\hspace{1em}closed interval, \pageref{def:Interval}

\hspace{1em}half-open interval, \pageref{def:Interval}

\hspace{1em}open interval, \pageref{def:Interval}

\vspace{1em}\large{\textbf{K}}

K-topology on R, \pageref{def:KTopologyOnTheRealLine}

\vspace{1em}\large{\textbf{L}}

limit, \pageref{def:Limit}

cluster point, \pageref{def:LimitPoint}

limit point, \pageref{def:LimitPoint}

point of accumulation, \pageref{def:LimitPoint}

lower limit topology on R, \pageref{def:LowerLimitTopologyOnTheRealLine}

\vspace{1em}\large{\textbf{N}}

neighbourhood, \pageref{def:Neighbourhood}

\vspace{1em}\large{\textbf{O}}

open map, \pageref{def:OpenMap}

open set, \pageref{def:OpenSet}

open sets, \pageref{def:OpenSets}

ordered square, \pageref{def:OrderedSquare}

order topology, \pageref{def:OrderTopology}

\vspace{1em}\large{\textbf{P}}

product topology, \pageref{def:ProductTopology}

projection, \pageref{def:Projection}

\vspace{1em}\large{\textbf{R}}

ray, \pageref{def:Ray}

\hspace{1em}closed ray, \pageref{def:Ray}

\hspace{1em}open ray, \pageref{def:Ray}

\vspace{1em}\large{\textbf{S}}

standard topology on R, \pageref{def:StandardTopologyOnTheRealLine}

subbasis, \pageref{def:Subbasis}

subspace, \pageref{def:SubspaceTopology}

subspace topology, \pageref{def:SubspaceTopology}

\vspace{1em}\large{\textbf{T}}

\mt{T_{1}} axiom, \pageref{def:T1Axiom}

topological imbedding, \pageref{def:TopologicalImbedding}

topology, \pageref{def:Topology}

topology generated by basis, \pageref{def:TopologyGeneratedByBasis}

topology space, \pageref{def:TopologySpace}

trivial topology, \pageref{def:TrivialTopology}

\end{multicols}

      \newpage

      \section*{Theorems}

\vspace{1em}\noindent\large{\textbf{C}}

Comparison of the box and product topologies, \pageref{theorem:ComparisonOfBoxProductTopology}

\vspace{1em}\noindent\large{\textbf{M}}

Maps into products, \pageref{theorem:MapsIntoProducts}

\vspace{1em}\noindent\large{\textbf{R}}

Rules for constructing continuous functions, \pageref{theorem:RulesForConstructingContinuousFunctions}

\vspace{1em}\noindent\large{\textbf{T}}

The pasting lemma, \pageref{theorem:ThePastingLemma}

The sequence lemma, \pageref{theorem:TheSequenceLemma}

\vspace{1em}\noindent\large{\textbf{U}}

Uniform limit theorem, \pageref{theorem:UniformLimitTheorem}



      \newpage

      \chapter{Topology Spaces and Continuous Function}

      \section{Continuous Function}

\begin{definition}[continuous]\label{def:Continuous}\footnote{
      As the continuity of a function is different as the topological spaces are different. So if we want to emphasis this fact, we say that \mt{f} is continuous \defineNewWord{relative}\label{def:ContinuousRelativeTo} to specific topologies on \mtb{X} and \mtb{Y}.
}
      Let \mtb{X} and \mtb{Y} be topological spaces. A function \mt{f: \mathbb{X}\rightarrow \mathbb{Y}} is said to be \defineNewWord{continuous} if for each open subset \mt{V} of \tmb{Y}, the set \mt{f^{-1}(V)} is an open subset of \mtb{X}.
\end{definition}

\begin{theorem}
      Let \mtb{X} and \mtb{Y} be topological spaces; let \mt{f: \mathbb{X}\rightarrow\mathbb{Y}}. Then the following are equivalent.
      \begin{enumerate}
            \item \mt{f} is continuous.
            \item For every subset \mt{A} of \mt{X}, one has \mt{f(\closure{A})\subseteq\closure{f(A)}}.
            \item For every closed set \mt{B} of \mtb{Y}, the set \mt{f^{-1}(B)} is closed in \mtb{X}.
            \item For each \mt{x\in\mathbb{X}} and each neighbourhood of \mt{V} of \mt{f(x)}, there is a neighbourhood \mt{U} of \mt{x} such that \mt{f(U) \subseteq V}.
      \end{enumerate}
\end{theorem}

\begin{proof}

      \hspace{1em}

      1 \mt{\Rightarrow} 3:

      Let \mt{A} be a open set in \mtb{Y}. \mt{f^{-1}(\mathbb{Y}-A) = \mathbb{X} - f^{-1}(A)}.
      
      \vspace{1em}

      3 \mt{\Rightarrow} 1:

      Let \mt{A} be a closed set in \mtb{Y}. \mt{f^{-1}(\mathbb{Y}-A) = \mathbb{X} - f^{-1}(A)}.

      \vspace{1em}

      1 \mt{\Rightarrow} 2:

      For \mt{x \in \closure{A}}, we take a open set \mt{f(x) \in U \subseteq \mathbb{Y}}. Thus \mt{x \in f^{-1}(U) \cap A \neq \emptyset}. Thus \mt{U \cap f(A) \neq \emptyset}. So \mt{f(x) \in \closure{f(A)}}. Thus \mt{f(\closure{A})\subseteq\closure{f(A)}}.

      \vspace{1em}

      2 \mt{\Rightarrow} 3:

      Suppose \mt{f} is not continuous. Then there must exists \mt{V}, such that \mt{f^{-1}(V) = U} is not closed. Thus \mt{\closure{U} \supset B = f^{-1}(A)}. Thus \mt{f{\closure{B}} \supset A}. However \mt{f(\closure{B}) \subseteq \closure{f(B)} = A}. There is a contradiction. So \mt{f} must be continuous.

      \vspace{1em}

      1 \mt{\Rightarrow} 4:

      For every neighbourhood \mt{V} of \mt{f(x)}, \mt{f^{-1}(V)} is a neighbourhood of \mt{x} that \mt{f(f^{-1}(V)) \subseteq V}.

      \vspace{1em}

      4 \mt{\Rightarrow} 1:

      We take a open set \mt{V} of \mtb{Y}. Let \mt{S} be the collection of all open set \mt{U} in \mtb{X} such that \mt{f(U) \subseteq V}. The set cannot be empty unless \mt{f^{-1}(V) = \emptyset}. Let \mt{U_{0}} denote the union of all the element in \mt{S}. We prove that \mt{U_{0} = f^{-1}(V)}.

      For all element \mt{x \in U_{0}}, \mt{f(x) \in V}. Thus \mt{U_{0} \subseteq f^{-1}(V)}.

      For all element \mt{x \in f^{-1}(V)}. There is a \mt{U'} such that \mt{x \in U', f(U')\subseteq V}. This follows from the condition 4. Thus \mt{U' \in S}. Thus \mt{x \in U_{0}}. Thus \mt{U_{0} \subseteq f^{-1}(V)}. As \mt{U_{0}} is union of open set, \mt{U_{0}} is also open. Thus, \mt{f^{-1}(V)} is also open.
      
      Thus \mt{f} is continuous.
\end{proof}

\begin{definition}[homeomorphism]\label{def:Homeomorphism}\footnote{
      A equivalent way to define homeomorphism, is that for any open subset \mt{U} of \mtb{X}, \mt{f(U)} is open \ioi \mt{U} is open.
}
      Let \mtb{X} and \mtb{Y} be topological space; let \mt{f: \mathbb{X} \rightarrow \mathbb{Y}} be a bijection. If both the function \mt{f} and the inverse function
      \begin{equation*}
            f^{-1}: \mathbb{Y} \rightarrow \mathbb{X}
      \end{equation*}
      are continuous, then f is called a \defineNewWord{homeomorphism}
\end{definition}

\begin{definition}[topological imbedding]\label{def:TopologicalImbedding}
      Suppose that \mt{f: \mathbb{X} \rightarrow \mathbb{Y}} is an injective continuous map, where \mtb{X} and \mtb{Y} are topological spaces. Let \mtb{Z} be the image set \mt{f(\mathbb{X})}, considered as a subspace of \mtb{Y}; then the function \mt{f': \mathbb{X} \rightarrow \mathbb{Z}} obtained by restricting the range of \mt{f} is bijective. If \mt{f'} happens to be a homeomorphism of \mtb{X} with \mtb{Z}, we say that the map \mt{f: \mathbb{X} \rightarrow \mathbb{Y}} is a \defineNewWord{topological imbedding}, or simply an \defineNewWord{imbedding}, of \mtb{X} in \mtb{Y}.
\end{definition}

\begin{theorem}[Rules for constructing continuous functions]\label{theorem:RulesForConstructingContinuousFunctions}
      Let \mtb{X}, \mtb{Y}, and \mtb{Z} be topological spaces.
      \begin{enumerate}
            \item (Constant function) If \mt{f: \mathbb{X} \rightarrow \mathbb{Y}} maps all of \mtb{X} into the single point \mt{y_{0}} of \mtb{Y}, then \mt{f} is continuous.
            \item (Inclusion) If \mt{A} is a subspace of \mtb{X}, the inclusion function \mt{j: A \rightarrow \mathbb{X}} is continuous.
            \item (Composites) If \mt{f: \mathbb{X}\rightarrow \mathbb{Y}} and \mt{g:\mathbb{Y}\rightarrow\mathbb{Z}} are continuous, then the map \mt{g \circ f: \mathbb{X} \rightarrow \mathbb{Z}} is continuous.
            \item (Restricting the domain) If \mt{f: \mathbb{X} \rightarrow \mathbb{Y}} is continuous, and if \mt{A} is a subspace of \mtb{X}, then the restriction function \mt{f|A : A \rightarrow \mathbb{Y}} is continuous.
            \item (Restricting or expanding the range) Let \mt{f:\mathbb{X}\rightarrow\mathbb{Y}} is continuous. Let \mtb{Z} be a subspace of \mtb{Y} containing the image \mt{f(\mathbb{X})}, the function \mt{h: \mathbb{X}\rightarrow \mathbb{Z}} obtained by restricting the range of \mt{f} is continuous. If \mtb{Z} is a space having \mtb{Y} as a subspace, then the function \mt{h: \mathbb{X}\rightarrow\mathbb{Y}} obtained by expanding the range of \mt{f} is continuous.
            \item (Local formulation of continuity) The map \mt{f: \mathbb{X}\rightarrow\mathbb{Y}} is continuous if \mtb{X} can be written as the union of open sets \mt{U_{\alpha}} such set \mt{f|U_{\alpha}} is continuous for each \mt{\alpha}
      \end{enumerate}
\end{theorem}

\begin{proof}

      \hspace{1em}

      \begin{enumerate}
            \item \mt{f^{-1}(U)} of any open set \mt{U} is \mtb{X}, thus \mt{f} is continuous.
            \item For every open subset \mt{U} of \mtb{X}, \mt{j^{-1}(U) = U\cap A} is continuous in \mt{A}. Thus \mt{j} is a continuous function.
            \item For every open subset \mt{U} of \mtb{Z}, \mt{f^{-1}(U)} is open in \mtb{Y}, and \mt{g^{-1}(f^{-1}(U))} is open in \mtb{X}. Thus, \mt{g \circ f} is continuous
            \item For every open subset \mt{U} of \mtb{Y}, \mt{f^{-1}(U)} is open in \mtb{X}, thus \mt{f^{-1}(U)\cap A} is open in \mt{A}. Thus the function \mt{f|A} is continuous.
            \item If \mtb{Z} is a subspace of \mtb{Y}, then every open subset of \mtb{Z} can be represented as \mt{U\cap\mathbb{Z}}, where \mt{U} is a open subset of \mtb{Y}. Thus \mt{h^{-1}(U\cap\mathbb{Z})=g^{-1}(\mathbb{Z})\cap g^{-1}(U) = \mathbb{X}\cap g^{-1}(U)} which is a open subset of \mt{X}, thus \mt{h} is continuous.
            
            If \mtb{Y} is a subspace of \mtb{Z}. Then we take a open subset \mt{U} of \tmb{Z}. \mt{h^{-1}(U) = g^(-1)(U\cap \mathbb{Y}) } which is open in \mtb{X}, thus \mt{h} is continuous.

            \item if \mt{f|U_{\alpha}} is continuous for each \mt{\alpha}. For every open subset \mt{U} of \mtb{Y}.
            \begin{equation*}
                  U = \cup_{\alpha} (U_{\alpha}\cap U)
            \end{equation*}
            where \mt{U_{\alpha}\cap U} is open both in \mt{U_{\alpha}} and in \mtb{Y}. 

            Thus,
            \begin{eqnarray*}
                  f^{-1}(U) &=& f^{-1}(\cup_{\alpha} (U_{\alpha}\cap U)) \\
                  &=& \cup_{\alpha} ((f|U_{\alpha})^{-1}(U_{\alpha}\cap U))
            \end{eqnarray*}
            and each \mt{(f|U_{\alpha})^{-1}(U_{\alpha}\cap U)} is open, thus \mt{f^{-1}(U)} is open.
      \end{enumerate}
\end{proof}

\begin{theorem}[The pasting lemma]\label{theorem:ThePastingLemma}\footnote{
      The proof of this theorem is similar to the "Local formulation of continuity" condition of "Rules for constructing continuous functions", so we omit the proof here.
}
      Let \mt{
            \mathbb{X} = A \cup B
      }, where \mt{A,B} are closed in \mtb{X}. Let \mt{f: A\rightarrow \mathbb{Y}} and \mt{g: B \rightarrow \mathbb{Y}} be continuous. If \mt{f(x)=g(x)} for every \mt{x\in A\cap B}, then \mt{f,g} combine to give a continuous function \mt{h: \mathbb{X}\rightarrow\mathbb{Y}}, defined by setting \mt{h(x)=f(x),x\in A} and \mt{h(x)=g(x),x\in B}.
\end{theorem}

\begin{theorem}[Maps into products]\label{theorem:MapsIntoProducts} \footnote{
      The map \mt{f_{1},f_{2}} are called the \defineNewWord{coordinate functions}\label{def:CoordinateFunctions} of \mt{f}
}
      Let \mt{f: A \rightarrow \mathbb{X}\times\mathbb{Y}} be given by the equation
      \begin{equation*}
            f(a) = (f_{1}(a),f_{2}(a))
      \end{equation*}

      Then, the function \mt{f} is continuous \ioi the functions
      \begin{equation*}
            f_{1}: A \rightarrow \mathbb{X}, f_{2}: A \rightarrow \mathbb{Y}
      \end{equation*}
      are continuous.
\end{theorem}

\begin{proof}
      Let \mt{\pi_{1},\pi_{2}} be the projection function
      \begin{eqnarray*}
            \pi_{1}&:& \mathbb{X}\times\mathbb{Y} \rightarrow \mathbb{X} \\
            \pi_{2}&:& \mathbb{X}\times\mathbb{Y} \rightarrow \mathbb{Y}
      \end{eqnarray*}

      \vspace{1em}

      We first proof that if \mt{U} is an open subset of \productSet{\mathbb{X}}{\mathbb{Y}},
      \begin{equation*}
            f^{-1}(U) = f_{1}^{-1}(\pi_{1}(U)) \cap f_{2}^{-1}(\pi_{2}(U))
      \end{equation*}

      Let \mt{x \times y \in U}, \mt{f^{-1}(x \times y)} contains all \mt{a} such that \mt{f(a) = x \times y}.

      Then for any \mt{a \in f^{-1}(x \times y)}, \mt{a \in f_{1}^{-1}(\pi_{1}(x \times y)), a \in f_{2}^{-1}(\pi_{2}(x \times y))}.

      Thus, \mt{f^{-1}(x \times y) \subseteq f_{1}^{-1}(\pi_{1}(x \times y)) \cap f_{2}^{-1}(\pi_{2}(x \times y))}.

      Thus \mt{f^{-1}(U) \subseteq f_{1}^{-1}(\pi_{1}(U)) \cap f_{2}^{-1}(\pi_{2}(U))}.

      \vspace{1em}

      Also, if \mt{a \in f_{1}^{-1}(\pi_{1}(x \times y)), a \in f_{2}^{-1}(\pi_{2}(x \times y))}, \mt{f_{1}(a) = x, f_{2}(a) = y}.

      Thus \mt{f(a) = x \times y}.
      Thus \mt{a \in f^{-1}(x \times y)}.

      Thus \mt{f^{-1}(U) = f_{1}^{-1}(\pi_{1}(U)) \cap f_{2}^{-1}(\pi_{2}(U))}

      \vspace{1em}

      Let \mt{U} be any open subset of \productSet{\mathbb{X}}{\mathbb{Y}}

      \begin{equation*}
            f^{-1}(U) = f_{1}^{-1}(\pi_{1}(U)) \cap f_{2}^{-1}(\pi_{2}(U))
      \end{equation*}

      Where \mt{f_{1}^{-1}(\pi_{1}(U))} and \mt{f_{2}^{-1}(\pi_{2}(U))} are both open set. Thus \mt{f^{-1}(U)} is open.
\end{proof}

      \section{Continuous Function}

\begin{definition}[continuous]\label{def:Continuous}\footnote{
      As the continuity of a function is different as the topological spaces are different. So if we want to emphasis this fact, we say that \mt{f} is continuous \defineNewWord{relative}\label{def:ContinuousRelativeTo} to specific topologies on \mtb{X} and \mtb{Y}.
}
      Let \mtb{X} and \mtb{Y} be topological spaces. A function \mt{f: \mathbb{X}\rightarrow \mathbb{Y}} is said to be \defineNewWord{continuous} if for each open subset \mt{V} of \tmb{Y}, the set \mt{f^{-1}(V)} is an open subset of \mtb{X}.
\end{definition}

\begin{theorem}
      Let \mtb{X} and \mtb{Y} be topological spaces; let \mt{f: \mathbb{X}\rightarrow\mathbb{Y}}. Then the following are equivalent.
      \begin{enumerate}
            \item \mt{f} is continuous.
            \item For every subset \mt{A} of \mt{X}, one has \mt{f(\closure{A})\subseteq\closure{f(A)}}.
            \item For every closed set \mt{B} of \mtb{Y}, the set \mt{f^{-1}(B)} is closed in \mtb{X}.
            \item For each \mt{x\in\mathbb{X}} and each neighbourhood of \mt{V} of \mt{f(x)}, there is a neighbourhood \mt{U} of \mt{x} such that \mt{f(U) \subseteq V}.
      \end{enumerate}
\end{theorem}

\begin{proof}

      \hspace{1em}

      1 \mt{\Rightarrow} 3:

      Let \mt{A} be a open set in \mtb{Y}. \mt{f^{-1}(\mathbb{Y}-A) = \mathbb{X} - f^{-1}(A)}.
      
      \vspace{1em}

      3 \mt{\Rightarrow} 1:

      Let \mt{A} be a closed set in \mtb{Y}. \mt{f^{-1}(\mathbb{Y}-A) = \mathbb{X} - f^{-1}(A)}.

      \vspace{1em}

      1 \mt{\Rightarrow} 2:

      For \mt{x \in \closure{A}}, we take a open set \mt{f(x) \in U \subseteq \mathbb{Y}}. Thus \mt{x \in f^{-1}(U) \cap A \neq \emptyset}. Thus \mt{U \cap f(A) \neq \emptyset}. So \mt{f(x) \in \closure{f(A)}}. Thus \mt{f(\closure{A})\subseteq\closure{f(A)}}.

      \vspace{1em}

      2 \mt{\Rightarrow} 3:

      Suppose \mt{f} is not continuous. Then there must exists \mt{V}, such that \mt{f^{-1}(V) = U} is not closed. Thus \mt{\closure{U} \supset B = f^{-1}(A)}. Thus \mt{f{\closure{B}} \supset A}. However \mt{f(\closure{B}) \subseteq \closure{f(B)} = A}. There is a contradiction. So \mt{f} must be continuous.

      \vspace{1em}

      1 \mt{\Rightarrow} 4:

      For every neighbourhood \mt{V} of \mt{f(x)}, \mt{f^{-1}(V)} is a neighbourhood of \mt{x} that \mt{f(f^{-1}(V)) \subseteq V}.

      \vspace{1em}

      4 \mt{\Rightarrow} 1:

      We take a open set \mt{V} of \mtb{Y}. Let \mt{S} be the collection of all open set \mt{U} in \mtb{X} such that \mt{f(U) \subseteq V}. The set cannot be empty unless \mt{f^{-1}(V) = \emptyset}. Let \mt{U_{0}} denote the union of all the element in \mt{S}. We prove that \mt{U_{0} = f^{-1}(V)}.

      For all element \mt{x \in U_{0}}, \mt{f(x) \in V}. Thus \mt{U_{0} \subseteq f^{-1}(V)}.

      For all element \mt{x \in f^{-1}(V)}. There is a \mt{U'} such that \mt{x \in U', f(U')\subseteq V}. This follows from the condition 4. Thus \mt{U' \in S}. Thus \mt{x \in U_{0}}. Thus \mt{U_{0} \subseteq f^{-1}(V)}. As \mt{U_{0}} is union of open set, \mt{U_{0}} is also open. Thus, \mt{f^{-1}(V)} is also open.
      
      Thus \mt{f} is continuous.
\end{proof}

\begin{definition}[homeomorphism]\label{def:Homeomorphism}\footnote{
      A equivalent way to define homeomorphism, is that for any open subset \mt{U} of \mtb{X}, \mt{f(U)} is open \ioi \mt{U} is open.
}
      Let \mtb{X} and \mtb{Y} be topological space; let \mt{f: \mathbb{X} \rightarrow \mathbb{Y}} be a bijection. If both the function \mt{f} and the inverse function
      \begin{equation*}
            f^{-1}: \mathbb{Y} \rightarrow \mathbb{X}
      \end{equation*}
      are continuous, then f is called a \defineNewWord{homeomorphism}
\end{definition}

\begin{definition}[topological imbedding]\label{def:TopologicalImbedding}
      Suppose that \mt{f: \mathbb{X} \rightarrow \mathbb{Y}} is an injective continuous map, where \mtb{X} and \mtb{Y} are topological spaces. Let \mtb{Z} be the image set \mt{f(\mathbb{X})}, considered as a subspace of \mtb{Y}; then the function \mt{f': \mathbb{X} \rightarrow \mathbb{Z}} obtained by restricting the range of \mt{f} is bijective. If \mt{f'} happens to be a homeomorphism of \mtb{X} with \mtb{Z}, we say that the map \mt{f: \mathbb{X} \rightarrow \mathbb{Y}} is a \defineNewWord{topological imbedding}, or simply an \defineNewWord{imbedding}, of \mtb{X} in \mtb{Y}.
\end{definition}

\begin{theorem}[Rules for constructing continuous functions]\label{theorem:RulesForConstructingContinuousFunctions}
      Let \mtb{X}, \mtb{Y}, and \mtb{Z} be topological spaces.
      \begin{enumerate}
            \item (Constant function) If \mt{f: \mathbb{X} \rightarrow \mathbb{Y}} maps all of \mtb{X} into the single point \mt{y_{0}} of \mtb{Y}, then \mt{f} is continuous.
            \item (Inclusion) If \mt{A} is a subspace of \mtb{X}, the inclusion function \mt{j: A \rightarrow \mathbb{X}} is continuous.
            \item (Composites) If \mt{f: \mathbb{X}\rightarrow \mathbb{Y}} and \mt{g:\mathbb{Y}\rightarrow\mathbb{Z}} are continuous, then the map \mt{g \circ f: \mathbb{X} \rightarrow \mathbb{Z}} is continuous.
            \item (Restricting the domain) If \mt{f: \mathbb{X} \rightarrow \mathbb{Y}} is continuous, and if \mt{A} is a subspace of \mtb{X}, then the restriction function \mt{f|A : A \rightarrow \mathbb{Y}} is continuous.
            \item (Restricting or expanding the range) Let \mt{f:\mathbb{X}\rightarrow\mathbb{Y}} is continuous. Let \mtb{Z} be a subspace of \mtb{Y} containing the image \mt{f(\mathbb{X})}, the function \mt{h: \mathbb{X}\rightarrow \mathbb{Z}} obtained by restricting the range of \mt{f} is continuous. If \mtb{Z} is a space having \mtb{Y} as a subspace, then the function \mt{h: \mathbb{X}\rightarrow\mathbb{Y}} obtained by expanding the range of \mt{f} is continuous.
            \item (Local formulation of continuity) The map \mt{f: \mathbb{X}\rightarrow\mathbb{Y}} is continuous if \mtb{X} can be written as the union of open sets \mt{U_{\alpha}} such set \mt{f|U_{\alpha}} is continuous for each \mt{\alpha}
      \end{enumerate}
\end{theorem}

\begin{proof}

      \hspace{1em}

      \begin{enumerate}
            \item \mt{f^{-1}(U)} of any open set \mt{U} is \mtb{X}, thus \mt{f} is continuous.
            \item For every open subset \mt{U} of \mtb{X}, \mt{j^{-1}(U) = U\cap A} is continuous in \mt{A}. Thus \mt{j} is a continuous function.
            \item For every open subset \mt{U} of \mtb{Z}, \mt{f^{-1}(U)} is open in \mtb{Y}, and \mt{g^{-1}(f^{-1}(U))} is open in \mtb{X}. Thus, \mt{g \circ f} is continuous
            \item For every open subset \mt{U} of \mtb{Y}, \mt{f^{-1}(U)} is open in \mtb{X}, thus \mt{f^{-1}(U)\cap A} is open in \mt{A}. Thus the function \mt{f|A} is continuous.
            \item If \mtb{Z} is a subspace of \mtb{Y}, then every open subset of \mtb{Z} can be represented as \mt{U\cap\mathbb{Z}}, where \mt{U} is a open subset of \mtb{Y}. Thus \mt{h^{-1}(U\cap\mathbb{Z})=g^{-1}(\mathbb{Z})\cap g^{-1}(U) = \mathbb{X}\cap g^{-1}(U)} which is a open subset of \mt{X}, thus \mt{h} is continuous.
            
            If \mtb{Y} is a subspace of \mtb{Z}. Then we take a open subset \mt{U} of \tmb{Z}. \mt{h^{-1}(U) = g^(-1)(U\cap \mathbb{Y}) } which is open in \mtb{X}, thus \mt{h} is continuous.

            \item if \mt{f|U_{\alpha}} is continuous for each \mt{\alpha}. For every open subset \mt{U} of \mtb{Y}.
            \begin{equation*}
                  U = \cup_{\alpha} (U_{\alpha}\cap U)
            \end{equation*}
            where \mt{U_{\alpha}\cap U} is open both in \mt{U_{\alpha}} and in \mtb{Y}. 

            Thus,
            \begin{eqnarray*}
                  f^{-1}(U) &=& f^{-1}(\cup_{\alpha} (U_{\alpha}\cap U)) \\
                  &=& \cup_{\alpha} ((f|U_{\alpha})^{-1}(U_{\alpha}\cap U))
            \end{eqnarray*}
            and each \mt{(f|U_{\alpha})^{-1}(U_{\alpha}\cap U)} is open, thus \mt{f^{-1}(U)} is open.
      \end{enumerate}
\end{proof}

\begin{theorem}[The pasting lemma]\label{theorem:ThePastingLemma}\footnote{
      The proof of this theorem is similar to the "Local formulation of continuity" condition of "Rules for constructing continuous functions", so we omit the proof here.
}
      Let \mt{
            \mathbb{X} = A \cup B
      }, where \mt{A,B} are closed in \mtb{X}. Let \mt{f: A\rightarrow \mathbb{Y}} and \mt{g: B \rightarrow \mathbb{Y}} be continuous. If \mt{f(x)=g(x)} for every \mt{x\in A\cap B}, then \mt{f,g} combine to give a continuous function \mt{h: \mathbb{X}\rightarrow\mathbb{Y}}, defined by setting \mt{h(x)=f(x),x\in A} and \mt{h(x)=g(x),x\in B}.
\end{theorem}

\begin{theorem}[Maps into products]\label{theorem:MapsIntoProducts} \footnote{
      The map \mt{f_{1},f_{2}} are called the \defineNewWord{coordinate functions}\label{def:CoordinateFunctions} of \mt{f}
}
      Let \mt{f: A \rightarrow \mathbb{X}\times\mathbb{Y}} be given by the equation
      \begin{equation*}
            f(a) = (f_{1}(a),f_{2}(a))
      \end{equation*}

      Then, the function \mt{f} is continuous \ioi the functions
      \begin{equation*}
            f_{1}: A \rightarrow \mathbb{X}, f_{2}: A \rightarrow \mathbb{Y}
      \end{equation*}
      are continuous.
\end{theorem}

\begin{proof}
      Let \mt{\pi_{1},\pi_{2}} be the projection function
      \begin{eqnarray*}
            \pi_{1}&:& \mathbb{X}\times\mathbb{Y} \rightarrow \mathbb{X} \\
            \pi_{2}&:& \mathbb{X}\times\mathbb{Y} \rightarrow \mathbb{Y}
      \end{eqnarray*}

      \vspace{1em}

      We first proof that if \mt{U} is an open subset of \productSet{\mathbb{X}}{\mathbb{Y}},
      \begin{equation*}
            f^{-1}(U) = f_{1}^{-1}(\pi_{1}(U)) \cap f_{2}^{-1}(\pi_{2}(U))
      \end{equation*}

      Let \mt{x \times y \in U}, \mt{f^{-1}(x \times y)} contains all \mt{a} such that \mt{f(a) = x \times y}.

      Then for any \mt{a \in f^{-1}(x \times y)}, \mt{a \in f_{1}^{-1}(\pi_{1}(x \times y)), a \in f_{2}^{-1}(\pi_{2}(x \times y))}.

      Thus, \mt{f^{-1}(x \times y) \subseteq f_{1}^{-1}(\pi_{1}(x \times y)) \cap f_{2}^{-1}(\pi_{2}(x \times y))}.

      Thus \mt{f^{-1}(U) \subseteq f_{1}^{-1}(\pi_{1}(U)) \cap f_{2}^{-1}(\pi_{2}(U))}.

      \vspace{1em}

      Also, if \mt{a \in f_{1}^{-1}(\pi_{1}(x \times y)), a \in f_{2}^{-1}(\pi_{2}(x \times y))}, \mt{f_{1}(a) = x, f_{2}(a) = y}.

      Thus \mt{f(a) = x \times y}.
      Thus \mt{a \in f^{-1}(x \times y)}.

      Thus \mt{f^{-1}(U) = f_{1}^{-1}(\pi_{1}(U)) \cap f_{2}^{-1}(\pi_{2}(U))}

      \vspace{1em}

      Let \mt{U} be any open subset of \productSet{\mathbb{X}}{\mathbb{Y}}

      \begin{equation*}
            f^{-1}(U) = f_{1}^{-1}(\pi_{1}(U)) \cap f_{2}^{-1}(\pi_{2}(U))
      \end{equation*}

      Where \mt{f_{1}^{-1}(\pi_{1}(U))} and \mt{f_{2}^{-1}(\pi_{2}(U))} are both open set. Thus \mt{f^{-1}(U)} is open.
\end{proof}
      \subsection{Exercise}

\begin{enumerate}
      \item Give an counter example why \mt{
            \closure{\cup A_{\alpha}} = \cup \closure{A_{\alpha}}
      } dose not hold.

      \begin{proof}
            Consider the X be the K-topology on the real line.
      
            Let
            \begin{eqnarray*}
                  A_{n} &=& (\frac{1}{n+1},\frac{1}{n}), n \in \mathbb{Z}_{+} \\
                  A &=& \cup A_{n}
            \end{eqnarray*}
      
            Then 
            \begin{eqnarray*}
                  \closure{A_{n}} &=& [\frac{1}{n+1},\frac{1}{n}] \\
                  \cup \closure{A_{n}} &=& (0,1]
            \end{eqnarray*}
      
            However, as every neighbourhood of \mt{0} intersect \mt{\cup A_{\alpha}}. \mt{0 \in \closure{\cup A_{\alpha}}}.
      
            Thus, \mt{
                  \closure{\cup A_{\alpha}} \neq \cup \closure{A_{\alpha}}
            }
      \end{proof}

      \item Prove that 
      \begin{equation*}
            \closure{A-B} \supseteq \closure{A} - \closure{B}
      \end{equation*}

      \begin{proof}
            If \mt{x \in \closure{A} - \closure{B}}. Then
            \begin{eqnarray*}
                  x \in \closure{A}, x \notin \closure{B}
            \end{eqnarray*}.

            Thus for open set \mt{U} containing \mt{x}
            \begin{eqnarray*}
                  &\exists& U_{1} \cap B = \emptyset \\
                  &\forall& U \cap A \neq \emptyset
            \end{eqnarray*}

            Suppose that \mt{x \notin \closure{A-B}}. Then
            \begin{eqnarray*}
                  \exists U_{0} \cap (A-B) = \emptyset
            \end{eqnarray*}

            Thus,
            \begin{eqnarray*}
                  U_{0} \cap A \subseteq B
            \end{eqnarray*}

            Thus,
            \begin{eqnarray*}
                  U_{1} \cap B &=& \emptyset \\
                  U_{1} \cap U_{0} \cap A &=& \emptyset
            \end{eqnarray*}

            As \mt{U_{1} \cap U_{0}} is an open set containing \mt{x}, so there is contradiction with \mt{x \in \closure{A}}. Thus \mt{x \in \closure{A-B}}.
      \end{proof}

      \item A \defineNewWord{diagonal}\label{def:Diagonal} is a subset \mt{
            \Delta = \{ x \times x | x \in \mathbb{X} \}
      } of the product topology \mtb{X \times X} where \mtb{X} is a topological space. Show that the diagonal is closed in \mtb{X \times X} \ioi \mtb{X} is a Hausdorff space.

      \begin{proof}
            If \mtb{X} is a Hausdorff space. For every element \productSet{x}{y} of \productSet{\mathbb{X}}{\mathbb{X}} that not in \mt{\Delta}. We take disjoint set \mt{U_{x},U_{y}} where \mt{
                  x \in U_{x}, y \in U_{y}
            }. Then \mt{
                  \mathbb{X} \times \mathbb{X} - \Delta = \cup_{x \neq y} U_{x} \times U_{y}
            }. Where \mt{\cup_{x \neq y} U_{x} \times U_{y}} is an open set. Thus \mt{\Delta} is a closed set.

            Conversely, if \mt{\Delta} is a closed set, suppose that \mtb{X} is not a Hausdorff space. Then there exists distinct \mt{x, y} such that every neighbourhood of \mt{x} and \mt{y} intersect. Let \mtb{B} be a basis of topology of \mtb{X}. Then \mt{
                  x \times y \in \mathbb{X} \times \mathbb{X} - \Delta
            }. However we cannot find \mt{
                  B_{1}, B_{2} \in \mathbb{B}, x \times y \in B_{1} \times B_{2} \subset \mathbb{X} \times \mathbb{X} - \Delta
            }. Then \mt{\Delta} is not a closed set. So there is a contradiction, then \mtb{X} must be a Hausdorff space.
      \end{proof}

      \item Prove that \mt{T_{1}} axiom is equivalent to the condition such that for every distinct pair \mt{x,y} of \mtb{X}, there exists neighbourhood of \mt{x} does not contain \mt{y}.
      
      \begin{proof}
            First if \mt{T_{1}} axiom hold, then for every pair \mt{x,y}, the neighbourhood \mt{\mathbb{X}-\{y\}} of \mt{x} does not contain \mt{y}, so the second condition hold.

            Conversely, if the second condition hold. Suppose that we can find a finite points set say \mt{\{x_{1}, x_{2}, x_{3} \dots}\}, then there must exists \mt{x \in \{x_{1}, x_{2}, x_{3} \dots}\} such that the set \mt{\{x\}} is not closed. Then \mt{
                  \closure{\{x\}} - \{x\} \neq \emptyset
            }. Let \mt{y \in \closure{\{x\}} - \{x\} }, then every neighbourhood of y must contain \mt{x}, this is a contradiction to the second condition, so the \mt{T_{1}} axiom must hold.
      \end{proof}

      \item If \mt{A \subseteq \mathbb{X}}, we define the \defineNewWord{boundary}\label{def:Boundary} of \mt{A} by the equation
      \begin{equation*}
            \text{Bd} A = \closure{A} \cap \closure{\mathbb{X}-A}
      \end{equation*}
      \begin{enumerate}
            \item Show that \mt{\text{Int}A} and \mt{\text{Bd} A} are disjoint and \mt{
                  \closure{A} = \text{Int} A \cup \text{Bd} A
            }.
            
            \begin{proof}
                  For every \mt{ x \in \boundary{A} }, every open set contain \mt{x} must intersect \mt{A} and \mt{\mathbb{X}-A} so, there is no open set \mt{U} contain \mt{x}, \mt{U \subseteq A}.

                  For every \mt{x' \in \interior{A}}, there exists \mt{U' \subseteq A}, so \mt{\boundary{A}} and \mt{\interior{A}} are disjoint sets.

                  For every \mt{x \in \closure{A}}, \mt{x \in \boundary{A}} or \mt{x \notin \boundary{A}}. We discuss the condition that \mt{x \notin \boundary{A}}.

                  Then \mt{x \notin \closure{\mathbb{X}-A}}, then there exists a open set \mt{U} containing \mt{x}, that does not intersect with \mt{\mathbb{X}-A}. Thus \mt{U \subseteq A}, thus \mt{x \in \interior{A}}. So \mt{
                        \closure{A} \subseteq \text{Int} A \cup \text{Bd} A
                  }.

                  Then, \mt{\boundary{A} \subseteq \closure{A}}, \mt{\interior{A} \subseteq A \subseteq \closure{A}}. Thus, \mt{
                        \closure{A} \supseteq \text{Int} A \cup \text{Bd} A
                  }

                  So, \mt{
                        \closure{A} = \text{Int} A \cup \text{Bd} A
                  }
            \end{proof}

            \item Show that \mt{\boundary{A} = \emptyset} \ioi \mt{A} is both open and closed.
            
            \begin{proof}
                  So, \mt{\interior{A} = \closure{A}}, then \mt{\boundary{A} = \emptyset} follows directly from \mt{
                        \closure{A} = \text{Int} A \cup \text{Bd} A
                  }.
            \end{proof}

            \item Show that \mt{U} is open \ioi \mt{\boundary{U} = \closure{U}-U}.
            
            \begin{proof}
                  Suppose U is open. Then \mt{\closure{\mathbb{X}-U} = \mathbb{X} - U}. Then for every \mt{x \in U}, \mt{x \notin \mathbb{X} - U, x \notin \closure{\mathbb{X}-U}}. Thus \mt{\closure{U} \cap \closure{\mathbb{X}-U}=\closure{U}-U}.

                  Conversely, suppose \mt{\boundary{U} = \closure{U}-U}. Then for every \mt{x \in U}, \mt{x \notin \boundary{U}}. Then as \mt{\closure{U} = \interior{U}\cup \boundary{U}}, \mt{x \in \interior{U}}. So \mt{\interior{U} \supseteq U}. Thus \mt{U = \interior{U}}. Thus, \mt{U} is open.
            \end{proof}

      \end{enumerate}
\end{enumerate}



      \section{Continuous Function}

\begin{definition}[continuous]\label{def:Continuous}\footnote{
      As the continuity of a function is different as the topological spaces are different. So if we want to emphasis this fact, we say that \mt{f} is continuous \defineNewWord{relative}\label{def:ContinuousRelativeTo} to specific topologies on \mtb{X} and \mtb{Y}.
}
      Let \mtb{X} and \mtb{Y} be topological spaces. A function \mt{f: \mathbb{X}\rightarrow \mathbb{Y}} is said to be \defineNewWord{continuous} if for each open subset \mt{V} of \tmb{Y}, the set \mt{f^{-1}(V)} is an open subset of \mtb{X}.
\end{definition}

\begin{theorem}
      Let \mtb{X} and \mtb{Y} be topological spaces; let \mt{f: \mathbb{X}\rightarrow\mathbb{Y}}. Then the following are equivalent.
      \begin{enumerate}
            \item \mt{f} is continuous.
            \item For every subset \mt{A} of \mt{X}, one has \mt{f(\closure{A})\subseteq\closure{f(A)}}.
            \item For every closed set \mt{B} of \mtb{Y}, the set \mt{f^{-1}(B)} is closed in \mtb{X}.
            \item For each \mt{x\in\mathbb{X}} and each neighbourhood of \mt{V} of \mt{f(x)}, there is a neighbourhood \mt{U} of \mt{x} such that \mt{f(U) \subseteq V}.
      \end{enumerate}
\end{theorem}

\begin{proof}

      \hspace{1em}

      1 \mt{\Rightarrow} 3:

      Let \mt{A} be a open set in \mtb{Y}. \mt{f^{-1}(\mathbb{Y}-A) = \mathbb{X} - f^{-1}(A)}.
      
      \vspace{1em}

      3 \mt{\Rightarrow} 1:

      Let \mt{A} be a closed set in \mtb{Y}. \mt{f^{-1}(\mathbb{Y}-A) = \mathbb{X} - f^{-1}(A)}.

      \vspace{1em}

      1 \mt{\Rightarrow} 2:

      For \mt{x \in \closure{A}}, we take a open set \mt{f(x) \in U \subseteq \mathbb{Y}}. Thus \mt{x \in f^{-1}(U) \cap A \neq \emptyset}. Thus \mt{U \cap f(A) \neq \emptyset}. So \mt{f(x) \in \closure{f(A)}}. Thus \mt{f(\closure{A})\subseteq\closure{f(A)}}.

      \vspace{1em}

      2 \mt{\Rightarrow} 3:

      Suppose \mt{f} is not continuous. Then there must exists \mt{V}, such that \mt{f^{-1}(V) = U} is not closed. Thus \mt{\closure{U} \supset B = f^{-1}(A)}. Thus \mt{f{\closure{B}} \supset A}. However \mt{f(\closure{B}) \subseteq \closure{f(B)} = A}. There is a contradiction. So \mt{f} must be continuous.

      \vspace{1em}

      1 \mt{\Rightarrow} 4:

      For every neighbourhood \mt{V} of \mt{f(x)}, \mt{f^{-1}(V)} is a neighbourhood of \mt{x} that \mt{f(f^{-1}(V)) \subseteq V}.

      \vspace{1em}

      4 \mt{\Rightarrow} 1:

      We take a open set \mt{V} of \mtb{Y}. Let \mt{S} be the collection of all open set \mt{U} in \mtb{X} such that \mt{f(U) \subseteq V}. The set cannot be empty unless \mt{f^{-1}(V) = \emptyset}. Let \mt{U_{0}} denote the union of all the element in \mt{S}. We prove that \mt{U_{0} = f^{-1}(V)}.

      For all element \mt{x \in U_{0}}, \mt{f(x) \in V}. Thus \mt{U_{0} \subseteq f^{-1}(V)}.

      For all element \mt{x \in f^{-1}(V)}. There is a \mt{U'} such that \mt{x \in U', f(U')\subseteq V}. This follows from the condition 4. Thus \mt{U' \in S}. Thus \mt{x \in U_{0}}. Thus \mt{U_{0} \subseteq f^{-1}(V)}. As \mt{U_{0}} is union of open set, \mt{U_{0}} is also open. Thus, \mt{f^{-1}(V)} is also open.
      
      Thus \mt{f} is continuous.
\end{proof}

\begin{definition}[homeomorphism]\label{def:Homeomorphism}\footnote{
      A equivalent way to define homeomorphism, is that for any open subset \mt{U} of \mtb{X}, \mt{f(U)} is open \ioi \mt{U} is open.
}
      Let \mtb{X} and \mtb{Y} be topological space; let \mt{f: \mathbb{X} \rightarrow \mathbb{Y}} be a bijection. If both the function \mt{f} and the inverse function
      \begin{equation*}
            f^{-1}: \mathbb{Y} \rightarrow \mathbb{X}
      \end{equation*}
      are continuous, then f is called a \defineNewWord{homeomorphism}
\end{definition}

\begin{definition}[topological imbedding]\label{def:TopologicalImbedding}
      Suppose that \mt{f: \mathbb{X} \rightarrow \mathbb{Y}} is an injective continuous map, where \mtb{X} and \mtb{Y} are topological spaces. Let \mtb{Z} be the image set \mt{f(\mathbb{X})}, considered as a subspace of \mtb{Y}; then the function \mt{f': \mathbb{X} \rightarrow \mathbb{Z}} obtained by restricting the range of \mt{f} is bijective. If \mt{f'} happens to be a homeomorphism of \mtb{X} with \mtb{Z}, we say that the map \mt{f: \mathbb{X} \rightarrow \mathbb{Y}} is a \defineNewWord{topological imbedding}, or simply an \defineNewWord{imbedding}, of \mtb{X} in \mtb{Y}.
\end{definition}

\begin{theorem}[Rules for constructing continuous functions]\label{theorem:RulesForConstructingContinuousFunctions}
      Let \mtb{X}, \mtb{Y}, and \mtb{Z} be topological spaces.
      \begin{enumerate}
            \item (Constant function) If \mt{f: \mathbb{X} \rightarrow \mathbb{Y}} maps all of \mtb{X} into the single point \mt{y_{0}} of \mtb{Y}, then \mt{f} is continuous.
            \item (Inclusion) If \mt{A} is a subspace of \mtb{X}, the inclusion function \mt{j: A \rightarrow \mathbb{X}} is continuous.
            \item (Composites) If \mt{f: \mathbb{X}\rightarrow \mathbb{Y}} and \mt{g:\mathbb{Y}\rightarrow\mathbb{Z}} are continuous, then the map \mt{g \circ f: \mathbb{X} \rightarrow \mathbb{Z}} is continuous.
            \item (Restricting the domain) If \mt{f: \mathbb{X} \rightarrow \mathbb{Y}} is continuous, and if \mt{A} is a subspace of \mtb{X}, then the restriction function \mt{f|A : A \rightarrow \mathbb{Y}} is continuous.
            \item (Restricting or expanding the range) Let \mt{f:\mathbb{X}\rightarrow\mathbb{Y}} is continuous. Let \mtb{Z} be a subspace of \mtb{Y} containing the image \mt{f(\mathbb{X})}, the function \mt{h: \mathbb{X}\rightarrow \mathbb{Z}} obtained by restricting the range of \mt{f} is continuous. If \mtb{Z} is a space having \mtb{Y} as a subspace, then the function \mt{h: \mathbb{X}\rightarrow\mathbb{Y}} obtained by expanding the range of \mt{f} is continuous.
            \item (Local formulation of continuity) The map \mt{f: \mathbb{X}\rightarrow\mathbb{Y}} is continuous if \mtb{X} can be written as the union of open sets \mt{U_{\alpha}} such set \mt{f|U_{\alpha}} is continuous for each \mt{\alpha}
      \end{enumerate}
\end{theorem}

\begin{proof}

      \hspace{1em}

      \begin{enumerate}
            \item \mt{f^{-1}(U)} of any open set \mt{U} is \mtb{X}, thus \mt{f} is continuous.
            \item For every open subset \mt{U} of \mtb{X}, \mt{j^{-1}(U) = U\cap A} is continuous in \mt{A}. Thus \mt{j} is a continuous function.
            \item For every open subset \mt{U} of \mtb{Z}, \mt{f^{-1}(U)} is open in \mtb{Y}, and \mt{g^{-1}(f^{-1}(U))} is open in \mtb{X}. Thus, \mt{g \circ f} is continuous
            \item For every open subset \mt{U} of \mtb{Y}, \mt{f^{-1}(U)} is open in \mtb{X}, thus \mt{f^{-1}(U)\cap A} is open in \mt{A}. Thus the function \mt{f|A} is continuous.
            \item If \mtb{Z} is a subspace of \mtb{Y}, then every open subset of \mtb{Z} can be represented as \mt{U\cap\mathbb{Z}}, where \mt{U} is a open subset of \mtb{Y}. Thus \mt{h^{-1}(U\cap\mathbb{Z})=g^{-1}(\mathbb{Z})\cap g^{-1}(U) = \mathbb{X}\cap g^{-1}(U)} which is a open subset of \mt{X}, thus \mt{h} is continuous.
            
            If \mtb{Y} is a subspace of \mtb{Z}. Then we take a open subset \mt{U} of \tmb{Z}. \mt{h^{-1}(U) = g^(-1)(U\cap \mathbb{Y}) } which is open in \mtb{X}, thus \mt{h} is continuous.

            \item if \mt{f|U_{\alpha}} is continuous for each \mt{\alpha}. For every open subset \mt{U} of \mtb{Y}.
            \begin{equation*}
                  U = \cup_{\alpha} (U_{\alpha}\cap U)
            \end{equation*}
            where \mt{U_{\alpha}\cap U} is open both in \mt{U_{\alpha}} and in \mtb{Y}. 

            Thus,
            \begin{eqnarray*}
                  f^{-1}(U) &=& f^{-1}(\cup_{\alpha} (U_{\alpha}\cap U)) \\
                  &=& \cup_{\alpha} ((f|U_{\alpha})^{-1}(U_{\alpha}\cap U))
            \end{eqnarray*}
            and each \mt{(f|U_{\alpha})^{-1}(U_{\alpha}\cap U)} is open, thus \mt{f^{-1}(U)} is open.
      \end{enumerate}
\end{proof}

\begin{theorem}[The pasting lemma]\label{theorem:ThePastingLemma}\footnote{
      The proof of this theorem is similar to the "Local formulation of continuity" condition of "Rules for constructing continuous functions", so we omit the proof here.
}
      Let \mt{
            \mathbb{X} = A \cup B
      }, where \mt{A,B} are closed in \mtb{X}. Let \mt{f: A\rightarrow \mathbb{Y}} and \mt{g: B \rightarrow \mathbb{Y}} be continuous. If \mt{f(x)=g(x)} for every \mt{x\in A\cap B}, then \mt{f,g} combine to give a continuous function \mt{h: \mathbb{X}\rightarrow\mathbb{Y}}, defined by setting \mt{h(x)=f(x),x\in A} and \mt{h(x)=g(x),x\in B}.
\end{theorem}

\begin{theorem}[Maps into products]\label{theorem:MapsIntoProducts} \footnote{
      The map \mt{f_{1},f_{2}} are called the \defineNewWord{coordinate functions}\label{def:CoordinateFunctions} of \mt{f}
}
      Let \mt{f: A \rightarrow \mathbb{X}\times\mathbb{Y}} be given by the equation
      \begin{equation*}
            f(a) = (f_{1}(a),f_{2}(a))
      \end{equation*}

      Then, the function \mt{f} is continuous \ioi the functions
      \begin{equation*}
            f_{1}: A \rightarrow \mathbb{X}, f_{2}: A \rightarrow \mathbb{Y}
      \end{equation*}
      are continuous.
\end{theorem}

\begin{proof}
      Let \mt{\pi_{1},\pi_{2}} be the projection function
      \begin{eqnarray*}
            \pi_{1}&:& \mathbb{X}\times\mathbb{Y} \rightarrow \mathbb{X} \\
            \pi_{2}&:& \mathbb{X}\times\mathbb{Y} \rightarrow \mathbb{Y}
      \end{eqnarray*}

      \vspace{1em}

      We first proof that if \mt{U} is an open subset of \productSet{\mathbb{X}}{\mathbb{Y}},
      \begin{equation*}
            f^{-1}(U) = f_{1}^{-1}(\pi_{1}(U)) \cap f_{2}^{-1}(\pi_{2}(U))
      \end{equation*}

      Let \mt{x \times y \in U}, \mt{f^{-1}(x \times y)} contains all \mt{a} such that \mt{f(a) = x \times y}.

      Then for any \mt{a \in f^{-1}(x \times y)}, \mt{a \in f_{1}^{-1}(\pi_{1}(x \times y)), a \in f_{2}^{-1}(\pi_{2}(x \times y))}.

      Thus, \mt{f^{-1}(x \times y) \subseteq f_{1}^{-1}(\pi_{1}(x \times y)) \cap f_{2}^{-1}(\pi_{2}(x \times y))}.

      Thus \mt{f^{-1}(U) \subseteq f_{1}^{-1}(\pi_{1}(U)) \cap f_{2}^{-1}(\pi_{2}(U))}.

      \vspace{1em}

      Also, if \mt{a \in f_{1}^{-1}(\pi_{1}(x \times y)), a \in f_{2}^{-1}(\pi_{2}(x \times y))}, \mt{f_{1}(a) = x, f_{2}(a) = y}.

      Thus \mt{f(a) = x \times y}.
      Thus \mt{a \in f^{-1}(x \times y)}.

      Thus \mt{f^{-1}(U) = f_{1}^{-1}(\pi_{1}(U)) \cap f_{2}^{-1}(\pi_{2}(U))}

      \vspace{1em}

      Let \mt{U} be any open subset of \productSet{\mathbb{X}}{\mathbb{Y}}

      \begin{equation*}
            f^{-1}(U) = f_{1}^{-1}(\pi_{1}(U)) \cap f_{2}^{-1}(\pi_{2}(U))
      \end{equation*}

      Where \mt{f_{1}^{-1}(\pi_{1}(U))} and \mt{f_{2}^{-1}(\pi_{2}(U))} are both open set. Thus \mt{f^{-1}(U)} is open.
\end{proof}
      \subsection{Exercise}

\begin{enumerate}
      \item Give an counter example why \mt{
            \closure{\cup A_{\alpha}} = \cup \closure{A_{\alpha}}
      } dose not hold.

      \begin{proof}
            Consider the X be the K-topology on the real line.
      
            Let
            \begin{eqnarray*}
                  A_{n} &=& (\frac{1}{n+1},\frac{1}{n}), n \in \mathbb{Z}_{+} \\
                  A &=& \cup A_{n}
            \end{eqnarray*}
      
            Then 
            \begin{eqnarray*}
                  \closure{A_{n}} &=& [\frac{1}{n+1},\frac{1}{n}] \\
                  \cup \closure{A_{n}} &=& (0,1]
            \end{eqnarray*}
      
            However, as every neighbourhood of \mt{0} intersect \mt{\cup A_{\alpha}}. \mt{0 \in \closure{\cup A_{\alpha}}}.
      
            Thus, \mt{
                  \closure{\cup A_{\alpha}} \neq \cup \closure{A_{\alpha}}
            }
      \end{proof}

      \item Prove that 
      \begin{equation*}
            \closure{A-B} \supseteq \closure{A} - \closure{B}
      \end{equation*}

      \begin{proof}
            If \mt{x \in \closure{A} - \closure{B}}. Then
            \begin{eqnarray*}
                  x \in \closure{A}, x \notin \closure{B}
            \end{eqnarray*}.

            Thus for open set \mt{U} containing \mt{x}
            \begin{eqnarray*}
                  &\exists& U_{1} \cap B = \emptyset \\
                  &\forall& U \cap A \neq \emptyset
            \end{eqnarray*}

            Suppose that \mt{x \notin \closure{A-B}}. Then
            \begin{eqnarray*}
                  \exists U_{0} \cap (A-B) = \emptyset
            \end{eqnarray*}

            Thus,
            \begin{eqnarray*}
                  U_{0} \cap A \subseteq B
            \end{eqnarray*}

            Thus,
            \begin{eqnarray*}
                  U_{1} \cap B &=& \emptyset \\
                  U_{1} \cap U_{0} \cap A &=& \emptyset
            \end{eqnarray*}

            As \mt{U_{1} \cap U_{0}} is an open set containing \mt{x}, so there is contradiction with \mt{x \in \closure{A}}. Thus \mt{x \in \closure{A-B}}.
      \end{proof}

      \item A \defineNewWord{diagonal}\label{def:Diagonal} is a subset \mt{
            \Delta = \{ x \times x | x \in \mathbb{X} \}
      } of the product topology \mtb{X \times X} where \mtb{X} is a topological space. Show that the diagonal is closed in \mtb{X \times X} \ioi \mtb{X} is a Hausdorff space.

      \begin{proof}
            If \mtb{X} is a Hausdorff space. For every element \productSet{x}{y} of \productSet{\mathbb{X}}{\mathbb{X}} that not in \mt{\Delta}. We take disjoint set \mt{U_{x},U_{y}} where \mt{
                  x \in U_{x}, y \in U_{y}
            }. Then \mt{
                  \mathbb{X} \times \mathbb{X} - \Delta = \cup_{x \neq y} U_{x} \times U_{y}
            }. Where \mt{\cup_{x \neq y} U_{x} \times U_{y}} is an open set. Thus \mt{\Delta} is a closed set.

            Conversely, if \mt{\Delta} is a closed set, suppose that \mtb{X} is not a Hausdorff space. Then there exists distinct \mt{x, y} such that every neighbourhood of \mt{x} and \mt{y} intersect. Let \mtb{B} be a basis of topology of \mtb{X}. Then \mt{
                  x \times y \in \mathbb{X} \times \mathbb{X} - \Delta
            }. However we cannot find \mt{
                  B_{1}, B_{2} \in \mathbb{B}, x \times y \in B_{1} \times B_{2} \subset \mathbb{X} \times \mathbb{X} - \Delta
            }. Then \mt{\Delta} is not a closed set. So there is a contradiction, then \mtb{X} must be a Hausdorff space.
      \end{proof}

      \item Prove that \mt{T_{1}} axiom is equivalent to the condition such that for every distinct pair \mt{x,y} of \mtb{X}, there exists neighbourhood of \mt{x} does not contain \mt{y}.
      
      \begin{proof}
            First if \mt{T_{1}} axiom hold, then for every pair \mt{x,y}, the neighbourhood \mt{\mathbb{X}-\{y\}} of \mt{x} does not contain \mt{y}, so the second condition hold.

            Conversely, if the second condition hold. Suppose that we can find a finite points set say \mt{\{x_{1}, x_{2}, x_{3} \dots}\}, then there must exists \mt{x \in \{x_{1}, x_{2}, x_{3} \dots}\} such that the set \mt{\{x\}} is not closed. Then \mt{
                  \closure{\{x\}} - \{x\} \neq \emptyset
            }. Let \mt{y \in \closure{\{x\}} - \{x\} }, then every neighbourhood of y must contain \mt{x}, this is a contradiction to the second condition, so the \mt{T_{1}} axiom must hold.
      \end{proof}

      \item If \mt{A \subseteq \mathbb{X}}, we define the \defineNewWord{boundary}\label{def:Boundary} of \mt{A} by the equation
      \begin{equation*}
            \text{Bd} A = \closure{A} \cap \closure{\mathbb{X}-A}
      \end{equation*}
      \begin{enumerate}
            \item Show that \mt{\text{Int}A} and \mt{\text{Bd} A} are disjoint and \mt{
                  \closure{A} = \text{Int} A \cup \text{Bd} A
            }.
            
            \begin{proof}
                  For every \mt{ x \in \boundary{A} }, every open set contain \mt{x} must intersect \mt{A} and \mt{\mathbb{X}-A} so, there is no open set \mt{U} contain \mt{x}, \mt{U \subseteq A}.

                  For every \mt{x' \in \interior{A}}, there exists \mt{U' \subseteq A}, so \mt{\boundary{A}} and \mt{\interior{A}} are disjoint sets.

                  For every \mt{x \in \closure{A}}, \mt{x \in \boundary{A}} or \mt{x \notin \boundary{A}}. We discuss the condition that \mt{x \notin \boundary{A}}.

                  Then \mt{x \notin \closure{\mathbb{X}-A}}, then there exists a open set \mt{U} containing \mt{x}, that does not intersect with \mt{\mathbb{X}-A}. Thus \mt{U \subseteq A}, thus \mt{x \in \interior{A}}. So \mt{
                        \closure{A} \subseteq \text{Int} A \cup \text{Bd} A
                  }.

                  Then, \mt{\boundary{A} \subseteq \closure{A}}, \mt{\interior{A} \subseteq A \subseteq \closure{A}}. Thus, \mt{
                        \closure{A} \supseteq \text{Int} A \cup \text{Bd} A
                  }

                  So, \mt{
                        \closure{A} = \text{Int} A \cup \text{Bd} A
                  }
            \end{proof}

            \item Show that \mt{\boundary{A} = \emptyset} \ioi \mt{A} is both open and closed.
            
            \begin{proof}
                  So, \mt{\interior{A} = \closure{A}}, then \mt{\boundary{A} = \emptyset} follows directly from \mt{
                        \closure{A} = \text{Int} A \cup \text{Bd} A
                  }.
            \end{proof}

            \item Show that \mt{U} is open \ioi \mt{\boundary{U} = \closure{U}-U}.
            
            \begin{proof}
                  Suppose U is open. Then \mt{\closure{\mathbb{X}-U} = \mathbb{X} - U}. Then for every \mt{x \in U}, \mt{x \notin \mathbb{X} - U, x \notin \closure{\mathbb{X}-U}}. Thus \mt{\closure{U} \cap \closure{\mathbb{X}-U}=\closure{U}-U}.

                  Conversely, suppose \mt{\boundary{U} = \closure{U}-U}. Then for every \mt{x \in U}, \mt{x \notin \boundary{U}}. Then as \mt{\closure{U} = \interior{U}\cup \boundary{U}}, \mt{x \in \interior{U}}. So \mt{\interior{U} \supseteq U}. Thus \mt{U = \interior{U}}. Thus, \mt{U} is open.
            \end{proof}

      \end{enumerate}
\end{enumerate}



      \section{Continuous Function}

\begin{definition}[continuous]\label{def:Continuous}\footnote{
      As the continuity of a function is different as the topological spaces are different. So if we want to emphasis this fact, we say that \mt{f} is continuous \defineNewWord{relative}\label{def:ContinuousRelativeTo} to specific topologies on \mtb{X} and \mtb{Y}.
}
      Let \mtb{X} and \mtb{Y} be topological spaces. A function \mt{f: \mathbb{X}\rightarrow \mathbb{Y}} is said to be \defineNewWord{continuous} if for each open subset \mt{V} of \tmb{Y}, the set \mt{f^{-1}(V)} is an open subset of \mtb{X}.
\end{definition}

\begin{theorem}
      Let \mtb{X} and \mtb{Y} be topological spaces; let \mt{f: \mathbb{X}\rightarrow\mathbb{Y}}. Then the following are equivalent.
      \begin{enumerate}
            \item \mt{f} is continuous.
            \item For every subset \mt{A} of \mt{X}, one has \mt{f(\closure{A})\subseteq\closure{f(A)}}.
            \item For every closed set \mt{B} of \mtb{Y}, the set \mt{f^{-1}(B)} is closed in \mtb{X}.
            \item For each \mt{x\in\mathbb{X}} and each neighbourhood of \mt{V} of \mt{f(x)}, there is a neighbourhood \mt{U} of \mt{x} such that \mt{f(U) \subseteq V}.
      \end{enumerate}
\end{theorem}

\begin{proof}

      \hspace{1em}

      1 \mt{\Rightarrow} 3:

      Let \mt{A} be a open set in \mtb{Y}. \mt{f^{-1}(\mathbb{Y}-A) = \mathbb{X} - f^{-1}(A)}.
      
      \vspace{1em}

      3 \mt{\Rightarrow} 1:

      Let \mt{A} be a closed set in \mtb{Y}. \mt{f^{-1}(\mathbb{Y}-A) = \mathbb{X} - f^{-1}(A)}.

      \vspace{1em}

      1 \mt{\Rightarrow} 2:

      For \mt{x \in \closure{A}}, we take a open set \mt{f(x) \in U \subseteq \mathbb{Y}}. Thus \mt{x \in f^{-1}(U) \cap A \neq \emptyset}. Thus \mt{U \cap f(A) \neq \emptyset}. So \mt{f(x) \in \closure{f(A)}}. Thus \mt{f(\closure{A})\subseteq\closure{f(A)}}.

      \vspace{1em}

      2 \mt{\Rightarrow} 3:

      Suppose \mt{f} is not continuous. Then there must exists \mt{V}, such that \mt{f^{-1}(V) = U} is not closed. Thus \mt{\closure{U} \supset B = f^{-1}(A)}. Thus \mt{f{\closure{B}} \supset A}. However \mt{f(\closure{B}) \subseteq \closure{f(B)} = A}. There is a contradiction. So \mt{f} must be continuous.

      \vspace{1em}

      1 \mt{\Rightarrow} 4:

      For every neighbourhood \mt{V} of \mt{f(x)}, \mt{f^{-1}(V)} is a neighbourhood of \mt{x} that \mt{f(f^{-1}(V)) \subseteq V}.

      \vspace{1em}

      4 \mt{\Rightarrow} 1:

      We take a open set \mt{V} of \mtb{Y}. Let \mt{S} be the collection of all open set \mt{U} in \mtb{X} such that \mt{f(U) \subseteq V}. The set cannot be empty unless \mt{f^{-1}(V) = \emptyset}. Let \mt{U_{0}} denote the union of all the element in \mt{S}. We prove that \mt{U_{0} = f^{-1}(V)}.

      For all element \mt{x \in U_{0}}, \mt{f(x) \in V}. Thus \mt{U_{0} \subseteq f^{-1}(V)}.

      For all element \mt{x \in f^{-1}(V)}. There is a \mt{U'} such that \mt{x \in U', f(U')\subseteq V}. This follows from the condition 4. Thus \mt{U' \in S}. Thus \mt{x \in U_{0}}. Thus \mt{U_{0} \subseteq f^{-1}(V)}. As \mt{U_{0}} is union of open set, \mt{U_{0}} is also open. Thus, \mt{f^{-1}(V)} is also open.
      
      Thus \mt{f} is continuous.
\end{proof}

\begin{definition}[homeomorphism]\label{def:Homeomorphism}\footnote{
      A equivalent way to define homeomorphism, is that for any open subset \mt{U} of \mtb{X}, \mt{f(U)} is open \ioi \mt{U} is open.
}
      Let \mtb{X} and \mtb{Y} be topological space; let \mt{f: \mathbb{X} \rightarrow \mathbb{Y}} be a bijection. If both the function \mt{f} and the inverse function
      \begin{equation*}
            f^{-1}: \mathbb{Y} \rightarrow \mathbb{X}
      \end{equation*}
      are continuous, then f is called a \defineNewWord{homeomorphism}
\end{definition}

\begin{definition}[topological imbedding]\label{def:TopologicalImbedding}
      Suppose that \mt{f: \mathbb{X} \rightarrow \mathbb{Y}} is an injective continuous map, where \mtb{X} and \mtb{Y} are topological spaces. Let \mtb{Z} be the image set \mt{f(\mathbb{X})}, considered as a subspace of \mtb{Y}; then the function \mt{f': \mathbb{X} \rightarrow \mathbb{Z}} obtained by restricting the range of \mt{f} is bijective. If \mt{f'} happens to be a homeomorphism of \mtb{X} with \mtb{Z}, we say that the map \mt{f: \mathbb{X} \rightarrow \mathbb{Y}} is a \defineNewWord{topological imbedding}, or simply an \defineNewWord{imbedding}, of \mtb{X} in \mtb{Y}.
\end{definition}

\begin{theorem}[Rules for constructing continuous functions]\label{theorem:RulesForConstructingContinuousFunctions}
      Let \mtb{X}, \mtb{Y}, and \mtb{Z} be topological spaces.
      \begin{enumerate}
            \item (Constant function) If \mt{f: \mathbb{X} \rightarrow \mathbb{Y}} maps all of \mtb{X} into the single point \mt{y_{0}} of \mtb{Y}, then \mt{f} is continuous.
            \item (Inclusion) If \mt{A} is a subspace of \mtb{X}, the inclusion function \mt{j: A \rightarrow \mathbb{X}} is continuous.
            \item (Composites) If \mt{f: \mathbb{X}\rightarrow \mathbb{Y}} and \mt{g:\mathbb{Y}\rightarrow\mathbb{Z}} are continuous, then the map \mt{g \circ f: \mathbb{X} \rightarrow \mathbb{Z}} is continuous.
            \item (Restricting the domain) If \mt{f: \mathbb{X} \rightarrow \mathbb{Y}} is continuous, and if \mt{A} is a subspace of \mtb{X}, then the restriction function \mt{f|A : A \rightarrow \mathbb{Y}} is continuous.
            \item (Restricting or expanding the range) Let \mt{f:\mathbb{X}\rightarrow\mathbb{Y}} is continuous. Let \mtb{Z} be a subspace of \mtb{Y} containing the image \mt{f(\mathbb{X})}, the function \mt{h: \mathbb{X}\rightarrow \mathbb{Z}} obtained by restricting the range of \mt{f} is continuous. If \mtb{Z} is a space having \mtb{Y} as a subspace, then the function \mt{h: \mathbb{X}\rightarrow\mathbb{Y}} obtained by expanding the range of \mt{f} is continuous.
            \item (Local formulation of continuity) The map \mt{f: \mathbb{X}\rightarrow\mathbb{Y}} is continuous if \mtb{X} can be written as the union of open sets \mt{U_{\alpha}} such set \mt{f|U_{\alpha}} is continuous for each \mt{\alpha}
      \end{enumerate}
\end{theorem}

\begin{proof}

      \hspace{1em}

      \begin{enumerate}
            \item \mt{f^{-1}(U)} of any open set \mt{U} is \mtb{X}, thus \mt{f} is continuous.
            \item For every open subset \mt{U} of \mtb{X}, \mt{j^{-1}(U) = U\cap A} is continuous in \mt{A}. Thus \mt{j} is a continuous function.
            \item For every open subset \mt{U} of \mtb{Z}, \mt{f^{-1}(U)} is open in \mtb{Y}, and \mt{g^{-1}(f^{-1}(U))} is open in \mtb{X}. Thus, \mt{g \circ f} is continuous
            \item For every open subset \mt{U} of \mtb{Y}, \mt{f^{-1}(U)} is open in \mtb{X}, thus \mt{f^{-1}(U)\cap A} is open in \mt{A}. Thus the function \mt{f|A} is continuous.
            \item If \mtb{Z} is a subspace of \mtb{Y}, then every open subset of \mtb{Z} can be represented as \mt{U\cap\mathbb{Z}}, where \mt{U} is a open subset of \mtb{Y}. Thus \mt{h^{-1}(U\cap\mathbb{Z})=g^{-1}(\mathbb{Z})\cap g^{-1}(U) = \mathbb{X}\cap g^{-1}(U)} which is a open subset of \mt{X}, thus \mt{h} is continuous.
            
            If \mtb{Y} is a subspace of \mtb{Z}. Then we take a open subset \mt{U} of \tmb{Z}. \mt{h^{-1}(U) = g^(-1)(U\cap \mathbb{Y}) } which is open in \mtb{X}, thus \mt{h} is continuous.

            \item if \mt{f|U_{\alpha}} is continuous for each \mt{\alpha}. For every open subset \mt{U} of \mtb{Y}.
            \begin{equation*}
                  U = \cup_{\alpha} (U_{\alpha}\cap U)
            \end{equation*}
            where \mt{U_{\alpha}\cap U} is open both in \mt{U_{\alpha}} and in \mtb{Y}. 

            Thus,
            \begin{eqnarray*}
                  f^{-1}(U) &=& f^{-1}(\cup_{\alpha} (U_{\alpha}\cap U)) \\
                  &=& \cup_{\alpha} ((f|U_{\alpha})^{-1}(U_{\alpha}\cap U))
            \end{eqnarray*}
            and each \mt{(f|U_{\alpha})^{-1}(U_{\alpha}\cap U)} is open, thus \mt{f^{-1}(U)} is open.
      \end{enumerate}
\end{proof}

\begin{theorem}[The pasting lemma]\label{theorem:ThePastingLemma}\footnote{
      The proof of this theorem is similar to the "Local formulation of continuity" condition of "Rules for constructing continuous functions", so we omit the proof here.
}
      Let \mt{
            \mathbb{X} = A \cup B
      }, where \mt{A,B} are closed in \mtb{X}. Let \mt{f: A\rightarrow \mathbb{Y}} and \mt{g: B \rightarrow \mathbb{Y}} be continuous. If \mt{f(x)=g(x)} for every \mt{x\in A\cap B}, then \mt{f,g} combine to give a continuous function \mt{h: \mathbb{X}\rightarrow\mathbb{Y}}, defined by setting \mt{h(x)=f(x),x\in A} and \mt{h(x)=g(x),x\in B}.
\end{theorem}

\begin{theorem}[Maps into products]\label{theorem:MapsIntoProducts} \footnote{
      The map \mt{f_{1},f_{2}} are called the \defineNewWord{coordinate functions}\label{def:CoordinateFunctions} of \mt{f}
}
      Let \mt{f: A \rightarrow \mathbb{X}\times\mathbb{Y}} be given by the equation
      \begin{equation*}
            f(a) = (f_{1}(a),f_{2}(a))
      \end{equation*}

      Then, the function \mt{f} is continuous \ioi the functions
      \begin{equation*}
            f_{1}: A \rightarrow \mathbb{X}, f_{2}: A \rightarrow \mathbb{Y}
      \end{equation*}
      are continuous.
\end{theorem}

\begin{proof}
      Let \mt{\pi_{1},\pi_{2}} be the projection function
      \begin{eqnarray*}
            \pi_{1}&:& \mathbb{X}\times\mathbb{Y} \rightarrow \mathbb{X} \\
            \pi_{2}&:& \mathbb{X}\times\mathbb{Y} \rightarrow \mathbb{Y}
      \end{eqnarray*}

      \vspace{1em}

      We first proof that if \mt{U} is an open subset of \productSet{\mathbb{X}}{\mathbb{Y}},
      \begin{equation*}
            f^{-1}(U) = f_{1}^{-1}(\pi_{1}(U)) \cap f_{2}^{-1}(\pi_{2}(U))
      \end{equation*}

      Let \mt{x \times y \in U}, \mt{f^{-1}(x \times y)} contains all \mt{a} such that \mt{f(a) = x \times y}.

      Then for any \mt{a \in f^{-1}(x \times y)}, \mt{a \in f_{1}^{-1}(\pi_{1}(x \times y)), a \in f_{2}^{-1}(\pi_{2}(x \times y))}.

      Thus, \mt{f^{-1}(x \times y) \subseteq f_{1}^{-1}(\pi_{1}(x \times y)) \cap f_{2}^{-1}(\pi_{2}(x \times y))}.

      Thus \mt{f^{-1}(U) \subseteq f_{1}^{-1}(\pi_{1}(U)) \cap f_{2}^{-1}(\pi_{2}(U))}.

      \vspace{1em}

      Also, if \mt{a \in f_{1}^{-1}(\pi_{1}(x \times y)), a \in f_{2}^{-1}(\pi_{2}(x \times y))}, \mt{f_{1}(a) = x, f_{2}(a) = y}.

      Thus \mt{f(a) = x \times y}.
      Thus \mt{a \in f^{-1}(x \times y)}.

      Thus \mt{f^{-1}(U) = f_{1}^{-1}(\pi_{1}(U)) \cap f_{2}^{-1}(\pi_{2}(U))}

      \vspace{1em}

      Let \mt{U} be any open subset of \productSet{\mathbb{X}}{\mathbb{Y}}

      \begin{equation*}
            f^{-1}(U) = f_{1}^{-1}(\pi_{1}(U)) \cap f_{2}^{-1}(\pi_{2}(U))
      \end{equation*}

      Where \mt{f_{1}^{-1}(\pi_{1}(U))} and \mt{f_{2}^{-1}(\pi_{2}(U))} are both open set. Thus \mt{f^{-1}(U)} is open.
\end{proof}
      \subsection{Exercise}

\begin{enumerate}
      \item Give an counter example why \mt{
            \closure{\cup A_{\alpha}} = \cup \closure{A_{\alpha}}
      } dose not hold.

      \begin{proof}
            Consider the X be the K-topology on the real line.
      
            Let
            \begin{eqnarray*}
                  A_{n} &=& (\frac{1}{n+1},\frac{1}{n}), n \in \mathbb{Z}_{+} \\
                  A &=& \cup A_{n}
            \end{eqnarray*}
      
            Then 
            \begin{eqnarray*}
                  \closure{A_{n}} &=& [\frac{1}{n+1},\frac{1}{n}] \\
                  \cup \closure{A_{n}} &=& (0,1]
            \end{eqnarray*}
      
            However, as every neighbourhood of \mt{0} intersect \mt{\cup A_{\alpha}}. \mt{0 \in \closure{\cup A_{\alpha}}}.
      
            Thus, \mt{
                  \closure{\cup A_{\alpha}} \neq \cup \closure{A_{\alpha}}
            }
      \end{proof}

      \item Prove that 
      \begin{equation*}
            \closure{A-B} \supseteq \closure{A} - \closure{B}
      \end{equation*}

      \begin{proof}
            If \mt{x \in \closure{A} - \closure{B}}. Then
            \begin{eqnarray*}
                  x \in \closure{A}, x \notin \closure{B}
            \end{eqnarray*}.

            Thus for open set \mt{U} containing \mt{x}
            \begin{eqnarray*}
                  &\exists& U_{1} \cap B = \emptyset \\
                  &\forall& U \cap A \neq \emptyset
            \end{eqnarray*}

            Suppose that \mt{x \notin \closure{A-B}}. Then
            \begin{eqnarray*}
                  \exists U_{0} \cap (A-B) = \emptyset
            \end{eqnarray*}

            Thus,
            \begin{eqnarray*}
                  U_{0} \cap A \subseteq B
            \end{eqnarray*}

            Thus,
            \begin{eqnarray*}
                  U_{1} \cap B &=& \emptyset \\
                  U_{1} \cap U_{0} \cap A &=& \emptyset
            \end{eqnarray*}

            As \mt{U_{1} \cap U_{0}} is an open set containing \mt{x}, so there is contradiction with \mt{x \in \closure{A}}. Thus \mt{x \in \closure{A-B}}.
      \end{proof}

      \item A \defineNewWord{diagonal}\label{def:Diagonal} is a subset \mt{
            \Delta = \{ x \times x | x \in \mathbb{X} \}
      } of the product topology \mtb{X \times X} where \mtb{X} is a topological space. Show that the diagonal is closed in \mtb{X \times X} \ioi \mtb{X} is a Hausdorff space.

      \begin{proof}
            If \mtb{X} is a Hausdorff space. For every element \productSet{x}{y} of \productSet{\mathbb{X}}{\mathbb{X}} that not in \mt{\Delta}. We take disjoint set \mt{U_{x},U_{y}} where \mt{
                  x \in U_{x}, y \in U_{y}
            }. Then \mt{
                  \mathbb{X} \times \mathbb{X} - \Delta = \cup_{x \neq y} U_{x} \times U_{y}
            }. Where \mt{\cup_{x \neq y} U_{x} \times U_{y}} is an open set. Thus \mt{\Delta} is a closed set.

            Conversely, if \mt{\Delta} is a closed set, suppose that \mtb{X} is not a Hausdorff space. Then there exists distinct \mt{x, y} such that every neighbourhood of \mt{x} and \mt{y} intersect. Let \mtb{B} be a basis of topology of \mtb{X}. Then \mt{
                  x \times y \in \mathbb{X} \times \mathbb{X} - \Delta
            }. However we cannot find \mt{
                  B_{1}, B_{2} \in \mathbb{B}, x \times y \in B_{1} \times B_{2} \subset \mathbb{X} \times \mathbb{X} - \Delta
            }. Then \mt{\Delta} is not a closed set. So there is a contradiction, then \mtb{X} must be a Hausdorff space.
      \end{proof}

      \item Prove that \mt{T_{1}} axiom is equivalent to the condition such that for every distinct pair \mt{x,y} of \mtb{X}, there exists neighbourhood of \mt{x} does not contain \mt{y}.
      
      \begin{proof}
            First if \mt{T_{1}} axiom hold, then for every pair \mt{x,y}, the neighbourhood \mt{\mathbb{X}-\{y\}} of \mt{x} does not contain \mt{y}, so the second condition hold.

            Conversely, if the second condition hold. Suppose that we can find a finite points set say \mt{\{x_{1}, x_{2}, x_{3} \dots}\}, then there must exists \mt{x \in \{x_{1}, x_{2}, x_{3} \dots}\} such that the set \mt{\{x\}} is not closed. Then \mt{
                  \closure{\{x\}} - \{x\} \neq \emptyset
            }. Let \mt{y \in \closure{\{x\}} - \{x\} }, then every neighbourhood of y must contain \mt{x}, this is a contradiction to the second condition, so the \mt{T_{1}} axiom must hold.
      \end{proof}

      \item If \mt{A \subseteq \mathbb{X}}, we define the \defineNewWord{boundary}\label{def:Boundary} of \mt{A} by the equation
      \begin{equation*}
            \text{Bd} A = \closure{A} \cap \closure{\mathbb{X}-A}
      \end{equation*}
      \begin{enumerate}
            \item Show that \mt{\text{Int}A} and \mt{\text{Bd} A} are disjoint and \mt{
                  \closure{A} = \text{Int} A \cup \text{Bd} A
            }.
            
            \begin{proof}
                  For every \mt{ x \in \boundary{A} }, every open set contain \mt{x} must intersect \mt{A} and \mt{\mathbb{X}-A} so, there is no open set \mt{U} contain \mt{x}, \mt{U \subseteq A}.

                  For every \mt{x' \in \interior{A}}, there exists \mt{U' \subseteq A}, so \mt{\boundary{A}} and \mt{\interior{A}} are disjoint sets.

                  For every \mt{x \in \closure{A}}, \mt{x \in \boundary{A}} or \mt{x \notin \boundary{A}}. We discuss the condition that \mt{x \notin \boundary{A}}.

                  Then \mt{x \notin \closure{\mathbb{X}-A}}, then there exists a open set \mt{U} containing \mt{x}, that does not intersect with \mt{\mathbb{X}-A}. Thus \mt{U \subseteq A}, thus \mt{x \in \interior{A}}. So \mt{
                        \closure{A} \subseteq \text{Int} A \cup \text{Bd} A
                  }.

                  Then, \mt{\boundary{A} \subseteq \closure{A}}, \mt{\interior{A} \subseteq A \subseteq \closure{A}}. Thus, \mt{
                        \closure{A} \supseteq \text{Int} A \cup \text{Bd} A
                  }

                  So, \mt{
                        \closure{A} = \text{Int} A \cup \text{Bd} A
                  }
            \end{proof}

            \item Show that \mt{\boundary{A} = \emptyset} \ioi \mt{A} is both open and closed.
            
            \begin{proof}
                  So, \mt{\interior{A} = \closure{A}}, then \mt{\boundary{A} = \emptyset} follows directly from \mt{
                        \closure{A} = \text{Int} A \cup \text{Bd} A
                  }.
            \end{proof}

            \item Show that \mt{U} is open \ioi \mt{\boundary{U} = \closure{U}-U}.
            
            \begin{proof}
                  Suppose U is open. Then \mt{\closure{\mathbb{X}-U} = \mathbb{X} - U}. Then for every \mt{x \in U}, \mt{x \notin \mathbb{X} - U, x \notin \closure{\mathbb{X}-U}}. Thus \mt{\closure{U} \cap \closure{\mathbb{X}-U}=\closure{U}-U}.

                  Conversely, suppose \mt{\boundary{U} = \closure{U}-U}. Then for every \mt{x \in U}, \mt{x \notin \boundary{U}}. Then as \mt{\closure{U} = \interior{U}\cup \boundary{U}}, \mt{x \in \interior{U}}. So \mt{\interior{U} \supseteq U}. Thus \mt{U = \interior{U}}. Thus, \mt{U} is open.
            \end{proof}

      \end{enumerate}
\end{enumerate}



      \section{Continuous Function}

\begin{definition}[continuous]\label{def:Continuous}\footnote{
      As the continuity of a function is different as the topological spaces are different. So if we want to emphasis this fact, we say that \mt{f} is continuous \defineNewWord{relative}\label{def:ContinuousRelativeTo} to specific topologies on \mtb{X} and \mtb{Y}.
}
      Let \mtb{X} and \mtb{Y} be topological spaces. A function \mt{f: \mathbb{X}\rightarrow \mathbb{Y}} is said to be \defineNewWord{continuous} if for each open subset \mt{V} of \tmb{Y}, the set \mt{f^{-1}(V)} is an open subset of \mtb{X}.
\end{definition}

\begin{theorem}
      Let \mtb{X} and \mtb{Y} be topological spaces; let \mt{f: \mathbb{X}\rightarrow\mathbb{Y}}. Then the following are equivalent.
      \begin{enumerate}
            \item \mt{f} is continuous.
            \item For every subset \mt{A} of \mt{X}, one has \mt{f(\closure{A})\subseteq\closure{f(A)}}.
            \item For every closed set \mt{B} of \mtb{Y}, the set \mt{f^{-1}(B)} is closed in \mtb{X}.
            \item For each \mt{x\in\mathbb{X}} and each neighbourhood of \mt{V} of \mt{f(x)}, there is a neighbourhood \mt{U} of \mt{x} such that \mt{f(U) \subseteq V}.
      \end{enumerate}
\end{theorem}

\begin{proof}

      \hspace{1em}

      1 \mt{\Rightarrow} 3:

      Let \mt{A} be a open set in \mtb{Y}. \mt{f^{-1}(\mathbb{Y}-A) = \mathbb{X} - f^{-1}(A)}.
      
      \vspace{1em}

      3 \mt{\Rightarrow} 1:

      Let \mt{A} be a closed set in \mtb{Y}. \mt{f^{-1}(\mathbb{Y}-A) = \mathbb{X} - f^{-1}(A)}.

      \vspace{1em}

      1 \mt{\Rightarrow} 2:

      For \mt{x \in \closure{A}}, we take a open set \mt{f(x) \in U \subseteq \mathbb{Y}}. Thus \mt{x \in f^{-1}(U) \cap A \neq \emptyset}. Thus \mt{U \cap f(A) \neq \emptyset}. So \mt{f(x) \in \closure{f(A)}}. Thus \mt{f(\closure{A})\subseteq\closure{f(A)}}.

      \vspace{1em}

      2 \mt{\Rightarrow} 3:

      Suppose \mt{f} is not continuous. Then there must exists \mt{V}, such that \mt{f^{-1}(V) = U} is not closed. Thus \mt{\closure{U} \supset B = f^{-1}(A)}. Thus \mt{f{\closure{B}} \supset A}. However \mt{f(\closure{B}) \subseteq \closure{f(B)} = A}. There is a contradiction. So \mt{f} must be continuous.

      \vspace{1em}

      1 \mt{\Rightarrow} 4:

      For every neighbourhood \mt{V} of \mt{f(x)}, \mt{f^{-1}(V)} is a neighbourhood of \mt{x} that \mt{f(f^{-1}(V)) \subseteq V}.

      \vspace{1em}

      4 \mt{\Rightarrow} 1:

      We take a open set \mt{V} of \mtb{Y}. Let \mt{S} be the collection of all open set \mt{U} in \mtb{X} such that \mt{f(U) \subseteq V}. The set cannot be empty unless \mt{f^{-1}(V) = \emptyset}. Let \mt{U_{0}} denote the union of all the element in \mt{S}. We prove that \mt{U_{0} = f^{-1}(V)}.

      For all element \mt{x \in U_{0}}, \mt{f(x) \in V}. Thus \mt{U_{0} \subseteq f^{-1}(V)}.

      For all element \mt{x \in f^{-1}(V)}. There is a \mt{U'} such that \mt{x \in U', f(U')\subseteq V}. This follows from the condition 4. Thus \mt{U' \in S}. Thus \mt{x \in U_{0}}. Thus \mt{U_{0} \subseteq f^{-1}(V)}. As \mt{U_{0}} is union of open set, \mt{U_{0}} is also open. Thus, \mt{f^{-1}(V)} is also open.
      
      Thus \mt{f} is continuous.
\end{proof}

\begin{definition}[homeomorphism]\label{def:Homeomorphism}\footnote{
      A equivalent way to define homeomorphism, is that for any open subset \mt{U} of \mtb{X}, \mt{f(U)} is open \ioi \mt{U} is open.
}
      Let \mtb{X} and \mtb{Y} be topological space; let \mt{f: \mathbb{X} \rightarrow \mathbb{Y}} be a bijection. If both the function \mt{f} and the inverse function
      \begin{equation*}
            f^{-1}: \mathbb{Y} \rightarrow \mathbb{X}
      \end{equation*}
      are continuous, then f is called a \defineNewWord{homeomorphism}
\end{definition}

\begin{definition}[topological imbedding]\label{def:TopologicalImbedding}
      Suppose that \mt{f: \mathbb{X} \rightarrow \mathbb{Y}} is an injective continuous map, where \mtb{X} and \mtb{Y} are topological spaces. Let \mtb{Z} be the image set \mt{f(\mathbb{X})}, considered as a subspace of \mtb{Y}; then the function \mt{f': \mathbb{X} \rightarrow \mathbb{Z}} obtained by restricting the range of \mt{f} is bijective. If \mt{f'} happens to be a homeomorphism of \mtb{X} with \mtb{Z}, we say that the map \mt{f: \mathbb{X} \rightarrow \mathbb{Y}} is a \defineNewWord{topological imbedding}, or simply an \defineNewWord{imbedding}, of \mtb{X} in \mtb{Y}.
\end{definition}

\begin{theorem}[Rules for constructing continuous functions]\label{theorem:RulesForConstructingContinuousFunctions}
      Let \mtb{X}, \mtb{Y}, and \mtb{Z} be topological spaces.
      \begin{enumerate}
            \item (Constant function) If \mt{f: \mathbb{X} \rightarrow \mathbb{Y}} maps all of \mtb{X} into the single point \mt{y_{0}} of \mtb{Y}, then \mt{f} is continuous.
            \item (Inclusion) If \mt{A} is a subspace of \mtb{X}, the inclusion function \mt{j: A \rightarrow \mathbb{X}} is continuous.
            \item (Composites) If \mt{f: \mathbb{X}\rightarrow \mathbb{Y}} and \mt{g:\mathbb{Y}\rightarrow\mathbb{Z}} are continuous, then the map \mt{g \circ f: \mathbb{X} \rightarrow \mathbb{Z}} is continuous.
            \item (Restricting the domain) If \mt{f: \mathbb{X} \rightarrow \mathbb{Y}} is continuous, and if \mt{A} is a subspace of \mtb{X}, then the restriction function \mt{f|A : A \rightarrow \mathbb{Y}} is continuous.
            \item (Restricting or expanding the range) Let \mt{f:\mathbb{X}\rightarrow\mathbb{Y}} is continuous. Let \mtb{Z} be a subspace of \mtb{Y} containing the image \mt{f(\mathbb{X})}, the function \mt{h: \mathbb{X}\rightarrow \mathbb{Z}} obtained by restricting the range of \mt{f} is continuous. If \mtb{Z} is a space having \mtb{Y} as a subspace, then the function \mt{h: \mathbb{X}\rightarrow\mathbb{Y}} obtained by expanding the range of \mt{f} is continuous.
            \item (Local formulation of continuity) The map \mt{f: \mathbb{X}\rightarrow\mathbb{Y}} is continuous if \mtb{X} can be written as the union of open sets \mt{U_{\alpha}} such set \mt{f|U_{\alpha}} is continuous for each \mt{\alpha}
      \end{enumerate}
\end{theorem}

\begin{proof}

      \hspace{1em}

      \begin{enumerate}
            \item \mt{f^{-1}(U)} of any open set \mt{U} is \mtb{X}, thus \mt{f} is continuous.
            \item For every open subset \mt{U} of \mtb{X}, \mt{j^{-1}(U) = U\cap A} is continuous in \mt{A}. Thus \mt{j} is a continuous function.
            \item For every open subset \mt{U} of \mtb{Z}, \mt{f^{-1}(U)} is open in \mtb{Y}, and \mt{g^{-1}(f^{-1}(U))} is open in \mtb{X}. Thus, \mt{g \circ f} is continuous
            \item For every open subset \mt{U} of \mtb{Y}, \mt{f^{-1}(U)} is open in \mtb{X}, thus \mt{f^{-1}(U)\cap A} is open in \mt{A}. Thus the function \mt{f|A} is continuous.
            \item If \mtb{Z} is a subspace of \mtb{Y}, then every open subset of \mtb{Z} can be represented as \mt{U\cap\mathbb{Z}}, where \mt{U} is a open subset of \mtb{Y}. Thus \mt{h^{-1}(U\cap\mathbb{Z})=g^{-1}(\mathbb{Z})\cap g^{-1}(U) = \mathbb{X}\cap g^{-1}(U)} which is a open subset of \mt{X}, thus \mt{h} is continuous.
            
            If \mtb{Y} is a subspace of \mtb{Z}. Then we take a open subset \mt{U} of \tmb{Z}. \mt{h^{-1}(U) = g^(-1)(U\cap \mathbb{Y}) } which is open in \mtb{X}, thus \mt{h} is continuous.

            \item if \mt{f|U_{\alpha}} is continuous for each \mt{\alpha}. For every open subset \mt{U} of \mtb{Y}.
            \begin{equation*}
                  U = \cup_{\alpha} (U_{\alpha}\cap U)
            \end{equation*}
            where \mt{U_{\alpha}\cap U} is open both in \mt{U_{\alpha}} and in \mtb{Y}. 

            Thus,
            \begin{eqnarray*}
                  f^{-1}(U) &=& f^{-1}(\cup_{\alpha} (U_{\alpha}\cap U)) \\
                  &=& \cup_{\alpha} ((f|U_{\alpha})^{-1}(U_{\alpha}\cap U))
            \end{eqnarray*}
            and each \mt{(f|U_{\alpha})^{-1}(U_{\alpha}\cap U)} is open, thus \mt{f^{-1}(U)} is open.
      \end{enumerate}
\end{proof}

\begin{theorem}[The pasting lemma]\label{theorem:ThePastingLemma}\footnote{
      The proof of this theorem is similar to the "Local formulation of continuity" condition of "Rules for constructing continuous functions", so we omit the proof here.
}
      Let \mt{
            \mathbb{X} = A \cup B
      }, where \mt{A,B} are closed in \mtb{X}. Let \mt{f: A\rightarrow \mathbb{Y}} and \mt{g: B \rightarrow \mathbb{Y}} be continuous. If \mt{f(x)=g(x)} for every \mt{x\in A\cap B}, then \mt{f,g} combine to give a continuous function \mt{h: \mathbb{X}\rightarrow\mathbb{Y}}, defined by setting \mt{h(x)=f(x),x\in A} and \mt{h(x)=g(x),x\in B}.
\end{theorem}

\begin{theorem}[Maps into products]\label{theorem:MapsIntoProducts} \footnote{
      The map \mt{f_{1},f_{2}} are called the \defineNewWord{coordinate functions}\label{def:CoordinateFunctions} of \mt{f}
}
      Let \mt{f: A \rightarrow \mathbb{X}\times\mathbb{Y}} be given by the equation
      \begin{equation*}
            f(a) = (f_{1}(a),f_{2}(a))
      \end{equation*}

      Then, the function \mt{f} is continuous \ioi the functions
      \begin{equation*}
            f_{1}: A \rightarrow \mathbb{X}, f_{2}: A \rightarrow \mathbb{Y}
      \end{equation*}
      are continuous.
\end{theorem}

\begin{proof}
      Let \mt{\pi_{1},\pi_{2}} be the projection function
      \begin{eqnarray*}
            \pi_{1}&:& \mathbb{X}\times\mathbb{Y} \rightarrow \mathbb{X} \\
            \pi_{2}&:& \mathbb{X}\times\mathbb{Y} \rightarrow \mathbb{Y}
      \end{eqnarray*}

      \vspace{1em}

      We first proof that if \mt{U} is an open subset of \productSet{\mathbb{X}}{\mathbb{Y}},
      \begin{equation*}
            f^{-1}(U) = f_{1}^{-1}(\pi_{1}(U)) \cap f_{2}^{-1}(\pi_{2}(U))
      \end{equation*}

      Let \mt{x \times y \in U}, \mt{f^{-1}(x \times y)} contains all \mt{a} such that \mt{f(a) = x \times y}.

      Then for any \mt{a \in f^{-1}(x \times y)}, \mt{a \in f_{1}^{-1}(\pi_{1}(x \times y)), a \in f_{2}^{-1}(\pi_{2}(x \times y))}.

      Thus, \mt{f^{-1}(x \times y) \subseteq f_{1}^{-1}(\pi_{1}(x \times y)) \cap f_{2}^{-1}(\pi_{2}(x \times y))}.

      Thus \mt{f^{-1}(U) \subseteq f_{1}^{-1}(\pi_{1}(U)) \cap f_{2}^{-1}(\pi_{2}(U))}.

      \vspace{1em}

      Also, if \mt{a \in f_{1}^{-1}(\pi_{1}(x \times y)), a \in f_{2}^{-1}(\pi_{2}(x \times y))}, \mt{f_{1}(a) = x, f_{2}(a) = y}.

      Thus \mt{f(a) = x \times y}.
      Thus \mt{a \in f^{-1}(x \times y)}.

      Thus \mt{f^{-1}(U) = f_{1}^{-1}(\pi_{1}(U)) \cap f_{2}^{-1}(\pi_{2}(U))}

      \vspace{1em}

      Let \mt{U} be any open subset of \productSet{\mathbb{X}}{\mathbb{Y}}

      \begin{equation*}
            f^{-1}(U) = f_{1}^{-1}(\pi_{1}(U)) \cap f_{2}^{-1}(\pi_{2}(U))
      \end{equation*}

      Where \mt{f_{1}^{-1}(\pi_{1}(U))} and \mt{f_{2}^{-1}(\pi_{2}(U))} are both open set. Thus \mt{f^{-1}(U)} is open.
\end{proof}
      \subsection{Exercise}

\begin{enumerate}
      \item Give an counter example why \mt{
            \closure{\cup A_{\alpha}} = \cup \closure{A_{\alpha}}
      } dose not hold.

      \begin{proof}
            Consider the X be the K-topology on the real line.
      
            Let
            \begin{eqnarray*}
                  A_{n} &=& (\frac{1}{n+1},\frac{1}{n}), n \in \mathbb{Z}_{+} \\
                  A &=& \cup A_{n}
            \end{eqnarray*}
      
            Then 
            \begin{eqnarray*}
                  \closure{A_{n}} &=& [\frac{1}{n+1},\frac{1}{n}] \\
                  \cup \closure{A_{n}} &=& (0,1]
            \end{eqnarray*}
      
            However, as every neighbourhood of \mt{0} intersect \mt{\cup A_{\alpha}}. \mt{0 \in \closure{\cup A_{\alpha}}}.
      
            Thus, \mt{
                  \closure{\cup A_{\alpha}} \neq \cup \closure{A_{\alpha}}
            }
      \end{proof}

      \item Prove that 
      \begin{equation*}
            \closure{A-B} \supseteq \closure{A} - \closure{B}
      \end{equation*}

      \begin{proof}
            If \mt{x \in \closure{A} - \closure{B}}. Then
            \begin{eqnarray*}
                  x \in \closure{A}, x \notin \closure{B}
            \end{eqnarray*}.

            Thus for open set \mt{U} containing \mt{x}
            \begin{eqnarray*}
                  &\exists& U_{1} \cap B = \emptyset \\
                  &\forall& U \cap A \neq \emptyset
            \end{eqnarray*}

            Suppose that \mt{x \notin \closure{A-B}}. Then
            \begin{eqnarray*}
                  \exists U_{0} \cap (A-B) = \emptyset
            \end{eqnarray*}

            Thus,
            \begin{eqnarray*}
                  U_{0} \cap A \subseteq B
            \end{eqnarray*}

            Thus,
            \begin{eqnarray*}
                  U_{1} \cap B &=& \emptyset \\
                  U_{1} \cap U_{0} \cap A &=& \emptyset
            \end{eqnarray*}

            As \mt{U_{1} \cap U_{0}} is an open set containing \mt{x}, so there is contradiction with \mt{x \in \closure{A}}. Thus \mt{x \in \closure{A-B}}.
      \end{proof}

      \item A \defineNewWord{diagonal}\label{def:Diagonal} is a subset \mt{
            \Delta = \{ x \times x | x \in \mathbb{X} \}
      } of the product topology \mtb{X \times X} where \mtb{X} is a topological space. Show that the diagonal is closed in \mtb{X \times X} \ioi \mtb{X} is a Hausdorff space.

      \begin{proof}
            If \mtb{X} is a Hausdorff space. For every element \productSet{x}{y} of \productSet{\mathbb{X}}{\mathbb{X}} that not in \mt{\Delta}. We take disjoint set \mt{U_{x},U_{y}} where \mt{
                  x \in U_{x}, y \in U_{y}
            }. Then \mt{
                  \mathbb{X} \times \mathbb{X} - \Delta = \cup_{x \neq y} U_{x} \times U_{y}
            }. Where \mt{\cup_{x \neq y} U_{x} \times U_{y}} is an open set. Thus \mt{\Delta} is a closed set.

            Conversely, if \mt{\Delta} is a closed set, suppose that \mtb{X} is not a Hausdorff space. Then there exists distinct \mt{x, y} such that every neighbourhood of \mt{x} and \mt{y} intersect. Let \mtb{B} be a basis of topology of \mtb{X}. Then \mt{
                  x \times y \in \mathbb{X} \times \mathbb{X} - \Delta
            }. However we cannot find \mt{
                  B_{1}, B_{2} \in \mathbb{B}, x \times y \in B_{1} \times B_{2} \subset \mathbb{X} \times \mathbb{X} - \Delta
            }. Then \mt{\Delta} is not a closed set. So there is a contradiction, then \mtb{X} must be a Hausdorff space.
      \end{proof}

      \item Prove that \mt{T_{1}} axiom is equivalent to the condition such that for every distinct pair \mt{x,y} of \mtb{X}, there exists neighbourhood of \mt{x} does not contain \mt{y}.
      
      \begin{proof}
            First if \mt{T_{1}} axiom hold, then for every pair \mt{x,y}, the neighbourhood \mt{\mathbb{X}-\{y\}} of \mt{x} does not contain \mt{y}, so the second condition hold.

            Conversely, if the second condition hold. Suppose that we can find a finite points set say \mt{\{x_{1}, x_{2}, x_{3} \dots}\}, then there must exists \mt{x \in \{x_{1}, x_{2}, x_{3} \dots}\} such that the set \mt{\{x\}} is not closed. Then \mt{
                  \closure{\{x\}} - \{x\} \neq \emptyset
            }. Let \mt{y \in \closure{\{x\}} - \{x\} }, then every neighbourhood of y must contain \mt{x}, this is a contradiction to the second condition, so the \mt{T_{1}} axiom must hold.
      \end{proof}

      \item If \mt{A \subseteq \mathbb{X}}, we define the \defineNewWord{boundary}\label{def:Boundary} of \mt{A} by the equation
      \begin{equation*}
            \text{Bd} A = \closure{A} \cap \closure{\mathbb{X}-A}
      \end{equation*}
      \begin{enumerate}
            \item Show that \mt{\text{Int}A} and \mt{\text{Bd} A} are disjoint and \mt{
                  \closure{A} = \text{Int} A \cup \text{Bd} A
            }.
            
            \begin{proof}
                  For every \mt{ x \in \boundary{A} }, every open set contain \mt{x} must intersect \mt{A} and \mt{\mathbb{X}-A} so, there is no open set \mt{U} contain \mt{x}, \mt{U \subseteq A}.

                  For every \mt{x' \in \interior{A}}, there exists \mt{U' \subseteq A}, so \mt{\boundary{A}} and \mt{\interior{A}} are disjoint sets.

                  For every \mt{x \in \closure{A}}, \mt{x \in \boundary{A}} or \mt{x \notin \boundary{A}}. We discuss the condition that \mt{x \notin \boundary{A}}.

                  Then \mt{x \notin \closure{\mathbb{X}-A}}, then there exists a open set \mt{U} containing \mt{x}, that does not intersect with \mt{\mathbb{X}-A}. Thus \mt{U \subseteq A}, thus \mt{x \in \interior{A}}. So \mt{
                        \closure{A} \subseteq \text{Int} A \cup \text{Bd} A
                  }.

                  Then, \mt{\boundary{A} \subseteq \closure{A}}, \mt{\interior{A} \subseteq A \subseteq \closure{A}}. Thus, \mt{
                        \closure{A} \supseteq \text{Int} A \cup \text{Bd} A
                  }

                  So, \mt{
                        \closure{A} = \text{Int} A \cup \text{Bd} A
                  }
            \end{proof}

            \item Show that \mt{\boundary{A} = \emptyset} \ioi \mt{A} is both open and closed.
            
            \begin{proof}
                  So, \mt{\interior{A} = \closure{A}}, then \mt{\boundary{A} = \emptyset} follows directly from \mt{
                        \closure{A} = \text{Int} A \cup \text{Bd} A
                  }.
            \end{proof}

            \item Show that \mt{U} is open \ioi \mt{\boundary{U} = \closure{U}-U}.
            
            \begin{proof}
                  Suppose U is open. Then \mt{\closure{\mathbb{X}-U} = \mathbb{X} - U}. Then for every \mt{x \in U}, \mt{x \notin \mathbb{X} - U, x \notin \closure{\mathbb{X}-U}}. Thus \mt{\closure{U} \cap \closure{\mathbb{X}-U}=\closure{U}-U}.

                  Conversely, suppose \mt{\boundary{U} = \closure{U}-U}. Then for every \mt{x \in U}, \mt{x \notin \boundary{U}}. Then as \mt{\closure{U} = \interior{U}\cup \boundary{U}}, \mt{x \in \interior{U}}. So \mt{\interior{U} \supseteq U}. Thus \mt{U = \interior{U}}. Thus, \mt{U} is open.
            \end{proof}

      \end{enumerate}
\end{enumerate}



      \section{Continuous Function}

\begin{definition}[continuous]\label{def:Continuous}\footnote{
      As the continuity of a function is different as the topological spaces are different. So if we want to emphasis this fact, we say that \mt{f} is continuous \defineNewWord{relative}\label{def:ContinuousRelativeTo} to specific topologies on \mtb{X} and \mtb{Y}.
}
      Let \mtb{X} and \mtb{Y} be topological spaces. A function \mt{f: \mathbb{X}\rightarrow \mathbb{Y}} is said to be \defineNewWord{continuous} if for each open subset \mt{V} of \tmb{Y}, the set \mt{f^{-1}(V)} is an open subset of \mtb{X}.
\end{definition}

\begin{theorem}
      Let \mtb{X} and \mtb{Y} be topological spaces; let \mt{f: \mathbb{X}\rightarrow\mathbb{Y}}. Then the following are equivalent.
      \begin{enumerate}
            \item \mt{f} is continuous.
            \item For every subset \mt{A} of \mt{X}, one has \mt{f(\closure{A})\subseteq\closure{f(A)}}.
            \item For every closed set \mt{B} of \mtb{Y}, the set \mt{f^{-1}(B)} is closed in \mtb{X}.
            \item For each \mt{x\in\mathbb{X}} and each neighbourhood of \mt{V} of \mt{f(x)}, there is a neighbourhood \mt{U} of \mt{x} such that \mt{f(U) \subseteq V}.
      \end{enumerate}
\end{theorem}

\begin{proof}

      \hspace{1em}

      1 \mt{\Rightarrow} 3:

      Let \mt{A} be a open set in \mtb{Y}. \mt{f^{-1}(\mathbb{Y}-A) = \mathbb{X} - f^{-1}(A)}.
      
      \vspace{1em}

      3 \mt{\Rightarrow} 1:

      Let \mt{A} be a closed set in \mtb{Y}. \mt{f^{-1}(\mathbb{Y}-A) = \mathbb{X} - f^{-1}(A)}.

      \vspace{1em}

      1 \mt{\Rightarrow} 2:

      For \mt{x \in \closure{A}}, we take a open set \mt{f(x) \in U \subseteq \mathbb{Y}}. Thus \mt{x \in f^{-1}(U) \cap A \neq \emptyset}. Thus \mt{U \cap f(A) \neq \emptyset}. So \mt{f(x) \in \closure{f(A)}}. Thus \mt{f(\closure{A})\subseteq\closure{f(A)}}.

      \vspace{1em}

      2 \mt{\Rightarrow} 3:

      Suppose \mt{f} is not continuous. Then there must exists \mt{V}, such that \mt{f^{-1}(V) = U} is not closed. Thus \mt{\closure{U} \supset B = f^{-1}(A)}. Thus \mt{f{\closure{B}} \supset A}. However \mt{f(\closure{B}) \subseteq \closure{f(B)} = A}. There is a contradiction. So \mt{f} must be continuous.

      \vspace{1em}

      1 \mt{\Rightarrow} 4:

      For every neighbourhood \mt{V} of \mt{f(x)}, \mt{f^{-1}(V)} is a neighbourhood of \mt{x} that \mt{f(f^{-1}(V)) \subseteq V}.

      \vspace{1em}

      4 \mt{\Rightarrow} 1:

      We take a open set \mt{V} of \mtb{Y}. Let \mt{S} be the collection of all open set \mt{U} in \mtb{X} such that \mt{f(U) \subseteq V}. The set cannot be empty unless \mt{f^{-1}(V) = \emptyset}. Let \mt{U_{0}} denote the union of all the element in \mt{S}. We prove that \mt{U_{0} = f^{-1}(V)}.

      For all element \mt{x \in U_{0}}, \mt{f(x) \in V}. Thus \mt{U_{0} \subseteq f^{-1}(V)}.

      For all element \mt{x \in f^{-1}(V)}. There is a \mt{U'} such that \mt{x \in U', f(U')\subseteq V}. This follows from the condition 4. Thus \mt{U' \in S}. Thus \mt{x \in U_{0}}. Thus \mt{U_{0} \subseteq f^{-1}(V)}. As \mt{U_{0}} is union of open set, \mt{U_{0}} is also open. Thus, \mt{f^{-1}(V)} is also open.
      
      Thus \mt{f} is continuous.
\end{proof}

\begin{definition}[homeomorphism]\label{def:Homeomorphism}\footnote{
      A equivalent way to define homeomorphism, is that for any open subset \mt{U} of \mtb{X}, \mt{f(U)} is open \ioi \mt{U} is open.
}
      Let \mtb{X} and \mtb{Y} be topological space; let \mt{f: \mathbb{X} \rightarrow \mathbb{Y}} be a bijection. If both the function \mt{f} and the inverse function
      \begin{equation*}
            f^{-1}: \mathbb{Y} \rightarrow \mathbb{X}
      \end{equation*}
      are continuous, then f is called a \defineNewWord{homeomorphism}
\end{definition}

\begin{definition}[topological imbedding]\label{def:TopologicalImbedding}
      Suppose that \mt{f: \mathbb{X} \rightarrow \mathbb{Y}} is an injective continuous map, where \mtb{X} and \mtb{Y} are topological spaces. Let \mtb{Z} be the image set \mt{f(\mathbb{X})}, considered as a subspace of \mtb{Y}; then the function \mt{f': \mathbb{X} \rightarrow \mathbb{Z}} obtained by restricting the range of \mt{f} is bijective. If \mt{f'} happens to be a homeomorphism of \mtb{X} with \mtb{Z}, we say that the map \mt{f: \mathbb{X} \rightarrow \mathbb{Y}} is a \defineNewWord{topological imbedding}, or simply an \defineNewWord{imbedding}, of \mtb{X} in \mtb{Y}.
\end{definition}

\begin{theorem}[Rules for constructing continuous functions]\label{theorem:RulesForConstructingContinuousFunctions}
      Let \mtb{X}, \mtb{Y}, and \mtb{Z} be topological spaces.
      \begin{enumerate}
            \item (Constant function) If \mt{f: \mathbb{X} \rightarrow \mathbb{Y}} maps all of \mtb{X} into the single point \mt{y_{0}} of \mtb{Y}, then \mt{f} is continuous.
            \item (Inclusion) If \mt{A} is a subspace of \mtb{X}, the inclusion function \mt{j: A \rightarrow \mathbb{X}} is continuous.
            \item (Composites) If \mt{f: \mathbb{X}\rightarrow \mathbb{Y}} and \mt{g:\mathbb{Y}\rightarrow\mathbb{Z}} are continuous, then the map \mt{g \circ f: \mathbb{X} \rightarrow \mathbb{Z}} is continuous.
            \item (Restricting the domain) If \mt{f: \mathbb{X} \rightarrow \mathbb{Y}} is continuous, and if \mt{A} is a subspace of \mtb{X}, then the restriction function \mt{f|A : A \rightarrow \mathbb{Y}} is continuous.
            \item (Restricting or expanding the range) Let \mt{f:\mathbb{X}\rightarrow\mathbb{Y}} is continuous. Let \mtb{Z} be a subspace of \mtb{Y} containing the image \mt{f(\mathbb{X})}, the function \mt{h: \mathbb{X}\rightarrow \mathbb{Z}} obtained by restricting the range of \mt{f} is continuous. If \mtb{Z} is a space having \mtb{Y} as a subspace, then the function \mt{h: \mathbb{X}\rightarrow\mathbb{Y}} obtained by expanding the range of \mt{f} is continuous.
            \item (Local formulation of continuity) The map \mt{f: \mathbb{X}\rightarrow\mathbb{Y}} is continuous if \mtb{X} can be written as the union of open sets \mt{U_{\alpha}} such set \mt{f|U_{\alpha}} is continuous for each \mt{\alpha}
      \end{enumerate}
\end{theorem}

\begin{proof}

      \hspace{1em}

      \begin{enumerate}
            \item \mt{f^{-1}(U)} of any open set \mt{U} is \mtb{X}, thus \mt{f} is continuous.
            \item For every open subset \mt{U} of \mtb{X}, \mt{j^{-1}(U) = U\cap A} is continuous in \mt{A}. Thus \mt{j} is a continuous function.
            \item For every open subset \mt{U} of \mtb{Z}, \mt{f^{-1}(U)} is open in \mtb{Y}, and \mt{g^{-1}(f^{-1}(U))} is open in \mtb{X}. Thus, \mt{g \circ f} is continuous
            \item For every open subset \mt{U} of \mtb{Y}, \mt{f^{-1}(U)} is open in \mtb{X}, thus \mt{f^{-1}(U)\cap A} is open in \mt{A}. Thus the function \mt{f|A} is continuous.
            \item If \mtb{Z} is a subspace of \mtb{Y}, then every open subset of \mtb{Z} can be represented as \mt{U\cap\mathbb{Z}}, where \mt{U} is a open subset of \mtb{Y}. Thus \mt{h^{-1}(U\cap\mathbb{Z})=g^{-1}(\mathbb{Z})\cap g^{-1}(U) = \mathbb{X}\cap g^{-1}(U)} which is a open subset of \mt{X}, thus \mt{h} is continuous.
            
            If \mtb{Y} is a subspace of \mtb{Z}. Then we take a open subset \mt{U} of \tmb{Z}. \mt{h^{-1}(U) = g^(-1)(U\cap \mathbb{Y}) } which is open in \mtb{X}, thus \mt{h} is continuous.

            \item if \mt{f|U_{\alpha}} is continuous for each \mt{\alpha}. For every open subset \mt{U} of \mtb{Y}.
            \begin{equation*}
                  U = \cup_{\alpha} (U_{\alpha}\cap U)
            \end{equation*}
            where \mt{U_{\alpha}\cap U} is open both in \mt{U_{\alpha}} and in \mtb{Y}. 

            Thus,
            \begin{eqnarray*}
                  f^{-1}(U) &=& f^{-1}(\cup_{\alpha} (U_{\alpha}\cap U)) \\
                  &=& \cup_{\alpha} ((f|U_{\alpha})^{-1}(U_{\alpha}\cap U))
            \end{eqnarray*}
            and each \mt{(f|U_{\alpha})^{-1}(U_{\alpha}\cap U)} is open, thus \mt{f^{-1}(U)} is open.
      \end{enumerate}
\end{proof}

\begin{theorem}[The pasting lemma]\label{theorem:ThePastingLemma}\footnote{
      The proof of this theorem is similar to the "Local formulation of continuity" condition of "Rules for constructing continuous functions", so we omit the proof here.
}
      Let \mt{
            \mathbb{X} = A \cup B
      }, where \mt{A,B} are closed in \mtb{X}. Let \mt{f: A\rightarrow \mathbb{Y}} and \mt{g: B \rightarrow \mathbb{Y}} be continuous. If \mt{f(x)=g(x)} for every \mt{x\in A\cap B}, then \mt{f,g} combine to give a continuous function \mt{h: \mathbb{X}\rightarrow\mathbb{Y}}, defined by setting \mt{h(x)=f(x),x\in A} and \mt{h(x)=g(x),x\in B}.
\end{theorem}

\begin{theorem}[Maps into products]\label{theorem:MapsIntoProducts} \footnote{
      The map \mt{f_{1},f_{2}} are called the \defineNewWord{coordinate functions}\label{def:CoordinateFunctions} of \mt{f}
}
      Let \mt{f: A \rightarrow \mathbb{X}\times\mathbb{Y}} be given by the equation
      \begin{equation*}
            f(a) = (f_{1}(a),f_{2}(a))
      \end{equation*}

      Then, the function \mt{f} is continuous \ioi the functions
      \begin{equation*}
            f_{1}: A \rightarrow \mathbb{X}, f_{2}: A \rightarrow \mathbb{Y}
      \end{equation*}
      are continuous.
\end{theorem}

\begin{proof}
      Let \mt{\pi_{1},\pi_{2}} be the projection function
      \begin{eqnarray*}
            \pi_{1}&:& \mathbb{X}\times\mathbb{Y} \rightarrow \mathbb{X} \\
            \pi_{2}&:& \mathbb{X}\times\mathbb{Y} \rightarrow \mathbb{Y}
      \end{eqnarray*}

      \vspace{1em}

      We first proof that if \mt{U} is an open subset of \productSet{\mathbb{X}}{\mathbb{Y}},
      \begin{equation*}
            f^{-1}(U) = f_{1}^{-1}(\pi_{1}(U)) \cap f_{2}^{-1}(\pi_{2}(U))
      \end{equation*}

      Let \mt{x \times y \in U}, \mt{f^{-1}(x \times y)} contains all \mt{a} such that \mt{f(a) = x \times y}.

      Then for any \mt{a \in f^{-1}(x \times y)}, \mt{a \in f_{1}^{-1}(\pi_{1}(x \times y)), a \in f_{2}^{-1}(\pi_{2}(x \times y))}.

      Thus, \mt{f^{-1}(x \times y) \subseteq f_{1}^{-1}(\pi_{1}(x \times y)) \cap f_{2}^{-1}(\pi_{2}(x \times y))}.

      Thus \mt{f^{-1}(U) \subseteq f_{1}^{-1}(\pi_{1}(U)) \cap f_{2}^{-1}(\pi_{2}(U))}.

      \vspace{1em}

      Also, if \mt{a \in f_{1}^{-1}(\pi_{1}(x \times y)), a \in f_{2}^{-1}(\pi_{2}(x \times y))}, \mt{f_{1}(a) = x, f_{2}(a) = y}.

      Thus \mt{f(a) = x \times y}.
      Thus \mt{a \in f^{-1}(x \times y)}.

      Thus \mt{f^{-1}(U) = f_{1}^{-1}(\pi_{1}(U)) \cap f_{2}^{-1}(\pi_{2}(U))}

      \vspace{1em}

      Let \mt{U} be any open subset of \productSet{\mathbb{X}}{\mathbb{Y}}

      \begin{equation*}
            f^{-1}(U) = f_{1}^{-1}(\pi_{1}(U)) \cap f_{2}^{-1}(\pi_{2}(U))
      \end{equation*}

      Where \mt{f_{1}^{-1}(\pi_{1}(U))} and \mt{f_{2}^{-1}(\pi_{2}(U))} are both open set. Thus \mt{f^{-1}(U)} is open.
\end{proof}
      \subsection{Exercise}

\begin{enumerate}
      \item Give an counter example why \mt{
            \closure{\cup A_{\alpha}} = \cup \closure{A_{\alpha}}
      } dose not hold.

      \begin{proof}
            Consider the X be the K-topology on the real line.
      
            Let
            \begin{eqnarray*}
                  A_{n} &=& (\frac{1}{n+1},\frac{1}{n}), n \in \mathbb{Z}_{+} \\
                  A &=& \cup A_{n}
            \end{eqnarray*}
      
            Then 
            \begin{eqnarray*}
                  \closure{A_{n}} &=& [\frac{1}{n+1},\frac{1}{n}] \\
                  \cup \closure{A_{n}} &=& (0,1]
            \end{eqnarray*}
      
            However, as every neighbourhood of \mt{0} intersect \mt{\cup A_{\alpha}}. \mt{0 \in \closure{\cup A_{\alpha}}}.
      
            Thus, \mt{
                  \closure{\cup A_{\alpha}} \neq \cup \closure{A_{\alpha}}
            }
      \end{proof}

      \item Prove that 
      \begin{equation*}
            \closure{A-B} \supseteq \closure{A} - \closure{B}
      \end{equation*}

      \begin{proof}
            If \mt{x \in \closure{A} - \closure{B}}. Then
            \begin{eqnarray*}
                  x \in \closure{A}, x \notin \closure{B}
            \end{eqnarray*}.

            Thus for open set \mt{U} containing \mt{x}
            \begin{eqnarray*}
                  &\exists& U_{1} \cap B = \emptyset \\
                  &\forall& U \cap A \neq \emptyset
            \end{eqnarray*}

            Suppose that \mt{x \notin \closure{A-B}}. Then
            \begin{eqnarray*}
                  \exists U_{0} \cap (A-B) = \emptyset
            \end{eqnarray*}

            Thus,
            \begin{eqnarray*}
                  U_{0} \cap A \subseteq B
            \end{eqnarray*}

            Thus,
            \begin{eqnarray*}
                  U_{1} \cap B &=& \emptyset \\
                  U_{1} \cap U_{0} \cap A &=& \emptyset
            \end{eqnarray*}

            As \mt{U_{1} \cap U_{0}} is an open set containing \mt{x}, so there is contradiction with \mt{x \in \closure{A}}. Thus \mt{x \in \closure{A-B}}.
      \end{proof}

      \item A \defineNewWord{diagonal}\label{def:Diagonal} is a subset \mt{
            \Delta = \{ x \times x | x \in \mathbb{X} \}
      } of the product topology \mtb{X \times X} where \mtb{X} is a topological space. Show that the diagonal is closed in \mtb{X \times X} \ioi \mtb{X} is a Hausdorff space.

      \begin{proof}
            If \mtb{X} is a Hausdorff space. For every element \productSet{x}{y} of \productSet{\mathbb{X}}{\mathbb{X}} that not in \mt{\Delta}. We take disjoint set \mt{U_{x},U_{y}} where \mt{
                  x \in U_{x}, y \in U_{y}
            }. Then \mt{
                  \mathbb{X} \times \mathbb{X} - \Delta = \cup_{x \neq y} U_{x} \times U_{y}
            }. Where \mt{\cup_{x \neq y} U_{x} \times U_{y}} is an open set. Thus \mt{\Delta} is a closed set.

            Conversely, if \mt{\Delta} is a closed set, suppose that \mtb{X} is not a Hausdorff space. Then there exists distinct \mt{x, y} such that every neighbourhood of \mt{x} and \mt{y} intersect. Let \mtb{B} be a basis of topology of \mtb{X}. Then \mt{
                  x \times y \in \mathbb{X} \times \mathbb{X} - \Delta
            }. However we cannot find \mt{
                  B_{1}, B_{2} \in \mathbb{B}, x \times y \in B_{1} \times B_{2} \subset \mathbb{X} \times \mathbb{X} - \Delta
            }. Then \mt{\Delta} is not a closed set. So there is a contradiction, then \mtb{X} must be a Hausdorff space.
      \end{proof}

      \item Prove that \mt{T_{1}} axiom is equivalent to the condition such that for every distinct pair \mt{x,y} of \mtb{X}, there exists neighbourhood of \mt{x} does not contain \mt{y}.
      
      \begin{proof}
            First if \mt{T_{1}} axiom hold, then for every pair \mt{x,y}, the neighbourhood \mt{\mathbb{X}-\{y\}} of \mt{x} does not contain \mt{y}, so the second condition hold.

            Conversely, if the second condition hold. Suppose that we can find a finite points set say \mt{\{x_{1}, x_{2}, x_{3} \dots}\}, then there must exists \mt{x \in \{x_{1}, x_{2}, x_{3} \dots}\} such that the set \mt{\{x\}} is not closed. Then \mt{
                  \closure{\{x\}} - \{x\} \neq \emptyset
            }. Let \mt{y \in \closure{\{x\}} - \{x\} }, then every neighbourhood of y must contain \mt{x}, this is a contradiction to the second condition, so the \mt{T_{1}} axiom must hold.
      \end{proof}

      \item If \mt{A \subseteq \mathbb{X}}, we define the \defineNewWord{boundary}\label{def:Boundary} of \mt{A} by the equation
      \begin{equation*}
            \text{Bd} A = \closure{A} \cap \closure{\mathbb{X}-A}
      \end{equation*}
      \begin{enumerate}
            \item Show that \mt{\text{Int}A} and \mt{\text{Bd} A} are disjoint and \mt{
                  \closure{A} = \text{Int} A \cup \text{Bd} A
            }.
            
            \begin{proof}
                  For every \mt{ x \in \boundary{A} }, every open set contain \mt{x} must intersect \mt{A} and \mt{\mathbb{X}-A} so, there is no open set \mt{U} contain \mt{x}, \mt{U \subseteq A}.

                  For every \mt{x' \in \interior{A}}, there exists \mt{U' \subseteq A}, so \mt{\boundary{A}} and \mt{\interior{A}} are disjoint sets.

                  For every \mt{x \in \closure{A}}, \mt{x \in \boundary{A}} or \mt{x \notin \boundary{A}}. We discuss the condition that \mt{x \notin \boundary{A}}.

                  Then \mt{x \notin \closure{\mathbb{X}-A}}, then there exists a open set \mt{U} containing \mt{x}, that does not intersect with \mt{\mathbb{X}-A}. Thus \mt{U \subseteq A}, thus \mt{x \in \interior{A}}. So \mt{
                        \closure{A} \subseteq \text{Int} A \cup \text{Bd} A
                  }.

                  Then, \mt{\boundary{A} \subseteq \closure{A}}, \mt{\interior{A} \subseteq A \subseteq \closure{A}}. Thus, \mt{
                        \closure{A} \supseteq \text{Int} A \cup \text{Bd} A
                  }

                  So, \mt{
                        \closure{A} = \text{Int} A \cup \text{Bd} A
                  }
            \end{proof}

            \item Show that \mt{\boundary{A} = \emptyset} \ioi \mt{A} is both open and closed.
            
            \begin{proof}
                  So, \mt{\interior{A} = \closure{A}}, then \mt{\boundary{A} = \emptyset} follows directly from \mt{
                        \closure{A} = \text{Int} A \cup \text{Bd} A
                  }.
            \end{proof}

            \item Show that \mt{U} is open \ioi \mt{\boundary{U} = \closure{U}-U}.
            
            \begin{proof}
                  Suppose U is open. Then \mt{\closure{\mathbb{X}-U} = \mathbb{X} - U}. Then for every \mt{x \in U}, \mt{x \notin \mathbb{X} - U, x \notin \closure{\mathbb{X}-U}}. Thus \mt{\closure{U} \cap \closure{\mathbb{X}-U}=\closure{U}-U}.

                  Conversely, suppose \mt{\boundary{U} = \closure{U}-U}. Then for every \mt{x \in U}, \mt{x \notin \boundary{U}}. Then as \mt{\closure{U} = \interior{U}\cup \boundary{U}}, \mt{x \in \interior{U}}. So \mt{\interior{U} \supseteq U}. Thus \mt{U = \interior{U}}. Thus, \mt{U} is open.
            \end{proof}

      \end{enumerate}
\end{enumerate}



      \section{Continuous Function}

\begin{definition}[continuous]\label{def:Continuous}\footnote{
      As the continuity of a function is different as the topological spaces are different. So if we want to emphasis this fact, we say that \mt{f} is continuous \defineNewWord{relative}\label{def:ContinuousRelativeTo} to specific topologies on \mtb{X} and \mtb{Y}.
}
      Let \mtb{X} and \mtb{Y} be topological spaces. A function \mt{f: \mathbb{X}\rightarrow \mathbb{Y}} is said to be \defineNewWord{continuous} if for each open subset \mt{V} of \tmb{Y}, the set \mt{f^{-1}(V)} is an open subset of \mtb{X}.
\end{definition}

\begin{theorem}
      Let \mtb{X} and \mtb{Y} be topological spaces; let \mt{f: \mathbb{X}\rightarrow\mathbb{Y}}. Then the following are equivalent.
      \begin{enumerate}
            \item \mt{f} is continuous.
            \item For every subset \mt{A} of \mt{X}, one has \mt{f(\closure{A})\subseteq\closure{f(A)}}.
            \item For every closed set \mt{B} of \mtb{Y}, the set \mt{f^{-1}(B)} is closed in \mtb{X}.
            \item For each \mt{x\in\mathbb{X}} and each neighbourhood of \mt{V} of \mt{f(x)}, there is a neighbourhood \mt{U} of \mt{x} such that \mt{f(U) \subseteq V}.
      \end{enumerate}
\end{theorem}

\begin{proof}

      \hspace{1em}

      1 \mt{\Rightarrow} 3:

      Let \mt{A} be a open set in \mtb{Y}. \mt{f^{-1}(\mathbb{Y}-A) = \mathbb{X} - f^{-1}(A)}.
      
      \vspace{1em}

      3 \mt{\Rightarrow} 1:

      Let \mt{A} be a closed set in \mtb{Y}. \mt{f^{-1}(\mathbb{Y}-A) = \mathbb{X} - f^{-1}(A)}.

      \vspace{1em}

      1 \mt{\Rightarrow} 2:

      For \mt{x \in \closure{A}}, we take a open set \mt{f(x) \in U \subseteq \mathbb{Y}}. Thus \mt{x \in f^{-1}(U) \cap A \neq \emptyset}. Thus \mt{U \cap f(A) \neq \emptyset}. So \mt{f(x) \in \closure{f(A)}}. Thus \mt{f(\closure{A})\subseteq\closure{f(A)}}.

      \vspace{1em}

      2 \mt{\Rightarrow} 3:

      Suppose \mt{f} is not continuous. Then there must exists \mt{V}, such that \mt{f^{-1}(V) = U} is not closed. Thus \mt{\closure{U} \supset B = f^{-1}(A)}. Thus \mt{f{\closure{B}} \supset A}. However \mt{f(\closure{B}) \subseteq \closure{f(B)} = A}. There is a contradiction. So \mt{f} must be continuous.

      \vspace{1em}

      1 \mt{\Rightarrow} 4:

      For every neighbourhood \mt{V} of \mt{f(x)}, \mt{f^{-1}(V)} is a neighbourhood of \mt{x} that \mt{f(f^{-1}(V)) \subseteq V}.

      \vspace{1em}

      4 \mt{\Rightarrow} 1:

      We take a open set \mt{V} of \mtb{Y}. Let \mt{S} be the collection of all open set \mt{U} in \mtb{X} such that \mt{f(U) \subseteq V}. The set cannot be empty unless \mt{f^{-1}(V) = \emptyset}. Let \mt{U_{0}} denote the union of all the element in \mt{S}. We prove that \mt{U_{0} = f^{-1}(V)}.

      For all element \mt{x \in U_{0}}, \mt{f(x) \in V}. Thus \mt{U_{0} \subseteq f^{-1}(V)}.

      For all element \mt{x \in f^{-1}(V)}. There is a \mt{U'} such that \mt{x \in U', f(U')\subseteq V}. This follows from the condition 4. Thus \mt{U' \in S}. Thus \mt{x \in U_{0}}. Thus \mt{U_{0} \subseteq f^{-1}(V)}. As \mt{U_{0}} is union of open set, \mt{U_{0}} is also open. Thus, \mt{f^{-1}(V)} is also open.
      
      Thus \mt{f} is continuous.
\end{proof}

\begin{definition}[homeomorphism]\label{def:Homeomorphism}\footnote{
      A equivalent way to define homeomorphism, is that for any open subset \mt{U} of \mtb{X}, \mt{f(U)} is open \ioi \mt{U} is open.
}
      Let \mtb{X} and \mtb{Y} be topological space; let \mt{f: \mathbb{X} \rightarrow \mathbb{Y}} be a bijection. If both the function \mt{f} and the inverse function
      \begin{equation*}
            f^{-1}: \mathbb{Y} \rightarrow \mathbb{X}
      \end{equation*}
      are continuous, then f is called a \defineNewWord{homeomorphism}
\end{definition}

\begin{definition}[topological imbedding]\label{def:TopologicalImbedding}
      Suppose that \mt{f: \mathbb{X} \rightarrow \mathbb{Y}} is an injective continuous map, where \mtb{X} and \mtb{Y} are topological spaces. Let \mtb{Z} be the image set \mt{f(\mathbb{X})}, considered as a subspace of \mtb{Y}; then the function \mt{f': \mathbb{X} \rightarrow \mathbb{Z}} obtained by restricting the range of \mt{f} is bijective. If \mt{f'} happens to be a homeomorphism of \mtb{X} with \mtb{Z}, we say that the map \mt{f: \mathbb{X} \rightarrow \mathbb{Y}} is a \defineNewWord{topological imbedding}, or simply an \defineNewWord{imbedding}, of \mtb{X} in \mtb{Y}.
\end{definition}

\begin{theorem}[Rules for constructing continuous functions]\label{theorem:RulesForConstructingContinuousFunctions}
      Let \mtb{X}, \mtb{Y}, and \mtb{Z} be topological spaces.
      \begin{enumerate}
            \item (Constant function) If \mt{f: \mathbb{X} \rightarrow \mathbb{Y}} maps all of \mtb{X} into the single point \mt{y_{0}} of \mtb{Y}, then \mt{f} is continuous.
            \item (Inclusion) If \mt{A} is a subspace of \mtb{X}, the inclusion function \mt{j: A \rightarrow \mathbb{X}} is continuous.
            \item (Composites) If \mt{f: \mathbb{X}\rightarrow \mathbb{Y}} and \mt{g:\mathbb{Y}\rightarrow\mathbb{Z}} are continuous, then the map \mt{g \circ f: \mathbb{X} \rightarrow \mathbb{Z}} is continuous.
            \item (Restricting the domain) If \mt{f: \mathbb{X} \rightarrow \mathbb{Y}} is continuous, and if \mt{A} is a subspace of \mtb{X}, then the restriction function \mt{f|A : A \rightarrow \mathbb{Y}} is continuous.
            \item (Restricting or expanding the range) Let \mt{f:\mathbb{X}\rightarrow\mathbb{Y}} is continuous. Let \mtb{Z} be a subspace of \mtb{Y} containing the image \mt{f(\mathbb{X})}, the function \mt{h: \mathbb{X}\rightarrow \mathbb{Z}} obtained by restricting the range of \mt{f} is continuous. If \mtb{Z} is a space having \mtb{Y} as a subspace, then the function \mt{h: \mathbb{X}\rightarrow\mathbb{Y}} obtained by expanding the range of \mt{f} is continuous.
            \item (Local formulation of continuity) The map \mt{f: \mathbb{X}\rightarrow\mathbb{Y}} is continuous if \mtb{X} can be written as the union of open sets \mt{U_{\alpha}} such set \mt{f|U_{\alpha}} is continuous for each \mt{\alpha}
      \end{enumerate}
\end{theorem}

\begin{proof}

      \hspace{1em}

      \begin{enumerate}
            \item \mt{f^{-1}(U)} of any open set \mt{U} is \mtb{X}, thus \mt{f} is continuous.
            \item For every open subset \mt{U} of \mtb{X}, \mt{j^{-1}(U) = U\cap A} is continuous in \mt{A}. Thus \mt{j} is a continuous function.
            \item For every open subset \mt{U} of \mtb{Z}, \mt{f^{-1}(U)} is open in \mtb{Y}, and \mt{g^{-1}(f^{-1}(U))} is open in \mtb{X}. Thus, \mt{g \circ f} is continuous
            \item For every open subset \mt{U} of \mtb{Y}, \mt{f^{-1}(U)} is open in \mtb{X}, thus \mt{f^{-1}(U)\cap A} is open in \mt{A}. Thus the function \mt{f|A} is continuous.
            \item If \mtb{Z} is a subspace of \mtb{Y}, then every open subset of \mtb{Z} can be represented as \mt{U\cap\mathbb{Z}}, where \mt{U} is a open subset of \mtb{Y}. Thus \mt{h^{-1}(U\cap\mathbb{Z})=g^{-1}(\mathbb{Z})\cap g^{-1}(U) = \mathbb{X}\cap g^{-1}(U)} which is a open subset of \mt{X}, thus \mt{h} is continuous.
            
            If \mtb{Y} is a subspace of \mtb{Z}. Then we take a open subset \mt{U} of \tmb{Z}. \mt{h^{-1}(U) = g^(-1)(U\cap \mathbb{Y}) } which is open in \mtb{X}, thus \mt{h} is continuous.

            \item if \mt{f|U_{\alpha}} is continuous for each \mt{\alpha}. For every open subset \mt{U} of \mtb{Y}.
            \begin{equation*}
                  U = \cup_{\alpha} (U_{\alpha}\cap U)
            \end{equation*}
            where \mt{U_{\alpha}\cap U} is open both in \mt{U_{\alpha}} and in \mtb{Y}. 

            Thus,
            \begin{eqnarray*}
                  f^{-1}(U) &=& f^{-1}(\cup_{\alpha} (U_{\alpha}\cap U)) \\
                  &=& \cup_{\alpha} ((f|U_{\alpha})^{-1}(U_{\alpha}\cap U))
            \end{eqnarray*}
            and each \mt{(f|U_{\alpha})^{-1}(U_{\alpha}\cap U)} is open, thus \mt{f^{-1}(U)} is open.
      \end{enumerate}
\end{proof}

\begin{theorem}[The pasting lemma]\label{theorem:ThePastingLemma}\footnote{
      The proof of this theorem is similar to the "Local formulation of continuity" condition of "Rules for constructing continuous functions", so we omit the proof here.
}
      Let \mt{
            \mathbb{X} = A \cup B
      }, where \mt{A,B} are closed in \mtb{X}. Let \mt{f: A\rightarrow \mathbb{Y}} and \mt{g: B \rightarrow \mathbb{Y}} be continuous. If \mt{f(x)=g(x)} for every \mt{x\in A\cap B}, then \mt{f,g} combine to give a continuous function \mt{h: \mathbb{X}\rightarrow\mathbb{Y}}, defined by setting \mt{h(x)=f(x),x\in A} and \mt{h(x)=g(x),x\in B}.
\end{theorem}

\begin{theorem}[Maps into products]\label{theorem:MapsIntoProducts} \footnote{
      The map \mt{f_{1},f_{2}} are called the \defineNewWord{coordinate functions}\label{def:CoordinateFunctions} of \mt{f}
}
      Let \mt{f: A \rightarrow \mathbb{X}\times\mathbb{Y}} be given by the equation
      \begin{equation*}
            f(a) = (f_{1}(a),f_{2}(a))
      \end{equation*}

      Then, the function \mt{f} is continuous \ioi the functions
      \begin{equation*}
            f_{1}: A \rightarrow \mathbb{X}, f_{2}: A \rightarrow \mathbb{Y}
      \end{equation*}
      are continuous.
\end{theorem}

\begin{proof}
      Let \mt{\pi_{1},\pi_{2}} be the projection function
      \begin{eqnarray*}
            \pi_{1}&:& \mathbb{X}\times\mathbb{Y} \rightarrow \mathbb{X} \\
            \pi_{2}&:& \mathbb{X}\times\mathbb{Y} \rightarrow \mathbb{Y}
      \end{eqnarray*}

      \vspace{1em}

      We first proof that if \mt{U} is an open subset of \productSet{\mathbb{X}}{\mathbb{Y}},
      \begin{equation*}
            f^{-1}(U) = f_{1}^{-1}(\pi_{1}(U)) \cap f_{2}^{-1}(\pi_{2}(U))
      \end{equation*}

      Let \mt{x \times y \in U}, \mt{f^{-1}(x \times y)} contains all \mt{a} such that \mt{f(a) = x \times y}.

      Then for any \mt{a \in f^{-1}(x \times y)}, \mt{a \in f_{1}^{-1}(\pi_{1}(x \times y)), a \in f_{2}^{-1}(\pi_{2}(x \times y))}.

      Thus, \mt{f^{-1}(x \times y) \subseteq f_{1}^{-1}(\pi_{1}(x \times y)) \cap f_{2}^{-1}(\pi_{2}(x \times y))}.

      Thus \mt{f^{-1}(U) \subseteq f_{1}^{-1}(\pi_{1}(U)) \cap f_{2}^{-1}(\pi_{2}(U))}.

      \vspace{1em}

      Also, if \mt{a \in f_{1}^{-1}(\pi_{1}(x \times y)), a \in f_{2}^{-1}(\pi_{2}(x \times y))}, \mt{f_{1}(a) = x, f_{2}(a) = y}.

      Thus \mt{f(a) = x \times y}.
      Thus \mt{a \in f^{-1}(x \times y)}.

      Thus \mt{f^{-1}(U) = f_{1}^{-1}(\pi_{1}(U)) \cap f_{2}^{-1}(\pi_{2}(U))}

      \vspace{1em}

      Let \mt{U} be any open subset of \productSet{\mathbb{X}}{\mathbb{Y}}

      \begin{equation*}
            f^{-1}(U) = f_{1}^{-1}(\pi_{1}(U)) \cap f_{2}^{-1}(\pi_{2}(U))
      \end{equation*}

      Where \mt{f_{1}^{-1}(\pi_{1}(U))} and \mt{f_{2}^{-1}(\pi_{2}(U))} are both open set. Thus \mt{f^{-1}(U)} is open.
\end{proof}
      \subsection{Exercise}

\begin{enumerate}
      \item Give an counter example why \mt{
            \closure{\cup A_{\alpha}} = \cup \closure{A_{\alpha}}
      } dose not hold.

      \begin{proof}
            Consider the X be the K-topology on the real line.
      
            Let
            \begin{eqnarray*}
                  A_{n} &=& (\frac{1}{n+1},\frac{1}{n}), n \in \mathbb{Z}_{+} \\
                  A &=& \cup A_{n}
            \end{eqnarray*}
      
            Then 
            \begin{eqnarray*}
                  \closure{A_{n}} &=& [\frac{1}{n+1},\frac{1}{n}] \\
                  \cup \closure{A_{n}} &=& (0,1]
            \end{eqnarray*}
      
            However, as every neighbourhood of \mt{0} intersect \mt{\cup A_{\alpha}}. \mt{0 \in \closure{\cup A_{\alpha}}}.
      
            Thus, \mt{
                  \closure{\cup A_{\alpha}} \neq \cup \closure{A_{\alpha}}
            }
      \end{proof}

      \item Prove that 
      \begin{equation*}
            \closure{A-B} \supseteq \closure{A} - \closure{B}
      \end{equation*}

      \begin{proof}
            If \mt{x \in \closure{A} - \closure{B}}. Then
            \begin{eqnarray*}
                  x \in \closure{A}, x \notin \closure{B}
            \end{eqnarray*}.

            Thus for open set \mt{U} containing \mt{x}
            \begin{eqnarray*}
                  &\exists& U_{1} \cap B = \emptyset \\
                  &\forall& U \cap A \neq \emptyset
            \end{eqnarray*}

            Suppose that \mt{x \notin \closure{A-B}}. Then
            \begin{eqnarray*}
                  \exists U_{0} \cap (A-B) = \emptyset
            \end{eqnarray*}

            Thus,
            \begin{eqnarray*}
                  U_{0} \cap A \subseteq B
            \end{eqnarray*}

            Thus,
            \begin{eqnarray*}
                  U_{1} \cap B &=& \emptyset \\
                  U_{1} \cap U_{0} \cap A &=& \emptyset
            \end{eqnarray*}

            As \mt{U_{1} \cap U_{0}} is an open set containing \mt{x}, so there is contradiction with \mt{x \in \closure{A}}. Thus \mt{x \in \closure{A-B}}.
      \end{proof}

      \item A \defineNewWord{diagonal}\label{def:Diagonal} is a subset \mt{
            \Delta = \{ x \times x | x \in \mathbb{X} \}
      } of the product topology \mtb{X \times X} where \mtb{X} is a topological space. Show that the diagonal is closed in \mtb{X \times X} \ioi \mtb{X} is a Hausdorff space.

      \begin{proof}
            If \mtb{X} is a Hausdorff space. For every element \productSet{x}{y} of \productSet{\mathbb{X}}{\mathbb{X}} that not in \mt{\Delta}. We take disjoint set \mt{U_{x},U_{y}} where \mt{
                  x \in U_{x}, y \in U_{y}
            }. Then \mt{
                  \mathbb{X} \times \mathbb{X} - \Delta = \cup_{x \neq y} U_{x} \times U_{y}
            }. Where \mt{\cup_{x \neq y} U_{x} \times U_{y}} is an open set. Thus \mt{\Delta} is a closed set.

            Conversely, if \mt{\Delta} is a closed set, suppose that \mtb{X} is not a Hausdorff space. Then there exists distinct \mt{x, y} such that every neighbourhood of \mt{x} and \mt{y} intersect. Let \mtb{B} be a basis of topology of \mtb{X}. Then \mt{
                  x \times y \in \mathbb{X} \times \mathbb{X} - \Delta
            }. However we cannot find \mt{
                  B_{1}, B_{2} \in \mathbb{B}, x \times y \in B_{1} \times B_{2} \subset \mathbb{X} \times \mathbb{X} - \Delta
            }. Then \mt{\Delta} is not a closed set. So there is a contradiction, then \mtb{X} must be a Hausdorff space.
      \end{proof}

      \item Prove that \mt{T_{1}} axiom is equivalent to the condition such that for every distinct pair \mt{x,y} of \mtb{X}, there exists neighbourhood of \mt{x} does not contain \mt{y}.
      
      \begin{proof}
            First if \mt{T_{1}} axiom hold, then for every pair \mt{x,y}, the neighbourhood \mt{\mathbb{X}-\{y\}} of \mt{x} does not contain \mt{y}, so the second condition hold.

            Conversely, if the second condition hold. Suppose that we can find a finite points set say \mt{\{x_{1}, x_{2}, x_{3} \dots}\}, then there must exists \mt{x \in \{x_{1}, x_{2}, x_{3} \dots}\} such that the set \mt{\{x\}} is not closed. Then \mt{
                  \closure{\{x\}} - \{x\} \neq \emptyset
            }. Let \mt{y \in \closure{\{x\}} - \{x\} }, then every neighbourhood of y must contain \mt{x}, this is a contradiction to the second condition, so the \mt{T_{1}} axiom must hold.
      \end{proof}

      \item If \mt{A \subseteq \mathbb{X}}, we define the \defineNewWord{boundary}\label{def:Boundary} of \mt{A} by the equation
      \begin{equation*}
            \text{Bd} A = \closure{A} \cap \closure{\mathbb{X}-A}
      \end{equation*}
      \begin{enumerate}
            \item Show that \mt{\text{Int}A} and \mt{\text{Bd} A} are disjoint and \mt{
                  \closure{A} = \text{Int} A \cup \text{Bd} A
            }.
            
            \begin{proof}
                  For every \mt{ x \in \boundary{A} }, every open set contain \mt{x} must intersect \mt{A} and \mt{\mathbb{X}-A} so, there is no open set \mt{U} contain \mt{x}, \mt{U \subseteq A}.

                  For every \mt{x' \in \interior{A}}, there exists \mt{U' \subseteq A}, so \mt{\boundary{A}} and \mt{\interior{A}} are disjoint sets.

                  For every \mt{x \in \closure{A}}, \mt{x \in \boundary{A}} or \mt{x \notin \boundary{A}}. We discuss the condition that \mt{x \notin \boundary{A}}.

                  Then \mt{x \notin \closure{\mathbb{X}-A}}, then there exists a open set \mt{U} containing \mt{x}, that does not intersect with \mt{\mathbb{X}-A}. Thus \mt{U \subseteq A}, thus \mt{x \in \interior{A}}. So \mt{
                        \closure{A} \subseteq \text{Int} A \cup \text{Bd} A
                  }.

                  Then, \mt{\boundary{A} \subseteq \closure{A}}, \mt{\interior{A} \subseteq A \subseteq \closure{A}}. Thus, \mt{
                        \closure{A} \supseteq \text{Int} A \cup \text{Bd} A
                  }

                  So, \mt{
                        \closure{A} = \text{Int} A \cup \text{Bd} A
                  }
            \end{proof}

            \item Show that \mt{\boundary{A} = \emptyset} \ioi \mt{A} is both open and closed.
            
            \begin{proof}
                  So, \mt{\interior{A} = \closure{A}}, then \mt{\boundary{A} = \emptyset} follows directly from \mt{
                        \closure{A} = \text{Int} A \cup \text{Bd} A
                  }.
            \end{proof}

            \item Show that \mt{U} is open \ioi \mt{\boundary{U} = \closure{U}-U}.
            
            \begin{proof}
                  Suppose U is open. Then \mt{\closure{\mathbb{X}-U} = \mathbb{X} - U}. Then for every \mt{x \in U}, \mt{x \notin \mathbb{X} - U, x \notin \closure{\mathbb{X}-U}}. Thus \mt{\closure{U} \cap \closure{\mathbb{X}-U}=\closure{U}-U}.

                  Conversely, suppose \mt{\boundary{U} = \closure{U}-U}. Then for every \mt{x \in U}, \mt{x \notin \boundary{U}}. Then as \mt{\closure{U} = \interior{U}\cup \boundary{U}}, \mt{x \in \interior{U}}. So \mt{\interior{U} \supseteq U}. Thus \mt{U = \interior{U}}. Thus, \mt{U} is open.
            \end{proof}

      \end{enumerate}
\end{enumerate}



      \section{Continuous Function}

\begin{definition}[continuous]\label{def:Continuous}\footnote{
      As the continuity of a function is different as the topological spaces are different. So if we want to emphasis this fact, we say that \mt{f} is continuous \defineNewWord{relative}\label{def:ContinuousRelativeTo} to specific topologies on \mtb{X} and \mtb{Y}.
}
      Let \mtb{X} and \mtb{Y} be topological spaces. A function \mt{f: \mathbb{X}\rightarrow \mathbb{Y}} is said to be \defineNewWord{continuous} if for each open subset \mt{V} of \tmb{Y}, the set \mt{f^{-1}(V)} is an open subset of \mtb{X}.
\end{definition}

\begin{theorem}
      Let \mtb{X} and \mtb{Y} be topological spaces; let \mt{f: \mathbb{X}\rightarrow\mathbb{Y}}. Then the following are equivalent.
      \begin{enumerate}
            \item \mt{f} is continuous.
            \item For every subset \mt{A} of \mt{X}, one has \mt{f(\closure{A})\subseteq\closure{f(A)}}.
            \item For every closed set \mt{B} of \mtb{Y}, the set \mt{f^{-1}(B)} is closed in \mtb{X}.
            \item For each \mt{x\in\mathbb{X}} and each neighbourhood of \mt{V} of \mt{f(x)}, there is a neighbourhood \mt{U} of \mt{x} such that \mt{f(U) \subseteq V}.
      \end{enumerate}
\end{theorem}

\begin{proof}

      \hspace{1em}

      1 \mt{\Rightarrow} 3:

      Let \mt{A} be a open set in \mtb{Y}. \mt{f^{-1}(\mathbb{Y}-A) = \mathbb{X} - f^{-1}(A)}.
      
      \vspace{1em}

      3 \mt{\Rightarrow} 1:

      Let \mt{A} be a closed set in \mtb{Y}. \mt{f^{-1}(\mathbb{Y}-A) = \mathbb{X} - f^{-1}(A)}.

      \vspace{1em}

      1 \mt{\Rightarrow} 2:

      For \mt{x \in \closure{A}}, we take a open set \mt{f(x) \in U \subseteq \mathbb{Y}}. Thus \mt{x \in f^{-1}(U) \cap A \neq \emptyset}. Thus \mt{U \cap f(A) \neq \emptyset}. So \mt{f(x) \in \closure{f(A)}}. Thus \mt{f(\closure{A})\subseteq\closure{f(A)}}.

      \vspace{1em}

      2 \mt{\Rightarrow} 3:

      Suppose \mt{f} is not continuous. Then there must exists \mt{V}, such that \mt{f^{-1}(V) = U} is not closed. Thus \mt{\closure{U} \supset B = f^{-1}(A)}. Thus \mt{f{\closure{B}} \supset A}. However \mt{f(\closure{B}) \subseteq \closure{f(B)} = A}. There is a contradiction. So \mt{f} must be continuous.

      \vspace{1em}

      1 \mt{\Rightarrow} 4:

      For every neighbourhood \mt{V} of \mt{f(x)}, \mt{f^{-1}(V)} is a neighbourhood of \mt{x} that \mt{f(f^{-1}(V)) \subseteq V}.

      \vspace{1em}

      4 \mt{\Rightarrow} 1:

      We take a open set \mt{V} of \mtb{Y}. Let \mt{S} be the collection of all open set \mt{U} in \mtb{X} such that \mt{f(U) \subseteq V}. The set cannot be empty unless \mt{f^{-1}(V) = \emptyset}. Let \mt{U_{0}} denote the union of all the element in \mt{S}. We prove that \mt{U_{0} = f^{-1}(V)}.

      For all element \mt{x \in U_{0}}, \mt{f(x) \in V}. Thus \mt{U_{0} \subseteq f^{-1}(V)}.

      For all element \mt{x \in f^{-1}(V)}. There is a \mt{U'} such that \mt{x \in U', f(U')\subseteq V}. This follows from the condition 4. Thus \mt{U' \in S}. Thus \mt{x \in U_{0}}. Thus \mt{U_{0} \subseteq f^{-1}(V)}. As \mt{U_{0}} is union of open set, \mt{U_{0}} is also open. Thus, \mt{f^{-1}(V)} is also open.
      
      Thus \mt{f} is continuous.
\end{proof}

\begin{definition}[homeomorphism]\label{def:Homeomorphism}\footnote{
      A equivalent way to define homeomorphism, is that for any open subset \mt{U} of \mtb{X}, \mt{f(U)} is open \ioi \mt{U} is open.
}
      Let \mtb{X} and \mtb{Y} be topological space; let \mt{f: \mathbb{X} \rightarrow \mathbb{Y}} be a bijection. If both the function \mt{f} and the inverse function
      \begin{equation*}
            f^{-1}: \mathbb{Y} \rightarrow \mathbb{X}
      \end{equation*}
      are continuous, then f is called a \defineNewWord{homeomorphism}
\end{definition}

\begin{definition}[topological imbedding]\label{def:TopologicalImbedding}
      Suppose that \mt{f: \mathbb{X} \rightarrow \mathbb{Y}} is an injective continuous map, where \mtb{X} and \mtb{Y} are topological spaces. Let \mtb{Z} be the image set \mt{f(\mathbb{X})}, considered as a subspace of \mtb{Y}; then the function \mt{f': \mathbb{X} \rightarrow \mathbb{Z}} obtained by restricting the range of \mt{f} is bijective. If \mt{f'} happens to be a homeomorphism of \mtb{X} with \mtb{Z}, we say that the map \mt{f: \mathbb{X} \rightarrow \mathbb{Y}} is a \defineNewWord{topological imbedding}, or simply an \defineNewWord{imbedding}, of \mtb{X} in \mtb{Y}.
\end{definition}

\begin{theorem}[Rules for constructing continuous functions]\label{theorem:RulesForConstructingContinuousFunctions}
      Let \mtb{X}, \mtb{Y}, and \mtb{Z} be topological spaces.
      \begin{enumerate}
            \item (Constant function) If \mt{f: \mathbb{X} \rightarrow \mathbb{Y}} maps all of \mtb{X} into the single point \mt{y_{0}} of \mtb{Y}, then \mt{f} is continuous.
            \item (Inclusion) If \mt{A} is a subspace of \mtb{X}, the inclusion function \mt{j: A \rightarrow \mathbb{X}} is continuous.
            \item (Composites) If \mt{f: \mathbb{X}\rightarrow \mathbb{Y}} and \mt{g:\mathbb{Y}\rightarrow\mathbb{Z}} are continuous, then the map \mt{g \circ f: \mathbb{X} \rightarrow \mathbb{Z}} is continuous.
            \item (Restricting the domain) If \mt{f: \mathbb{X} \rightarrow \mathbb{Y}} is continuous, and if \mt{A} is a subspace of \mtb{X}, then the restriction function \mt{f|A : A \rightarrow \mathbb{Y}} is continuous.
            \item (Restricting or expanding the range) Let \mt{f:\mathbb{X}\rightarrow\mathbb{Y}} is continuous. Let \mtb{Z} be a subspace of \mtb{Y} containing the image \mt{f(\mathbb{X})}, the function \mt{h: \mathbb{X}\rightarrow \mathbb{Z}} obtained by restricting the range of \mt{f} is continuous. If \mtb{Z} is a space having \mtb{Y} as a subspace, then the function \mt{h: \mathbb{X}\rightarrow\mathbb{Y}} obtained by expanding the range of \mt{f} is continuous.
            \item (Local formulation of continuity) The map \mt{f: \mathbb{X}\rightarrow\mathbb{Y}} is continuous if \mtb{X} can be written as the union of open sets \mt{U_{\alpha}} such set \mt{f|U_{\alpha}} is continuous for each \mt{\alpha}
      \end{enumerate}
\end{theorem}

\begin{proof}

      \hspace{1em}

      \begin{enumerate}
            \item \mt{f^{-1}(U)} of any open set \mt{U} is \mtb{X}, thus \mt{f} is continuous.
            \item For every open subset \mt{U} of \mtb{X}, \mt{j^{-1}(U) = U\cap A} is continuous in \mt{A}. Thus \mt{j} is a continuous function.
            \item For every open subset \mt{U} of \mtb{Z}, \mt{f^{-1}(U)} is open in \mtb{Y}, and \mt{g^{-1}(f^{-1}(U))} is open in \mtb{X}. Thus, \mt{g \circ f} is continuous
            \item For every open subset \mt{U} of \mtb{Y}, \mt{f^{-1}(U)} is open in \mtb{X}, thus \mt{f^{-1}(U)\cap A} is open in \mt{A}. Thus the function \mt{f|A} is continuous.
            \item If \mtb{Z} is a subspace of \mtb{Y}, then every open subset of \mtb{Z} can be represented as \mt{U\cap\mathbb{Z}}, where \mt{U} is a open subset of \mtb{Y}. Thus \mt{h^{-1}(U\cap\mathbb{Z})=g^{-1}(\mathbb{Z})\cap g^{-1}(U) = \mathbb{X}\cap g^{-1}(U)} which is a open subset of \mt{X}, thus \mt{h} is continuous.
            
            If \mtb{Y} is a subspace of \mtb{Z}. Then we take a open subset \mt{U} of \tmb{Z}. \mt{h^{-1}(U) = g^(-1)(U\cap \mathbb{Y}) } which is open in \mtb{X}, thus \mt{h} is continuous.

            \item if \mt{f|U_{\alpha}} is continuous for each \mt{\alpha}. For every open subset \mt{U} of \mtb{Y}.
            \begin{equation*}
                  U = \cup_{\alpha} (U_{\alpha}\cap U)
            \end{equation*}
            where \mt{U_{\alpha}\cap U} is open both in \mt{U_{\alpha}} and in \mtb{Y}. 

            Thus,
            \begin{eqnarray*}
                  f^{-1}(U) &=& f^{-1}(\cup_{\alpha} (U_{\alpha}\cap U)) \\
                  &=& \cup_{\alpha} ((f|U_{\alpha})^{-1}(U_{\alpha}\cap U))
            \end{eqnarray*}
            and each \mt{(f|U_{\alpha})^{-1}(U_{\alpha}\cap U)} is open, thus \mt{f^{-1}(U)} is open.
      \end{enumerate}
\end{proof}

\begin{theorem}[The pasting lemma]\label{theorem:ThePastingLemma}\footnote{
      The proof of this theorem is similar to the "Local formulation of continuity" condition of "Rules for constructing continuous functions", so we omit the proof here.
}
      Let \mt{
            \mathbb{X} = A \cup B
      }, where \mt{A,B} are closed in \mtb{X}. Let \mt{f: A\rightarrow \mathbb{Y}} and \mt{g: B \rightarrow \mathbb{Y}} be continuous. If \mt{f(x)=g(x)} for every \mt{x\in A\cap B}, then \mt{f,g} combine to give a continuous function \mt{h: \mathbb{X}\rightarrow\mathbb{Y}}, defined by setting \mt{h(x)=f(x),x\in A} and \mt{h(x)=g(x),x\in B}.
\end{theorem}

\begin{theorem}[Maps into products]\label{theorem:MapsIntoProducts} \footnote{
      The map \mt{f_{1},f_{2}} are called the \defineNewWord{coordinate functions}\label{def:CoordinateFunctions} of \mt{f}
}
      Let \mt{f: A \rightarrow \mathbb{X}\times\mathbb{Y}} be given by the equation
      \begin{equation*}
            f(a) = (f_{1}(a),f_{2}(a))
      \end{equation*}

      Then, the function \mt{f} is continuous \ioi the functions
      \begin{equation*}
            f_{1}: A \rightarrow \mathbb{X}, f_{2}: A \rightarrow \mathbb{Y}
      \end{equation*}
      are continuous.
\end{theorem}

\begin{proof}
      Let \mt{\pi_{1},\pi_{2}} be the projection function
      \begin{eqnarray*}
            \pi_{1}&:& \mathbb{X}\times\mathbb{Y} \rightarrow \mathbb{X} \\
            \pi_{2}&:& \mathbb{X}\times\mathbb{Y} \rightarrow \mathbb{Y}
      \end{eqnarray*}

      \vspace{1em}

      We first proof that if \mt{U} is an open subset of \productSet{\mathbb{X}}{\mathbb{Y}},
      \begin{equation*}
            f^{-1}(U) = f_{1}^{-1}(\pi_{1}(U)) \cap f_{2}^{-1}(\pi_{2}(U))
      \end{equation*}

      Let \mt{x \times y \in U}, \mt{f^{-1}(x \times y)} contains all \mt{a} such that \mt{f(a) = x \times y}.

      Then for any \mt{a \in f^{-1}(x \times y)}, \mt{a \in f_{1}^{-1}(\pi_{1}(x \times y)), a \in f_{2}^{-1}(\pi_{2}(x \times y))}.

      Thus, \mt{f^{-1}(x \times y) \subseteq f_{1}^{-1}(\pi_{1}(x \times y)) \cap f_{2}^{-1}(\pi_{2}(x \times y))}.

      Thus \mt{f^{-1}(U) \subseteq f_{1}^{-1}(\pi_{1}(U)) \cap f_{2}^{-1}(\pi_{2}(U))}.

      \vspace{1em}

      Also, if \mt{a \in f_{1}^{-1}(\pi_{1}(x \times y)), a \in f_{2}^{-1}(\pi_{2}(x \times y))}, \mt{f_{1}(a) = x, f_{2}(a) = y}.

      Thus \mt{f(a) = x \times y}.
      Thus \mt{a \in f^{-1}(x \times y)}.

      Thus \mt{f^{-1}(U) = f_{1}^{-1}(\pi_{1}(U)) \cap f_{2}^{-1}(\pi_{2}(U))}

      \vspace{1em}

      Let \mt{U} be any open subset of \productSet{\mathbb{X}}{\mathbb{Y}}

      \begin{equation*}
            f^{-1}(U) = f_{1}^{-1}(\pi_{1}(U)) \cap f_{2}^{-1}(\pi_{2}(U))
      \end{equation*}

      Where \mt{f_{1}^{-1}(\pi_{1}(U))} and \mt{f_{2}^{-1}(\pi_{2}(U))} are both open set. Thus \mt{f^{-1}(U)} is open.
\end{proof}
      \subsection{Exercise}

\begin{enumerate}
      \item Give an counter example why \mt{
            \closure{\cup A_{\alpha}} = \cup \closure{A_{\alpha}}
      } dose not hold.

      \begin{proof}
            Consider the X be the K-topology on the real line.
      
            Let
            \begin{eqnarray*}
                  A_{n} &=& (\frac{1}{n+1},\frac{1}{n}), n \in \mathbb{Z}_{+} \\
                  A &=& \cup A_{n}
            \end{eqnarray*}
      
            Then 
            \begin{eqnarray*}
                  \closure{A_{n}} &=& [\frac{1}{n+1},\frac{1}{n}] \\
                  \cup \closure{A_{n}} &=& (0,1]
            \end{eqnarray*}
      
            However, as every neighbourhood of \mt{0} intersect \mt{\cup A_{\alpha}}. \mt{0 \in \closure{\cup A_{\alpha}}}.
      
            Thus, \mt{
                  \closure{\cup A_{\alpha}} \neq \cup \closure{A_{\alpha}}
            }
      \end{proof}

      \item Prove that 
      \begin{equation*}
            \closure{A-B} \supseteq \closure{A} - \closure{B}
      \end{equation*}

      \begin{proof}
            If \mt{x \in \closure{A} - \closure{B}}. Then
            \begin{eqnarray*}
                  x \in \closure{A}, x \notin \closure{B}
            \end{eqnarray*}.

            Thus for open set \mt{U} containing \mt{x}
            \begin{eqnarray*}
                  &\exists& U_{1} \cap B = \emptyset \\
                  &\forall& U \cap A \neq \emptyset
            \end{eqnarray*}

            Suppose that \mt{x \notin \closure{A-B}}. Then
            \begin{eqnarray*}
                  \exists U_{0} \cap (A-B) = \emptyset
            \end{eqnarray*}

            Thus,
            \begin{eqnarray*}
                  U_{0} \cap A \subseteq B
            \end{eqnarray*}

            Thus,
            \begin{eqnarray*}
                  U_{1} \cap B &=& \emptyset \\
                  U_{1} \cap U_{0} \cap A &=& \emptyset
            \end{eqnarray*}

            As \mt{U_{1} \cap U_{0}} is an open set containing \mt{x}, so there is contradiction with \mt{x \in \closure{A}}. Thus \mt{x \in \closure{A-B}}.
      \end{proof}

      \item A \defineNewWord{diagonal}\label{def:Diagonal} is a subset \mt{
            \Delta = \{ x \times x | x \in \mathbb{X} \}
      } of the product topology \mtb{X \times X} where \mtb{X} is a topological space. Show that the diagonal is closed in \mtb{X \times X} \ioi \mtb{X} is a Hausdorff space.

      \begin{proof}
            If \mtb{X} is a Hausdorff space. For every element \productSet{x}{y} of \productSet{\mathbb{X}}{\mathbb{X}} that not in \mt{\Delta}. We take disjoint set \mt{U_{x},U_{y}} where \mt{
                  x \in U_{x}, y \in U_{y}
            }. Then \mt{
                  \mathbb{X} \times \mathbb{X} - \Delta = \cup_{x \neq y} U_{x} \times U_{y}
            }. Where \mt{\cup_{x \neq y} U_{x} \times U_{y}} is an open set. Thus \mt{\Delta} is a closed set.

            Conversely, if \mt{\Delta} is a closed set, suppose that \mtb{X} is not a Hausdorff space. Then there exists distinct \mt{x, y} such that every neighbourhood of \mt{x} and \mt{y} intersect. Let \mtb{B} be a basis of topology of \mtb{X}. Then \mt{
                  x \times y \in \mathbb{X} \times \mathbb{X} - \Delta
            }. However we cannot find \mt{
                  B_{1}, B_{2} \in \mathbb{B}, x \times y \in B_{1} \times B_{2} \subset \mathbb{X} \times \mathbb{X} - \Delta
            }. Then \mt{\Delta} is not a closed set. So there is a contradiction, then \mtb{X} must be a Hausdorff space.
      \end{proof}

      \item Prove that \mt{T_{1}} axiom is equivalent to the condition such that for every distinct pair \mt{x,y} of \mtb{X}, there exists neighbourhood of \mt{x} does not contain \mt{y}.
      
      \begin{proof}
            First if \mt{T_{1}} axiom hold, then for every pair \mt{x,y}, the neighbourhood \mt{\mathbb{X}-\{y\}} of \mt{x} does not contain \mt{y}, so the second condition hold.

            Conversely, if the second condition hold. Suppose that we can find a finite points set say \mt{\{x_{1}, x_{2}, x_{3} \dots}\}, then there must exists \mt{x \in \{x_{1}, x_{2}, x_{3} \dots}\} such that the set \mt{\{x\}} is not closed. Then \mt{
                  \closure{\{x\}} - \{x\} \neq \emptyset
            }. Let \mt{y \in \closure{\{x\}} - \{x\} }, then every neighbourhood of y must contain \mt{x}, this is a contradiction to the second condition, so the \mt{T_{1}} axiom must hold.
      \end{proof}

      \item If \mt{A \subseteq \mathbb{X}}, we define the \defineNewWord{boundary}\label{def:Boundary} of \mt{A} by the equation
      \begin{equation*}
            \text{Bd} A = \closure{A} \cap \closure{\mathbb{X}-A}
      \end{equation*}
      \begin{enumerate}
            \item Show that \mt{\text{Int}A} and \mt{\text{Bd} A} are disjoint and \mt{
                  \closure{A} = \text{Int} A \cup \text{Bd} A
            }.
            
            \begin{proof}
                  For every \mt{ x \in \boundary{A} }, every open set contain \mt{x} must intersect \mt{A} and \mt{\mathbb{X}-A} so, there is no open set \mt{U} contain \mt{x}, \mt{U \subseteq A}.

                  For every \mt{x' \in \interior{A}}, there exists \mt{U' \subseteq A}, so \mt{\boundary{A}} and \mt{\interior{A}} are disjoint sets.

                  For every \mt{x \in \closure{A}}, \mt{x \in \boundary{A}} or \mt{x \notin \boundary{A}}. We discuss the condition that \mt{x \notin \boundary{A}}.

                  Then \mt{x \notin \closure{\mathbb{X}-A}}, then there exists a open set \mt{U} containing \mt{x}, that does not intersect with \mt{\mathbb{X}-A}. Thus \mt{U \subseteq A}, thus \mt{x \in \interior{A}}. So \mt{
                        \closure{A} \subseteq \text{Int} A \cup \text{Bd} A
                  }.

                  Then, \mt{\boundary{A} \subseteq \closure{A}}, \mt{\interior{A} \subseteq A \subseteq \closure{A}}. Thus, \mt{
                        \closure{A} \supseteq \text{Int} A \cup \text{Bd} A
                  }

                  So, \mt{
                        \closure{A} = \text{Int} A \cup \text{Bd} A
                  }
            \end{proof}

            \item Show that \mt{\boundary{A} = \emptyset} \ioi \mt{A} is both open and closed.
            
            \begin{proof}
                  So, \mt{\interior{A} = \closure{A}}, then \mt{\boundary{A} = \emptyset} follows directly from \mt{
                        \closure{A} = \text{Int} A \cup \text{Bd} A
                  }.
            \end{proof}

            \item Show that \mt{U} is open \ioi \mt{\boundary{U} = \closure{U}-U}.
            
            \begin{proof}
                  Suppose U is open. Then \mt{\closure{\mathbb{X}-U} = \mathbb{X} - U}. Then for every \mt{x \in U}, \mt{x \notin \mathbb{X} - U, x \notin \closure{\mathbb{X}-U}}. Thus \mt{\closure{U} \cap \closure{\mathbb{X}-U}=\closure{U}-U}.

                  Conversely, suppose \mt{\boundary{U} = \closure{U}-U}. Then for every \mt{x \in U}, \mt{x \notin \boundary{U}}. Then as \mt{\closure{U} = \interior{U}\cup \boundary{U}}, \mt{x \in \interior{U}}. So \mt{\interior{U} \supseteq U}. Thus \mt{U = \interior{U}}. Thus, \mt{U} is open.
            \end{proof}

      \end{enumerate}
\end{enumerate}



\end{document}