\subsubsection{Exercise}

\begin{enumerate}
      \item A map $ \mathit{f} : \mathbb{X} \rightarrow \mathbb{Y} $ is said to be a \defineNewWord{open map}\label{def:OpenMap} if for every open set $ U \subseteq \mathbb{X} $, the set $ \mathit{f}(U) $ is open in \mtb{Y}. Show that $ \pi : \mathbb{X} \times \mathbb{Y} \rightarrow \mathbb{X} $ is open map.
      
      \begin{proof}
            An open set in \productSet{\mathbb{X}}{\mathbb{Y}} can be represented by 
            \begin{equation*}
                  \cup( U_{i} \times U_{i}' )
            \end{equation*}
            where $ U_{i}, U_{i}' $ are open sets in \mtb{X}, \mtb{Y}, respectively.

            Also,
            \begin{equation*}
                  \cup( U_{i} \times U_{i}' ) = \cup( U_{i} ) \times \cup( U_{i}' )
            \end{equation*}

            Thus,
            \begin{equation*}
                  \pi(\cup( U_{i} \times U_{i}' )) = \cup( U_{i} )
            \end{equation*}

            Thus, $ \pi(U) $ is open in \mtb{X}.
      \end{proof}

      \item Let \mtb{X} and \mtb{X'} denote a single set in the topologies \mtb{T} and \mtb{T'}, respectively; let \mtb{Y} and \mtb{Y'} denote a single set in the topologies \mtb{U} and \mtb{U'}, respectively. \footnote{
            what does \mtb{X}, \mtb{X'}, \mtb{Y}, \mtb{Y'} really mean here?? I do not know, so I just put the exercise here without a proof.
      } Assume these sets are nonempty.
      \begin{enumerate}
            \item Show that if \mt{ \mathbb{T}' \supseteq \mathbb{T} } and \mt{ \mathbb{U}' \supseteq \mathbb{U} }, then the product topologies \productSet{\mathbb{X'}}{\mathbb{Y'}} is finer than the product topology on \productSet{\mathbb{X}}{\mathbb{Y}}.
            \item Does the converse of the previous statement hold?
      \end{enumerate}

      \item Show that the countable collection\footnote{
            The prove of this set is countable is typically similar to Cantor's enumeration of a countable collection of countable sets.
      }
      \begin{equation*}
            \{ 
                  (a,b) \times (c,d) | a < b, c < d, a \in \mathbb{Q}, b \in \mathbb{Q}, c \in \mathbb{Q}, d \in \mathbb{Q}
            \}
      \end{equation*}
      is a basis for \mt{\mathbb{R}^{2}}

      \begin{proof}
            This is obvious if you prove that $ (a,b) \times (c,d) $ is a rectangle in the \mt{\mathbb{R}^{2}} plane.
      \end{proof}

      \item Let \mtb{X} be an ordered set. If \mtb{Y} is a proper subset of \mtb{X} that is convex in \mtb{X} prove that \mtb{Y} may not be an interval or a ray in \mtb{X}.
      
      \begin{proof}
            Let \mt{\mathbb{X} = \mathbb{R}^{2}} with dictionary order. Then \mt{
                  Y = \{
                        (x,y) | -1 \le x \le 1 
                  \} 
            } is convex in \mtb{X}, however it is not an interval or a ray.
      \end{proof}

      There is a false prove given by myself.

      \begin{proof}
            Let \mtb{S} be a set that contain all intervals and rays of \mtb{Y}. We define a partial order on \mtb{S} by inclusion. So if there is a chain in \mtb{S}:
            \begin{equation*}
                  S_{1} \subseteq S_{2} \subseteq S_{3} \dots
            \end{equation*}

            Let 
            \begin{equation*}
                  S = S_{1} \cup S_{2} \cup S_{3} \cup \dots
            \end{equation*}

            Thus, \mt{S} is an upper bound of the chain.

            Thus, by Zorn's Lemma, there is a maximal element of \mtb{S}, say \mt{U}, then we prove that \mt{ U = \mathbb{Y} }.

            If \mt{ U \neq \mathbb{Y} }, then \mt{
                  \exists x, x \in \mathbb{Y} - U
            }. 

            If \mt{U} is a ray say \mt{ (a,+\infty) }. Then \mt{ x < a }, thus \mt{
                  U \subseteq (x,+\infty) \subseteq \mathbb{B}
            }, then there is contradiction with the maximal element.

            If \mt{U} is an interval, the circumstance is similar with the proof of \mt{U} is a ray.

            Thus \mtb{Y} is a ray or an interval.
      \end{proof}

      However, there is issue with this proof, the set \mt{S} does exists. However, it may not be an interval or ray, so it may not be contained in \mtb{S}
      
\end{enumerate}
