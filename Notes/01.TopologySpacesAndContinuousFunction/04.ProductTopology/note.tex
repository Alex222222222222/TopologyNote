\section{The Product Topology}

\begin{definition}[product topology]\label{def:ProductTopology}
      Let \textmathbb{X} and \textmathbb{Y} be topological spaces. The \defineNewWord{product topology} on $ \mathbb{X} \times \mathbb{Y} $ having a basis \textmathbb{B} containing all sets of the form $ U \times V $, where $ U $ and $ V $ is open sets of \textmathbb{X} and \tmb{Y} respectively.
\end{definition}

\begin{theorem}\omitObviuos
      If \tmb{B} and \tmb{C} is basis for the topology of \tmb{X} and \tmb{Y} respectively, then the collection
      \begin{equation*}
            \mathbb{D} = \{ B \times C | B \in \mathbb{B} and C \in \mathbb{C} \}
      \end{equation*}
      is a basis for the topology of $ \mathbb{X} \times \mathbb{Y} $
\end{theorem}

\begin{definition}[projection]\label{def:Projection}
      Let $ \pi_{1}: \mathbb{X} \times \mathbb{Y} \rightarrow \mathbb{X} $ be defined by the equation:
      \begin{equation*}
            \pi_{1}(x,y) = x
      \end{equation*} 

      Let $ \pi_{2}: \mathbb{X} \times \mathbb{Y} \rightarrow \mathbb{Y} $ be defined by the equation:
      \begin{equation*}
            \pi_{1}(x,y) = y
      \end{equation*} 

      The maps $ \pi_{1} $ and $ \pi_{2} $ are called the \defineNewWord{projections} of $ \mathbb{X} \times \mathbb{Y} $ onto its first and second factors, respectively.
\end{definition}

\begin{theorem}\omitObviuos
      The collection
      \begin{equation*}
            \mathbb{S} = \{ \pi_{1}^{-1}(U) | U open in \mathbb{X} \} \cup \{ \pi_{2}^{-1}(V) | V open in \mathbb{Y} \}
      \end{equation*}
      is a subbasis for the product topology on $ \mathbb{X} \times \mathbb{Y} $.
\end{theorem}

\begin{definition}[box topology]\label{def:BoxTopology}
      Let,
      \begin{equation*}
            \mathbb{X} = \mathbb{X}_{1} \times \mathbb{X}_{2} \times \dots \times \mathbb{X}_{n} \text{or} \mathbb{X}_{1} \times \mathbb{X}_{2} \times \dots
      \end{equation*}

      In the first case, all the sets of the form \mt{
            U_{1} \times \dots \times U_{n}
      } where \mt{
            U_{i}
      } is a open set of \mt{
            \mathbb{X}_{i}
      } form a basis.

      In the second case, all the sets of the form \mt{
            U_{1} \times U_{2} \times \dots
      } where \mt{
            U_{i}
      } is a open set of \mt{
            \mathbb{X}_{i}
      } also form a basis.

      Topology defined in this way was called a \defineNewWord{box topology}.
\end{definition}

\begin{definition}[product topology]\label{def:ProductTopologyInfinite}\footnote{
      In the finite case, the product topology and box topology are the same, however they differ when \mtb{X} is a infinite cartesian product.
}
      Let,
      \begin{equation*}
            \mathbb{X} = \mathbb{X}_{1} \times \mathbb{X}_{2} \times \dots \times \mathbb{X}_{n} \text{or} \mathbb{X}_{1} \times \mathbb{X}_{2} \times \dots
      \end{equation*}

      Let \mt{\pi_{i}} be the projection function\footnote{
            This is also called a \defineNewWord{projection mapping}\label{def:ProjectionMapping} in a cartesian product.
      } that
      \begin{equation*}
            \pi_{i}: \mathbb{X} \rightarrow \mathbb{X}_{i}
      \end{equation*}

      And if \mt{x \in \mathbb{X}}
      \begin{equation*}
            \pi_{i}(x) = x_{i}
      \end{equation*}

      All the set of the form \mt{\pi_{i}^{-1}(U_{i})} where \mt{i} is arbitrary and \mt{U_{i}} is an open set of \mt{\mathbb{X}_{i}}, form a subbasis of \mtb{X}. The topology generated by this subbasis is called \defineNewWord{product topology}. And \mtb{X} is called a \defineNewWord{product space}.
\end{definition}

\begin{definition}[J-tuple]\label{def:JTuple}
      Let \mt{J} be an index set\footnote{
            A index set was the set \mt{
                  \{1,\dots,n\}
            } or the set \mt{
                  \mathbb{Z}_{+}
            }.
      }. Give a set \mtb{X}, a \defineNewWord{J-tuple} is defined as a function \mt{
            x: J \rightarrow \mathbb{X}
      }. If \mt{\alpha} is an element of \mt{J}, \mt{
            x(\alpha)
      } is often denoted by \mt{
            x_{\alpha}
      } and is called the \mt{\alpha\text{th}} \defineNewWord{coordinate} of \mt{x}. And the function \mt{x} itself is often denoted by the symbol
      \begin{equation*}
            (x_{\alpha})_{\alpha \in J}
      \end{equation*}

      The set of all J-tuples of elements of \mtb{X} is often denoted by \mt{
            \mathbb{X}^{J}
      }.
\end{definition}

\begin{definition}[cartesian product]\label{def:CartesianProduct}
      Let \mt{
            \{A_{\alpha}\}_{\alpha \in J}
      } be an indexed family of sets; let \mt{
            \mathbb{X} = \bigcup_{\alpha \in J} A_{\alpha}
      }. The \defineNewWord{cartesian product} of this indexed family is denoted by
      \begin{equation*}
            \displaystyle
            \prod_{\alpha \in J} A_{\alpha}
      \end{equation*}

      And is defined to be the set of all J-tuples \mt{
            (x_{\alpha})_{\alpha \in J} 
      } of elements of \mtb{X} such that \mt{
            x_{\alpha} \in A_{\alpha}
      } for each \mt{
            \alpha \in J
      }. That is, it is the set of all functions
      \begin{equation*}
            x: J \rightarrow \bigcup_{\alpha \in J} A_{\alpha}
      \end{equation*}
      such that \mt{
            x(\alpha) \in A_{\alpha}
      } for each \mt{
            \alpha \in J
      }.
\end{definition}

\begin{theorem}[Comparison of the box and product topologies]\label{theorem:ComparisonOfBoxProductTopology}\footnote{
      It is assumed that it is given product topology when considering \mt{
            \prod X_{\alpha}
      } unless it state specifically.
}
      The box topology on \mt{
            \prod \mathbb{X}_{\alpha}
      } has a basis all sets of the form \mt{
            \prod U_{\alpha}
      } where \mt{
            U_{\alpha}
      } is open in \mt{
            X_{\alpha}
      } for each \mt{\alpha}. The product topology on \mt{
            \prod \mathbb{X}_{\alpha}
      } has a basis all sets of the form \mt{
            \prod U_{\alpha}
      } where \mt{
            U_{\alpha}
      } is open in \mt{
            X_{\alpha}
      } for each \mt{\alpha} and \mt{
            U_{\alpha}
      } equals \mt{
            \mathbb{X}_{\alpha}
      } except for finitely many values of \mt{\alpha}.
\end{theorem}

\begin{theorem}\omitObviuos
      Suppose the topology on each space \mt{
            \mathbb{X}_{\alpha}
      } is given by a basis \mt{
            \mathbb{X}_{\alpha}
      }. The collection of all sets of the form
      \begin{equation*}
            \prod_{\alpha \in J} B_{\alpha}
      \end{equation*}
      where \mt{
            B_{\alpha} \in \mathbb{B}_{\alpha}
      } form a basis for the box topology on \mt{
            \prod _{\alpha \in J} \mathbb{X}_{\alpha}
      }.

      The collection of all sets of the same form, where \mt{
            B_{\alpha} \in \mathbb{B}_{\alpha}
      } for finitely many indices \mt{\alpha} and \mt{
            B_{\alpha} = \mathbb{X}_{\alpha}
      } for all the remaining indices, will form a basis for the product topology \mt{
            \prod_{\alpha \in J}\mathbb{X}_{\alpha}
      }.
\end{theorem}

\begin{theorem}\omitObviuos
      Let \mt{
            A_{\alpha}
      } be a subspace of \mt{
            \mathbb{X}_{\alpha}
      }, for each \mt{
            \alpha \in J
      }. Then \mt{
            \prod A_{\alpha}
      } is a subspace of \mt{
            \prod \mathbb{X}_{\alpha}
      } if both products are given the box topology, or if both products are given the product topology.
\end{theorem}

\begin{theorem}\omitObviuos
      If each space \mt{
            \mathbb{X}_{\alpha}
      } is a Hausdorff space, then \mt{
            \prod \mathbb{X}_{\alpha}
      } is a Hausdorff space in both the box and product topologies.
\end{theorem}

\begin{theorem}
      Let \mt{
            \{
                  \mathbb{X}_{\alpha}      
            \}
      } be an indexed family of spaces; let \mt{
            A_{\alpha} \subseteq \mathbb{X}_{\alpha}
      } for each \mt{\alpha}. If \mt{
            \prod \mathbb{X}_{\alpha}
      } is given either the product or the box topology, then
      \begin{equation*}
            \prod \closure{A_{\alpha}} = \closure{\prod A_{\alpha}}
      \end{equation*}
\end{theorem}

\begin{proof}
      Let \mt{\pi_{\alpha}} represent the projection mapping.

      Let \mt{x} be an element of \mt{
            \prod \mathbb{X}_{\alpha}
      }. Let \mt{V} be an open set in \mt{
            \prod \mathbb{X}_{\alpha}
      } that containing \mt{x}.

      If \mt{
            x \in \prod \closure{A_{\alpha}}
      }, then \mt{
            \pi_{\alpha}(V)
      } is a open set in \mt{
            \mathbb{X}_{\alpha}
      } that containing \mt{
            x_{\alpha}
      }. Thus \mt{
            \pi_{\alpha}(V)
      } intersect with \mt{
            A_{\alpha}
      }. Thus \mt{V} intersect with \mt{
            \prod A_{\alpha}
      }. Thus \mt{
            x \in \closure{\prod A_{\alpha}}
      }.

      If \mt{
            x \in \closure{\prod A_{\alpha}}
      }. Let \mt{
            U_{\alpha}
      } be an open set of \mt{
            A_{\alpha}
      } that contain \mt{
            x_{\alpha}
      }. Let \mt{
            V = \prod U_{\beta}
      } such that \mt{
            U_{\beta} = \begin{cases}
                  \mathbb{X}_{\beta}, & \beta \neq \alpha \\
                  U_{\alpha}, & \beta = \alpha
            \end{cases}
      }. It is obvious that \mt{V} is an open set that contain \mt{x}. Thus \mt{V} intersect with \mt{
            \prod A_{\alpha}
      }. Thus \mt{
            U_{\alpha}
      } intersect with \mt{
            A_{\alpha}
      }. Thus \mt{
            x \in \prod \closure{A_{\alpha}}
      }.
\end{proof}

\begin{theorem}
      Let \mt{
            f: A \rightarrow \prod_{\alpha \in J} \mathbb{X}_{\alpha}
      } be given by the equation
      \begin{equation*}
            f(a) = (
                  f_{\alpha}(a)
            )_{\alpha \in J}
      \end{equation*}
      where \mt{
            f_{\alpha} : A \rightarrow \mathbb{X}_{\alpha}
      } for each \mt{\alpha}. Let \mt{
            \prod \mathbb{X}_{\alpha}
      } have the product topology. Then the function \mt{f} is continuous if and only if each function \mt{
            f_{\alpha}
      } is continuous.
\end{theorem}

\begin{proof}
      Let \mt{
            \pi_{\alpha}
      } be the projection mapping

      It is obvious that
      \begin{equation*}
            f^{-1}(U) = \bigcap_{\alpha \in J} f_{\alpha}^{-1}(\pi_{\alpha}(U))
      \end{equation*}

      If \mt{f_{\alpha}} is continuous. Let \mt{V} be a closed set of \mt{
            \prod_{\alpha \in J} \mathbb{X}_{\alpha}
      }. Then \mt{
            \pi_{\alpha}(V)
      } is closed. Then \mt{
            f^{-1}(V)
      } is intersect of closed set. Thus \mt{
            \pi_{\alpha}(V)
      } is closed. So \mt{f} is continuous.

      If \mt{f} is continuous. Let \mt{
            U_{\alpha}
      } be an open set of \mt{
            \mathbb{X}_{\alpha}
      }. Let \mt{
            U_{\beta} = \mathbb{X}_{\beta}
      } if \mt{
            \beta \neq \alpha
      }. Let \mt{
            V = \prod_{\beta \in J} U_{\beta}
      }. It is obvious that \mt{V} is an open set of \mt{
            \prod \mathbb{X}_{\alpha}
      }. And 
      \begin{eqnarray*}
            f^{-1}{V} &=& \bigcap_{\alpha \in J} f_{\alpha}^{-1}(\pi_{\alpha}(U)) \\
            &=& f_{\alpha}^{-1}(U_{\alpha})
      \end{eqnarray*}
      which is an open set in \mt{A}. Thus \mt{f_{\alpha}} is continuous.
\end{proof}