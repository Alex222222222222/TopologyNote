\section{Metric Topology}

\begin{definition}[metric]\label{def:Metric}
      A \defineNewWord{metric} on a set \mtb{X} is a function
      \begin{equation*}
            d: \mathbb{X} \times \mathbb{X} \rightarrow \mathbb{R}
      \end{equation*}
      having the following properties:
      \begin{enumerate}
            \item \mt{
                  d(x,y) \geq 0
            } for all \mt{
                  x,y \in \mathbb{X}
            }; equality hold if and only if \mt{
                  x = y
            }

            \item \mt{
                  d(x,y) = d(y,x), \forall x,y \in \mathbb{X}
            }

            \item (Triangle Inequality) \mt{
                  d(x,y) + d(y,z) \geq d(x,z), \forall x,y,z \in \mathbb{X}
            }
      \end{enumerate}

      Given a metric \mt{d} on \mtb{X}, the number \mt{
            d(x,y)
      } is often called the \defineNewWord{distance}\label{def:Distance} between \mt{x} and \mt{y} in the metric \mt{d}.
\end{definition}

\begin{definition}[\mt{\epsilon}-ball centered at \mt{x}]\label{def:EpsilonBallCenteredAtX}\footnote{
      When no confusion will arise, the metric \mt{d} may be omit in \mt{
            B_{d}(x,\epsilon)
      }
}
      Given metric \mt{d} on a set \mtb{X} and \mt{\epsilon > 0}. The set
      \begin{equation*}
            B_{d}(x,\epsilon) = \{
                  y | d(x,y) < \epsilon
            \}
      \end{equation*}
      is called \mt{\epsilon}-ball centered at \mt{x}.
\end{definition}

\begin{definition}[metric topology]\label{def:MetricTopology}
      If \mt{d} is a metric on the set \mtb{X}, then the collection of all \mt{\epsilon}-balls \mt{
            B_{d}(x,\epsilon)
      }, such that \mt{
            x \in \mathbb{X}
      } and \mt{
            \epsilon > 0
      }, is a basis for a topology on \mtb{X}, called the \defineNewWord{metric topology} induced by \mt{d}.
\end{definition}

\begin{definition}[metrizable]\label{def:Metrizable}
      If \mtb{X} is topological space, \mtb{X} is said to be \defineNewWord{metrizable} if there exists a metric \mt{d} on the set \mtb{X} that induces the topology of \mtb{X}. A \defineNewWord{metric space}\label{def:MetricSpace} is a metrizable space \mtb{X} together with a specific metric \mt{d} that gives the topology of \mtb{X}.
\end{definition}

\begin{definition}[bounded]\label{def:Bounded}
      Let \mtb{X} be a metric space with metric \mt{d}. A subset \mt{A} of \mtb{X} is said to be \defineNewWord{bounded} if there is some number \mt{M} such that
      \begin{equation*}
            d(a_{1},a_{2}) \leq M
      \end{equation*}
      for every pair \mt{a_{1}} and \mt{a_{2}} if points of \mt{A}.
\end{definition}

\begin{definition}[diameter]\label{def:Diameter}
      Let \mtb{X} be a metric space with metric \mt{d}. Let \mt{A} be a bounded subset of \mtb{X}. Then \defineNewWord{diameter} is defined to be
      \begin{equation*}
            \diam A = \sup\{
                  d(a_{1},a_{2}) | a_{1}, a_{2} \in A      
            \}
      \end{equation*}
\end{definition}

\begin{theorem}
      Let \mtb{X} be a metric space with metric \mt{d}. Define \mt{
            \overline{d}: \mathbb{X} \times \mathbb{X} \rightarrow \mathbb{R}
      } by the equation
      \begin{equation*}
            \overline{d}(x,y) = min\{
                  d(x,y),1
            \}
      \end{equation*}

      Then \mt{
            \overline{d}
      } is a metric that induces the same topology as \mt{d}.

      The metric \mt{
            \overline{d}
      } is called the \defineNewWord{standard bounded metric}\label{def:StandardBoundedMetric} corresponding to \mt{d}
\end{theorem}

\begin{proof}
      It is obvious that \mt{
            \overline{d}
      } is a metric.

      To prove that \mt{d} and \mt{
            \overline{d}
      } induces the same topology, it is suffice to prove that for all \mt{a \in X} and \mt{\epsilon > 0} there exists \mt{
            \{a_{\alpha}\}
      } and \mt{
            \{\epsilon_{\alpha}\}
      } where \mt{
            \epsilon_{\alpha} \leq 1
      } such that
      \begin{equation*}
            B_{d}(a,\epsilon) = \bigcup B_{\overline{d}}(a_{\alpha},\epsilon_{\alpha})
      \end{equation*}

      For every \mt{
            x \in B_{d}(a,\epsilon)
      } take \mt{
            a_{x} = x 
      } and \mt{
            \epsilon_{x} < min(\epsilon - d(a,x),1)
      }. Then
      \begin{equation*}
            B_{d}(a,\epsilon) \supseteq B_{\overline{d}}(a_{x},\epsilon_{x})
      \end{equation*}
      as for all \mt{
            y \in B_{\overline{d}}(a_{x},\epsilon_{x})
      }
      \begin{eqnarray*}
            d(a,y) &\leq& d(a,a_{x}) + d(a_{x},y) \\
            &<& min(\epsilon - d(a,x),1) + d(a,a_{x}) \\
            &\leq& \epsilon
      \end{eqnarray*}

      Thus
      \begin{equation*}
            B_{d}(a,\epsilon) = \bigcup_{
                  x \in B_{d}(a,\epsilon)
            } B_{\overline{d}}(a_{x},\epsilon_{x})
      \end{equation*}
\end{proof}

\begin{definition}[norm]\label{def:Norm}
      Given \mt{
            x = (x_{1},\dots,x_{n})
      } in \mt{
            \mathbb{R}^{n}
      }. The \defineNewWord{norm} of \mt{x} is defined by the equation
      \begin{equation*}
            \norm{x} = (
                  x_{1}^{2}+\dots+x_{n}^{2}
            )^{\frac{1}{2}}
      \end{equation*}
\end{definition}

\begin{definition}[euclidean metric]\label{def:EuclideanMetric}
      The euclidean metric \mt{d} on \mt{
            \mathbb{R}^{n}
      } is defined by
      \begin{equation*}
            d(x,y) = \norm{x-y}
      \end{equation*}
\end{definition}

\begin{definition}[square metric]\label{def:SquareMetric}
      The square metric \mt{
            \rho
      } on \mt{
            \mathbb{R}^{n}
      } is defined by
      \begin{equation*}
            \rho(x,y) = max\{
                  |x_{1}-y_{1}|,\dots,|x_{n}-y_{n}|
            \}
      \end{equation*}
\end{definition}

\begin{lemma}
      Let \mt{d} and \mt{d'} be two metrics on the set \mtb{X}; let \mtb{T} and \mtb{T'} be the topology induced by \mt{d} and \mt{d'} respectively. Then \mtb{T'} is finer than \mt{T} if and only if for all \mt{
            x \in \mathbb{X}
      } and \mt{
            \epsilon > 0
      }, there exists a \mt{
            \delta > 0
      } such that
      \begin{equation*}
            B_{d'}(x,\delta) \subseteq B_{d}(x,\epsilon)
      \end{equation*}
\end{lemma}

\begin{proof}
      If \mtb{T'} is finer than \mtb{T}. Then for all \mt{
            B_{d}(x,\epsilon)
      } there exists a open set \mt{U} that containing \mt{x} such that \mt{
            U \subseteq B_{d}(x,\epsilon)
      }. As \mt{
            \{{B_{d'}(x,\delta)}\}
      } is a basis of \mt{T'}, then there exists \mt{
            B_{d'}(x,\delta) \subseteq U
      } that containing \mt{x}.

      If for all \mt{
            B_{d}(x,\epsilon)
      }, there exists \mt{
            B_{d'}(x,\delta) \subseteq B_{d}(x,\epsilon)
      }. Then as \mt{
            \{{B_{d'}(x,\epsilon)}\}
      } and \mt{
            \{{B_{d}(x,\epsilon)}\}
      } are both basis, then \mtb{T'} is finer than \mt{T}.
\end{proof}

\begin{theorem}\omitObviuos
      The topologies on \mt{
            \mathbb{R}^{n}
      } induced by the euclidean metric \mt{d} and the square metric \mt{\rho} are the same as the product topology on \mt{
            \mathbb{R}^{n}
      }.
\end{theorem}

\begin{definition}[uniform metric, uniform topology]\label{def:UniformMetric}
      Given an index set \mt{J}, and given points \mt{
            x = (x_{\alpha})_{\alpha \in J}
      } and \mt{
            y = (y_{\alpha})_{\alpha \in J}
      } of \mt{
            \mathbb{R}^{J}
      }, let us define a metric \mt{
            \overline{\rho}
      } on \mt{
            \mathbb{R}^{J}
      } by the equation
      \begin{equation*}
            \overline{\rho}(x,y) = \sup\{
                  \overline{d}(x_{\alpha},y_{\alpha}) | \alpha \in J
            \}
      \end{equation*}
      where \mt{
            \overline{d}
      } is the standard bounded metric on \mtb{R}. \mt{
            \overline{\rho}
      } is called the \defineNewWord{uniform metric} on \mt{
            \mathbb{R}^{J}
      }, and the topology it induces is called the \defineNewWord{uniform topology}
\end{definition}

\begin{theorem} \omitObviuos
      The uniform topology on \mt{
            \mathbb{R}^{J}
      } is finer than the product topology and is coarser than the box topology.
\end{theorem}

\begin{theorem}
      Let \mt{
            \overline{d}(a,b) = \min\{
                  |a-b|,1
            \} 
      } be the standard bounded metric on \mtb{R}. If \mt{x} nad \mt{y} are two points of \mt{
            \mathbb{R}^{\omega}
      }, define
      \begin{equation*}
            D(x,y) = \sup\left\{
                  \frac{\overline{d}(x_{i},y_{i})}{i}
            \right\}
      \end{equation*}

      Then \mt{D} is a metric that induces the product topology on \mt{
            \mathbb{R}^{\omega}
      }
\end{theorem}

\begin{proof}
      The properties of a metric are satisfied trivially except for the triangle inequality, which is proved by noting that for all \mt{i},
      \begin{eqnarray*}
            \frac{\overline{d}(x_{i},z_{i})}{i} &\le& \frac{\overline{d}(x_{i},y_{i})}{i} + \frac{\overline{d}(y_{i},z_{i})}{i} \\
            &\le& D(x,y) + D(y,z)
      \end{eqnarray*}
      so that
      \begin{equation*}
            \sup\left\{
                  \frac{\overline{d}(x_{i},z_{i})}{i}
            \right\} \le D(x,y) + D(y,z)
      \end{equation*}

      The fact that \mt{D} gives the product topology requires a little more work. First, let \mt{U} be open in the metric topology and let \mt{
            x \in U
      }; we find an open set \mt{V} in the product topology such that \mt{
            x \in V \supseteq U
      }. Choose an \mt{
            \epsilon-ball
      } \mt{
            B_{D}(x,\epsilon)
      } lying in \mt{U}. Then choose \mt{N} large enough that \mt{
            \frac{1}{N} < \epsilon
      }. Finally, let \mt{V} be the basis element for the product topology
      \begin{equation*}
            V = (x_{1}-\epsilon,x_{1}+\epsilon) \times \dots \times (x_{N}-\epsilon,x_{N}+\epsilon) \times R \times R \times \dots
      \end{equation*}

      We assert that \mt{
            V \in B_{D}(x,\epsilon)
      }: Given any \mt{y} in \mt{
            \mathbb{R}^{\omega}
      }
      \begin{equation*}
            \frac{\overline{d}(x_{i},y_{i})}{i} \le \frac{1}{N}, \forall i \ge N
      \end{equation*}

      Therefore,
      \begin{equation*}
            D(x,y) \le \max\left\{
                  \frac{\overline{d}(x_{1},y_{1})}{1},\dots,\frac{\overline{d}(x_{N},y_{N})}{N},\frac{1}{N}
            \right\}
      \end{equation*}

      If \mt{y} is in \mt{V}, this expression is less than \mt{
            \epsilon
      }, so that \mt{
            V \subseteq B_{D}(x,\epsilon)
      }, as desired.

      Conversely, consider a basis element
      \begin{equation*}
            U = \prod_{i \in \mathbb{Z}_{+}} U_{i}
      \end{equation*}
      for the product topology, where \mt{
            U_{i}
      } is open in \mtb{R} for \mt{
            i = \alpha_{1},\dots,\alpha_{n}
      } and \mt{
            U_{i} = \mathbb{R}
      } for all other indices \mt{i}. Given \mt{
            x \in U
      }, we find an open set \mt{V} of the metric topology such that \mt{
            x \in V \supseteq U
      }. Choose an interval \mt{
            (x_{i}-\epsilon_{i},x_{i}+\epsilon_{i})
      } in \mtb{R} centered about \mt{
            x_{i}
      } and lying in \mt{
            U_{i}
      } for \mt{
            i = \alpha_{1},\dots,\alpha_{n}
      }; choose each \mt{
            \epsilon_{i} \le 1
      }. Then define
      \begin{equation*}
            \epsilon = \min \left\{
                  \frac{\epsilon_{i}}{i} | i = \alpha_{1},\dots,\alpha_{n}
            \right\}
      \end{equation*}

      We assert that
      \begin{equation*}
            x \in B_{D}(x,\epsilon) \subseteq U
      \end{equation*}

      Let \mt{y} be a point of \mt{
            B_{D}(x,\epsilon)
      }. Then for all \mt{i}
      \begin{equation*}
            \frac{\overline{d}(x_{i},y_{i})}{i} \le D(x,y) < \epsilon
      \end{equation*}

      Now if \mt{
            i = \alpha_{1},\dots,\alpha_{n}
      }, then \mt{
            \epsilon \le \frac{\epsilon_{i}}{i}
      }, so that \mt{
            \overline{d}(x_{i},y_{i}) < \epsilon_{i} \le 1
      }; it follows that \mt{
            |x_{i}-y_{i}| < \epsilon_{i}
      }. Therefore \mt{
            y \in \prod U_{i}
      }, as desired.

\end{proof}

\begin{definition}[Hilbert Cube]\label{def:HilbertCube}
      The set
      \begin{equation*}
            H = \prod_{
                  n \in \mathbb{Z}_{+}
            } [0,\frac{1}{n}]
      \end{equation*}
      is called \defineNewWord{
            Hilbert cube
      }
\end{definition}

\begin{definition}[$l^{2}$-topology]\label{def:L2Topology}
      Let \mtb{X} be the subset of \mt{
            \mathbb{R}^{\omega}
      } consisting of all sequences \mt{
            x
      } such that \mt{
            \sum x_{i}^{2}
      } converges.

      Then the formula
      \begin{equation*}
            d(x,y) = \left[
                  \sum_{i=1}^{\infty}(x_{i}-y_{i})^{2}
            \right]^{\frac{1}{2}}
      \end{equation*}
      defines a metric on \mtb{X}. The topology induced by \mt{d} is called the $l^{2}$-topology.
\end{definition}

\begin{definition}[countable basis at point \mt{x}]\label{def:CountableBasisAtPointX}
      A space is said to be have \defineNewWord{countable basis at point \mt{x}} if there is a countable collection \mt{
            \{
                  U_{n}
            \}_{n \in \mathbb{Z}_{+}}
      } of neighbourhoods of \mt{x} such that any neighbourhood \mt{U} of \mt{x} contains at least on of the sets \mt{
            U_{n}
      }. A space \mtb{X} that has a countable basis at each of its point is said to satisfy the \defineNewWord{
            first countability axiom
      }\label{def:FirstCountabilityAxiom}
\end{definition}

\begin{theorem}
      Let \mt{
            f: \mathbf{X} \rightarrow \mathbf{Y}
      } be metrizable with metric \mt{
            d_{\mathbf{X}}
      } and \mt{
            d_{\mathbf{Y}}
      }, respectively. Then continuity of \mt{f} is equivalent to the requirement that given \mt{
            x \in \mathbb{X}
      } and given \mt{
            \epsilon > 0
      }, there exists \mt{
            \delta > 0
      } such that
      \begin{equation*}
            d_{\mathbf{X}}(x,y) < \delta \implies d_{\mathbb{Y}}(f(x),f(y)) < \epsilon
      \end{equation*}
\end{theorem}

\begin{proof}
      Suppose \mt{f} is continuous. Given \mt{x} and \mt{
            \epsilon
      }, consider the set
      \begin{equation*}
            f^{-1}(B(f(x),\epsilon))
      \end{equation*}
      which is open in \mtb{X} and contains the point\mt{x}. It contains some \mt{
            \delta
      }-ball \mt{B(x,\delta)} centered at \mt{x}. If \mt{y} is in this \mt{
            \delta
      }-ball, then \mt{f(y)} is in this \mt{
            \delta
      }-ball as desired.

      Conversely, suppose that the \mt{
            \epsilon-\delta
      } condition is satisfied. Let \mt{V} be open in \mtb{Y}; we show that \mt{
            f^{-1}(V)
      } is open in \mtb{X}. Let \mt{x} be a point of the set \mt{
            f^{-1}(V)
      }. Since \mt{
            f(x) \in V
      } there is an \mt{
            \epsilon
      }-ball \mt{
            B(f(x),\epsilon)
      } centered at \mt{
            f(x)
      } and contained in \mt{V}. By the \mt{
            \epsilon-\delta
      } condition, there exists a \mt{
            \delta
      }-ball centered at \mt{x} such that \mt{
            f(B(x,\delta)) \subseteq B(f(x),\epsilon)
      }. Then \mt{
            B(x,\delta)
      } is a neighbourhood of \mt{x} contained in \mt{
            f^{-1}(V)
      }, so that \mt{
            f^{-1}(V)
      } is open, as desired.
\end{proof}

\begin{lemma}[The sequence lemma]\label{def:TheSequenceLemma}\omitObviuos
      Let \mtb{X} be a topological space; let \mt{
            A \subseteq \mathbb{X}
      } If there is a sequence of points of \mt{A} converging to \mt{x}, then \mt{
            x \in \closure{A}
      }, the converse holds if \mtb{X} is metrizable.
\end{lemma}

\begin{theorem}\omitObviuos
      Let \mt{
            f: \mathbb{X} \rightarrow \mathbb{Y}
      }. If the function \mt{f} is continuous, then for every convergent sequence \mt{
            x_{n} \rightarrow x
      }, the sequence \mt{
            f(x_{n})
      } converges to \mt{
            f(x)
      }. The converse holds if \mtb{X} is metrizable.
\end{theorem}

\begin{lemma}\omitObviuos
      The addition, subtraction, and multiplication operations are continuous functions from \mt{
            \mathbb{R} \times \mathbb{R}
      } into \mtb{R}; and the quotient operation is continuous function from \mt{
            \mathbb{R} \times (\mathbb{R} - \{0\})
      } into \mtb{R}.
\end{lemma}

\begin{theorem}\omitObviuos
      If \mtb{X} is a topological space, and if \mt{
            f,g: \mathbb{X} \rightarrow \mathbb{R}
      } are continuous functions, then \mt{
            f + g
      }, \mt{
            f - g
      } and \mt{
            f \cdot g
      } are continuous. If \mt{
            g(x) \neq 0
      } for all \mt{x}, then \mt{
            \frac{f}{g}
      } is continuous.
\end{theorem}

\begin{definition}[converge uniformly]\label{def:ConvergeUniformly}
      Let \mt{
            f_{n}: \mathbb{X} \rightarrow \mathbb{Y}
      } be a sequence of functions from the set \mtb{X} to the metric space \mtb{Y}. Let \mt{d} be the metric for \mtb{Y}. We say that the sequence \mt{
            (f_{n})
      } \defineNewWord{
            converges uniformly
      } to the function \mt{
            f : \mathbb{X} \rightarrow \mathbb{Y}
      } if given \mt{
            \epsilon > 0
      }, there exists an integer \mt{N} such that 
      \begin{equation*}
            d(f_{n}(x),f(x)) < \epsilon
      \end{equation*}
      for all \mt{
            n > N
      } and all \mt{
            x \in \mathbb{X}
      }
\end{definition}

\begin{theorem}[Uniform limit theorem]\label{def:UniformLimitTheorem}
      Let \mt{
            f_{n} : \mathbb{X} \rightarrow \mathbb{Y}
      } be a sequence of continuous functions from the topological space \mtb{X} to the metric space \mtb{Y}. If \mt{
            (f_{n})
      } converges uniformly to \mt{f}, then \mt{f} is continuous.
\end{theorem}

\begin{definition}[isometric imbedding]\label{def:IsometricImbedding}
      Let \mtb{X} and \mtb{Y} be metric spaces with metric \mt{
            d_{\mathbb{X}}
      } and \mt{
            d_{\mathbb{Y}}
      }, respectively. Let \mt{
            f: \mathbb{X} \rightarrow \mathbb{Y}
      } have the property that for every pair of points \mt{
            x_{1}
      }, \mt{
            x_{2}
      } of \mtb{X}, and 
      \begin{equation*}
            d_{\mathbb{Y}}(f(x_{1}),f(x_{2})) = d_{\mathbb{X}}(x_{1},x_{2})
      \end{equation*}

      \mt{f} is an topological imbedding and is called an \defineNewWord{isometric imbedding} of \mtb{X} in \mtb{Y}
\end{definition}