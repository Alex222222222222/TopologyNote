\documentclass[twoside]{article}

\usepackage[a4paper, marginparwidth=75pt]{geometry}

\usepackage{amssymb}
\usepackage{amsmath}
\usepackage{amsthm}

\usepackage{marginnote}

\newtheorem{definition}{Definition}[section]
\newtheorem{lemma}{Lemma}[section]

\title{Topology}
\author{Alex}

\newcommand{\textmathbb}[1]{$ \mathbb{#1} $}
\newcommand{\defineNewWorld}[1]{\textit{\textbf{#1}}}
\newcommand{\omitObviuos}{\footnote{We omit the proof of this lemma as it is obvious.}}

\begin{document}

      \tableofcontents

      \section{Basic Definition of Topology}

            \begin{definition}[topology]
                  A \textit{\textbf{topology}} on a set $ \mathbb{X} $ is a collection $ \mathbb{T} $ of subsets of $ \mathbb{X} $ having the following properties:

                  \begin{itemize}
                        \item $ \emptyset $ and $ \mathbb{X} $ are in $ \mathbb{T} $
                        \item The union of the elements of any sub collection of $ \mathbb{T} $ is in $ \mathbb{T} $
                        \item The intersection of the elements of any \textbf{finite} sub collection of $ \mathbb{T} $ is in $ \mathbb{T} $
                  \end{itemize} 
            \end{definition}

            \begin{definition}[topology space]
                  A \textit{\textbf{topological space}} is a set $ \mathbb{X} $ for which a topology $ \mathbb{T} $ has been specified.
            \end{definition}
            
            \begin{definition}[open set]
                  A \textit{\textbf{open set}} $ \mathbb{U} $ is a subset of $ \mathbb{X} $ that belongs to a topology $ \mathbb{T} $ of $ \mathbb{X} $.
            \end{definition}
            
            \begin{definition}[open sets]
                  A topology can also be called a \textit{\textbf{open sets}}
            \end{definition}

            \begin{definition}[discrete topology]
                  The set of all subsets of a set $ \mathbb{X} $ formed a topology called \textit{\textbf{discrete topology}}
            \end{definition}

            \begin{definition}[trivial topology]
                  The set consisting the set $ \mathbb{X} $ and $ \emptyset $ only formed a topology of $ \mathbb{X} $ called \textit{\textbf{trivial topology}}
            \end{definition}

            \begin{definition}[finite complement topology]
                  Let $ \mathbb{X} $ be a set. Let $ \mathbb{T}_{\mathit{f}} $ be the collection of all subsets $ \mathbb{U} $ of $ \mathbb{X} $ such that $ \mathbb{X} - \mathbb{U} $ either if a \textbf{finite} \marginpar{
                        The set $ \mathbb{U} $ can form a topology because of the definition of topology is intersection of finite sub collection. If this can be intersection of infinite sub collection, $ \mathbb{U} $ will not be a topology. 
                  } of is all of $ \mathbb{X} $. Then $ \mathbb{T}_{\mathit{f}} $ is a topology on $ \mathbb{X} $, called the \textit{\textbf{}}.
            \end{definition}

            \begin{definition}[finer, larger, strictly finer, strictly larger, coarser, smaller, strictly coarser, strictly smaller, comparable]
                  Let $ \mathbb{T} $ and $ \mathbb{T'} $ be two topology on a given set $ \mathbb{X} $. If $ \mathbb{T} $ is  a subset of $ \mathbb{T'} $, we say that $ \mathbb{T'} $ is \textit{\textbf{finer}} or \textit{\textbf{larger}} than $ \mathbb{T} $. If $ \mathbb{T} $ is a proper subset of $ \mathbb{T'} $, we say that $ \mathbb{T'} $ is \textit{\textbf{strictly finer}} or \textit{\textbf{strictly larger}}  than $ \mathbb{T} $.
                  We also say that $ \mathbb{T} $ is \textit{\textbf{coarser}} or \textit{\textbf{smaller}} or \textit{\textbf{strictly coarser}} or \textit{\textbf{strictly smaller}} than $ \mathbb{T'} $.
                  We say that $ \mathbb{T} $ and $ \mathbb{T'} $ is \textit{\textbf{comparable}} if either $ \mathbb{T} $ is a subset of $ \mathbb{T'} $ or $ \mathbb{T'} $ is a subset of $ \mathbb{T} $.
            \end{definition}

      \section{Basis for a Topology}

            \begin{definition}[basis]
                  If $ \mathbb{X} $ is a set, a \defineNewWorld{basis} for a topology on \textmathbb{X} is a collection \textmathbb{B} of subsets of \textmathbb{X} (called \defineNewWorld{basis elements}) such that:
                  \begin{itemize}
                        \item For each $ x \in \mathbb{X} $, there is at least one basis element $ B $ containing $ x $
                        \item If $ x $ belongs to the intersection of two basis elements $ B_{1} $ and $ B_{2} $, then there is another element $ x \in B_{3} \in \mathbb{B} $ such that $ B_{3} \subseteq B_{1} \cap B_{2} $
                  \end{itemize}
            \end{definition}

            \begin{definition}[topology generated by basis]
                  Let \textmathbb{B} be a basis on \textmathbb{X}. Let \textmathbb{U} be a set containing all subsets $ U $ of \textmathbb{X} such that for each element $ x \in U $, there is $ B \in \mathbb{B} $ that $ x \in B \subseteq U $.
                  Such \textmathbb{U} formed a topology on \textmathbb{X}, called \defineNewWorld{topology \textmathbb{T} generated by \textmathbb{B}}
            \end{definition}
            
            \begin{lemma}
                  Let \textmathbb{X} be a set. Let \textmathbb{B} be a basis for a topology \textmathbb{T} on \textmathbb{X}. Then \textmathbb{T} equals to the set of all possible unions of elements of \textmathbb{B}.
            \end{lemma}

            \begin{proof}
                  Let set \textmathbb{U} be the set of all possible unions of elements of \textmathbb{B}. For any $ U \in \mathbb{U} $. $ U = \cup B $ \marginpar{
                        Note that this expression may not be unique.
                  } for some $ B \in \mathbb{B} $. Thus, for every $ x \in U $, there exist a $ B' \in \mathbb{B} $ that $ x \in B' \subseteq U $. Thus, $ U \in \mathbb{T} $.

                  Conversely, for any $ U \in \mathbb{T} $. For any $ x \in U $, let $ x \in B_{x} \in U $. Then, $ U = \cup_{x \in U}B_{x} $. Thus, $ U \in \mathbb{U} $.

                  Therefore, \textmathbb{U} equals to \textmathbb{T}.
            \end{proof}

            \begin{lemma} \footnote{We omit the proof of this lemma as it is obvious.}
                  Let \textmathbb{X} be a topological space. Suppose that \textmathbb{C} is a collection of open sets of \textmathbb{X} such that for each open set $ U $ of \textmathbb{X} and each $ x \in U $, there is an element $ C \in \mathbb{C} $ such that $ x \in C \subseteq C $. Then \textmathbb{C} is a basis for the topology of \textmathbb{X}.
            \end{lemma}

            \begin{lemma} \footnote{We omit the proof of this lemma as it is obvious.}
                  Let \textmathbb{B} and \textmathbb{B'} be basis for the topologies \textmathbb{T} and \textmathbb{T'}, respectively, on \textmathbb{X}. Then the following are equivalent:
                  \begin{itemize}
                        \item \textmathbb{T'} is finer than \textmathbb{T}
                        \item For each $ x \in \mathbb{X} $ and each basis element $ B \in \mathbb{B} $ containing X, there is a basis element $ B' \in \mathbb{B'} $ such that $ x \in B' \subseteq B $.
                  \end{itemize}
            \end{lemma}

            \begin{definition}[standard topology on the real line]
                  Let be $ \mathbb{B} = \{ B | B = \{ x | a < x < b \}, a < b, a \in \mathbb{R}, b \in \mathbb{R} \} $. \textmathbb{B} formed a basis on real line. The topology generated by \textmathbb{B} is called the \defineNewWorld{standard topology on the real line} \marginpar{
                        Whenever we consider \textmathbb{R}, we shall suppose it is given this topology unless we specifically state otherwise.
                  } .
            \end{definition}

            \begin{definition}[lower limit topology on the real line]
                  Let be $ \mathbb{B} = \{ B | B = \{ x | a \le x < b \}, a < b, a \in \mathbb{R}, b \in \mathbb{R} \} $. \textmathbb{B} formed a basis on real line. The topology generated by \textmathbb{B} is called the \defineNewWorld{lower limit topology on the real line}. When \textmathbb{R} is given this topology,we denote it by $ \mathbb{R}_{l} $.
            \end{definition}

            \begin{definition}[K-topology on the real line]
                  Let be $ \mathbb{B} = \{ B | B = \{ x | a < x < b \}, a < b, a \in \mathbb{R}, b \in \mathbb{R} \} $. Let $ K = \{ x | x = \frac{1}{n}, n \in \mathbb{Z_{+}} \} $. $ \mathbb{B} \cup \{ B - K | B \in \mathbb{B} \} $ formed a basis on real line. The topology generated by \textmathbb{B} is called the \defineNewWorld{K-topology on the real line}. When \textmathbb{R} is given this topology,we denote it by $ \mathbb{R_{K}} $.
            \end{definition}

            \begin{lemma} \omitObviuos
                  The topologies $ \mathbb{R}_{l} $ and \textmathbb{R_{K}} is strictly finer than the standard topology on \textmathbb{R}.
            \end{lemma}

            \begin{lemma}
                  The topologies of $ \mathbb{R}_{l} $ and \textmathbb{R_{K}} is not comparable.
            \end{lemma}

            \begin{proof}
                  Let $ \mathbb{T}_{l} $ and \textmathbb{T_{K}} be topologies of $ \mathbb{R}_{l} $ and \textmathbb{R_{K}} respectively. Let $ K = \{ x | x = \frac{1}{n}, n \in \mathbb{Z_{+}} \} $.

                  We first proof that $ \mathbb{T}_{l} $ is not finer than \textmathbb{T_{K}}. Let $ U = \{ x | -1 < x < 1 \} - K, x = 0 $.
                  If there exist $ B = \{ x | a \le x < b \} \in \mathbb{T}_{l} $ such that $ x \in B \subseteq U $, then $ 0 < b < 1 $. Thus, there exist $ n \in \mathbb{Z_{+}} $ that $ 0 < \frac{1}{n} < b $. Thus $ B $ is not a subset of $ U $.

                  Then we proof that \textmathbb{T_{K}} is not finer than $ \mathbb{T}_{l} $. Let $ U' = \{ x | a' \le x < b' \} $.
                  If there exist $ B' = \{ x | a'' < x < b'' \} or \{ x | a'' < x < b'' \} - K $ such that $ {a'} \in B \subseteq U $.
                  Thus $ a'' < a < b'' $.
                  Thus there exist $ c $ that $ a'' < x < a, x \in B ,x \notin U' $. Thus $ B' \nsubseteq U' $.

                  Thus the topologies of $ \mathbb{R}_{l} $ and \textmathbb{R_{K}} is not comparable.
            \end{proof}

            \begin{definition}[subbasis]
                  A \defineNewWorld{subbasis} \textmathbb{S} for a topology on \textmathbb{X} is a collection of subsets of \textmathbb{X} whose union equals \textmathbb{X}. The \defineNewWorld{topology generated by the subbasis} \textmathbb{S} is defined to be the collection \textmathbb{T} \marginpar{
                        It is obvious that \textmathbb{T} is a topology, we just omit the proof here.
                  } of all unions of finite intersections of elements of \textmathbb{S}.
            \end{definition}

\end{document}